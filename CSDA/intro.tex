\section{Introduction}

% # Quadrupen

% ## Introduction

% - Interprétation géométrique et aspect algorithmique, pourquoi c’est bien:
%     - c’est générique
%     - c’est  précis
%     - dans quel cadre c’est bien
%         - problèmes de taille intermédiaire

% ## Arguments

% - résolution exacte
%     - intérêt pour l’interprétabilité
%     - voir la conclusion
% - point de vue unifiant
% - dualité
%     - qui permet le calcul de la borne
%     - on remplace le sous gradient par le worst case
% - créativité (attention surtout valable en 2d
% - stabilité
% - linéaire par morceau
% - ...
% - code
% - expériences
% - borne
%     - mais moins bonne que Fenchel
% - plus de sous gradient

% ## Contre-Arguments

% - rien de vraiment neuf
% - coût quadratique

% ## Conclusion

% - c’est un package qui devrait être utilisé pour interpréter les variables sélectionnées
%     - parcque c’est plus précis
%     - parce qu’en conséquences on gagne en précision sur le support, mais pas en terme de MSE

% ## Consigne CSDA

% - The journal requires that the manuscript contain either computational or data analysis component.  Papers which are purely theoretical are not appropriate.  Manuscripts describing simulation studies must
%     - 1. be thorough with regard to the choice of parameter settings,
%     - 2. not over-generalize the conclusions,
%     - 3. carefully describe the limitations of the simulation studies, and
%     - 4. should guide the user regarding when the recommended methods areappropriated.It is recommended that the author indicate why comparisons cannot be made theoretically.

% ## TODO

% - Reprendre l’intro
%     - une vue unifiante
%     - fondée sur une dualité alternative
%     - qui permet
%         - une vision géométrique
%             - créativité
%         - une analyse sans sous gradient
%         - solution exacte
%             - précision
%             - stabilité
%             - le calcul d’une borne
%     - analyse de la vitesse de convergence difficile
%         - donc nous faisons des simulations

Inferential statistics aim at drawing conclusions from data and from some kind
of assumption or prior information about the underlying distribution.  It is
well known that processes where data guide the choice of assumptions can lead to
paradoxes, resulting from overfitting issues, in particular in the large
dimensional setup when data are used to select explicative variables: the
seeming explanatory power of weakly relevant variables can be important when the
number of variables is similar to the number of data
points~\citep{Freedman83b,Ambroise02}.
In this context, we present here an unusual reformulation of variable selection
methods based on sparsity-inducing penalties: we show that these methods can be
interpreted as adaptive penalties, where adaptivity refers to the choice of the
eventual penalty from data.
However, contrary to the available reformulations so far, ours shows that the adaptation
of the sparsity-inducing penalty to data follows a disagreement principle, where
the least-favorable penalty is selected from data.

We believe that our new viewpoint can be instrumental in unified analyses of
algorithms and their derived estimators, and we show here two such examples:
first, we provide a generic algorithm for solving the data fitting problem;
then, we derive the general form of an optimality gap that may be used to
monitor convergence.
Our experimental section illustrates the effectiveness of the generic algorithm
motivated by our interpretation, when instanced on the elastic-net estimator:
the algorithm, which relies on solving linear systems
is accurate, and computationally efficient up to medium scale problems (thousands of
variables).
As a side experimental result, we show that solving problems with high
precision, as with the proposed approach, benefits to the performances, either
measured in terms of prediction accuracy or in terms of support error rate.


% % Coordinate descent as implemented in \mytexttt{glmnet} \citep{2009_JSS_Friedman}
% % is a versatile approach allowing to solve various sparse regression
% % problems in a simple and fast procedure. In a nutshell, it
% % consists in cycling through the parameters and updating each in turn
% % until convergence. Proximal methods are popular in the machine learning community since
% % they  are optimal  among the  first-order  techniques.   These methods
% % minimize  objective  functions with  a  nonsmooth  part by  successive
% % quadratic approximations. \mytexttt{SPAMS} (SPArse Modeling
% % Software) \citep{2012_FML_Bach} represents an effective  implementation  of  this
% % approach for sparse regression problems.  


% Section \ref{sec:robust} introduces our general robust regression formalization,
% which allows numerous variants that follow from the definition of the
% uncertainty set on the adversarial noise, thereby leading to different sparse
% regression problems.  
% Section \ref{sec:quadra} fully details the derivations for the Lasso and the
% group-Lasso (using the $\ell_{\infty,1}$ mixed-norm) problems, applied together
% with an $\ell_2$ ridge penalty (leading to what is known as the elastic net for
% the Lasso).
% %
% A description of the general-purpose active set algorithm derived from this
% formalism is given in Section \ref{sec:algo}, which also introduces a new bound
% on the optimality gap stemming from the new formalization and the resolution scheme.
% Finally, Section \ref{sec:experiments} demonstrates that our solver is highly
% efficient compared to existing algorithms and popular implementations.



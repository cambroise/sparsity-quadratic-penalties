\subsection{OSCAR}

The sparsity inducing penalties can be adapted to pursue different goals, such
as having equal coefficients.  This was first implemented for ordered features
with the fused Lasso \citep{Tibshirani05}, which encourages sparse and locally
constant solutions by penalizing the $\ell_1$-norm of both the coefficients and
their successive differences.

Even when there is no ordering between features, equality can be desired for
interpretability purposes.  OSCAR (Octagonal Shrinkage and Clustering Algorithm
for Regression) has been conceived in this siprit, to infer clusters of
variables in a supervised setting \citep{Bondell08}.
It is based on a penalizer encouraging the sparsity of the regression
coefficients and the equality of the non-zero entries.
By this means, correlated predictors that have a similar effect on the response
form ``predictive clusters'' represented by a single coefficient.



% %
% \begin{equation*}
%   \clH^\mathrm{oscar}_{\bfbeta^*} = \left\{ 
% % \bfbeta \in \mathbb{R}^p : \norm[1]{\bfbeta} \leq \eta_\beta 
% \right\}
%   \enspace.
% \end{equation*}
% %
% The dual assumption is that the $\ell_\infty$-norm of $\bfgamma$ should be
% controlled, say:
% %
% \begin{align*}
%   \mathcal{D}^\mathrm{Oscar}_{\bfgamma} & = \left\{ 
% % \bfgamma \in \mathbb{R}^p :
% % \sup_{\bfbeta\in\clH^\mathrm{Lasso}_{\bfbeta^*}} \bfgamma^\intercal\bfbeta \leq 1 
% \right\} \\
% %     & = \left\{ \bfgamma \in \mathbb{R}^p : \norm[\infty]{\bfgamma} \leq \eta_\gamma \right\} \\
% %     & = \mathbf{conv} \big\{ \left\{ -\eta_\gamma, \eta_\gamma \right\}^p \big\}
%   \enspace,
% \end{align*}
% where $\eta_\gamma$ is defined from $\eta_\beta$ and $\mathbf{conv}$ denotes convex hull, so that Problem
% \eqref{eq:robust:general:form3} reads:
% %
% \begin{align*}
%   & \min_{\bfbeta\in\mathbb{R}^p} \max_{\bfgamma \in \left\{ -\eta_\gamma, \eta_\gamma \right\}^p}
%       \Big\{ \norm{\bfX \bfbeta - \bfy}^2 + \lambda \norm{\bfbeta - \bfgamma}^2 \Big\} \\
%   \Leftrightarrow
%     & \min_{\bfbeta\in\mathbb{R}^p}
%       \norm{\bfX \bfbeta - \bfy}^2 + 2 \lambda \eta_\gamma \norm[1]{\bfbeta} + \lambda \norm{\bfbeta}^2 
%   \enspace,
% \end{align*}


% The lagrangian formulation of  OSCAR as a constrained optimization can be expressed as
% $$
%  \min_{\bfbeta}     \norm{\bfX  \bfbeta  - \bfy  }^2+\lambda \sum_{j=1}^p \left( c(j-1)+1 \right) |\beta|_{(j)},
% $$
% with $|\beta|_{(1)}\leq |\beta|_{(2)} \leq \cdots \leq |\beta|_{(p)}$. The penalty term can be expressed in a form close to our adverse quadratic penalty. Let us consider  the adverse vector domain to be
% $$
%  \clD_{\bfgamma}=\left\{ \bfgamma \in \mathbb{R}^p | \bfgamma= 
% \begin{pmatrix}
% \alpha_1 1\\
% \alpha_2 (c+1) \\
% \alpha_3 (2c+1) \\
% \vdots \\
% \alpha_p (p-1) c+1
% \end{pmatrix}, \ c\in \mathbb{R}^+ , 
% \norm[\infty]{(\alpha_1,\cdots,\alpha_p)} \leq \eta_\alpha \right\}
% $$
% and the permutation matrix
% $$
%  P_{\bfbeta} = \left\{ \mathbb{I}_{(\mbox{rank}\left( |\beta|_{(i)}=j \right)} \right\}_{i = 1 \cdots p, j=1 \cdots p}.
% $$

% We can reformulate the previous lagrangian as 
% \begin{equation*}
%     \min_{\bfbeta\in\mathbb{R}^p} \max_{\bfgamma \in \clD_{\bfgamma}}
%     \norm{\bfX \bfbeta - \bfy } + \lambda \norm{\bfbeta +
%     P_{\bfbeta}   P_{\bfbeta}  \bfgamma } \enspace.
% \end{equation*}



The rewriting of the initial problem  allows to see that the very same
optimization adaptive  constraint algorithm  used for the  elastic net
can be used to solve the OSCAR problem (see Table \ref{table:summary}).


\begin{figure}
  \begin{center} 
    \smallxylabelsquare{../figures/oscar_decomposition1}{$\beta_1$}{$\beta_2$}{}%
    \smallxylabelsquare{../figures/oscar_decomposition2}{$\beta_1$}{$\beta_2$}{}%
    \smallxylabelsquare{../figures/oscar_decomposition3}{$\beta_1$}{$\beta_2$}{}%
    \smallxylabelsquare{../figures/oscar_decomposition4}{$\beta_1$}{$\beta_2$}{}%
    \caption{Admissible sets (patches) for the OSCAR, defined by the
      intersection of the Euclidean balls whose centers are represented by
      crosses and boundaries are
      displayed in black.}
    \label{fig:oscar-penalty}
    \end{center} 
\end{figure}


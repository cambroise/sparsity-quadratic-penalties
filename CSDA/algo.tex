\section{Algorithm}\label{sec:algo}

The unified derivation for the problems presented in Section
\ref{sec:gammaperturb} suggests a unified processing based on the iterative
resolution of quadratic problems. 
This general algorithm is summarized in this section.
We then show that the new derivation can also be used in the analysis of this 
algorithm by describing an alternative to Fenchel duality 
\citep[used for example by][]{2012_FML_Bach} to assess convergence.
% by computing an upper-bound for the gap between the current solution and the
% optimal one.

\iflong
  \subsection{Active Set Approach}
\fi

The efficient approaches developed for sparse regression take advantage of the sparsity
of the solution by solving a series of small linear systems, the sizes of which are
incrementally increased/decreased.  
Here, as for the Lasso~\citep{2000_JCGS_Osborne,2004_AS_Efron}, this process boils down to an
iterative optimization scheme involving the resolution of quadratic problems.

The algorithm is based on the iterative update of the set of ``active''
variables, $\supp$, indexing the coefficients $\bfbeta_{\!\supp}$. A
variable $j$ is part of the active set when a single value $\gamma_j$ maximizes the criterion.
In other words a variable is active if it does not generate a singular point of the criterion.
Optimization with respect to all the active variables boils then down to a simple resolution of a
quadratic problem. For the Lasso a non-zero coefficient defines an active variable.  

The active set  is started from a sparse initial guess %, say $\supp=\emptyset$ ($\bfbeta=0$),
and iterates the three following steps:
\begin{enumerate}
\item \label{item:algo:step1} the first step solves
  Problem~\eqref{eq:general:dual} considering that $\supp$, the set of
  ``active'' variables, is correct; that is, the
  objective~\eqref{eq:general:dual} is optimized with respect to
  $\bfbeta_{\!\supp}$.  This penalized least squares problem is defined
  from $\mathbf{X}_{\centerdot\supp}$, which is the submatrix of
  $\mathbf{X}$ comprising all rows and the columns indexed by $\supp\,$
  and $\bfgamma_{\!\supp}$, which is set to
  its current most adversarial value.~\footnote{%
    When several $\bfgamma_{\!\supp}$ are equally unfavorable to
    $\bfbeta_{\!\supp}$, we use gradient information to pick the worst one
    among those when $\bfbeta_{\!\supp}$ moves along the steepest descent
    direction.
  }

\item \label{item:algo:step2} the second step updates $\bfbeta_{\!\supp}$
  if necessary (and possibly $\bfgamma_{\!\supp}$), so that
  $\bfgamma_{\!\supp}$ is indeed (one of) the most adversarial value of
  the current $\bfbeta_{\!\supp}$.
  This is easily checked with the problems given in
  Section~\ref{sec:gammaperturb}, where $\uball[*]^\eta$ is a convex polytope
  whose vertices (that is, extreme $\bfgamma$-values) are associated with a
  cone of coherent $\bfbeta$-value.
%  defined by the intersection of the hyperplanes that cut the middle of
%  the neighboring edges, normally to these edges.

\item  \label{item:algo:step3} the last step updates the active set $\supp$. 
  It relies on the ``worst-case gradient'' with respect to
  $\bfbeta$, where $\bfgamma$ is chosen so as to minimize infinitesimal
  improvements of the current solution.
  Again picking the right $\bfgamma$ is easy for the problems given in
  Section~\ref{sec:gammaperturb}.
  Once this is done, we first check whether some variables should quit the
  active set, and if this is not the case, we assess the completeness of
  $\supp$, by checking the optimality conditions with respect to inactive
  variables.  We add the variable, or the group of variables that most violates
  the worst-case optimality conditions.  When no such violation exists, the
  current solution is optimal, since, at this stage, the problem is solved
  exactly within the active set $\supp$.
\end{enumerate}
Algorithm~\ref{algo:active_set} provides a more comprehensive technical
description.

%
\begin{algorithm}[htbp]
  \begin{small}
 \SetSideCommentLeft
%  \DontPrintSemicolon
  \nlset{Init.} $\bfbeta \leftarrow \bfbeta^0$\\
  Determine the active set:  $\supp \leftarrow \{j: \left\| \bfgamma - \bfbeta \right\|_2^2 = \left\| \bfgamma' - \bfbeta \right\|_2^2 \Rightarrow \bfgamma_j' = \bfgamma_j \}$\\   %$\supp \leftarrow \{j:\beta_{j} > 0 \} $\\
  Pick a worst admissible $\bfgamma$, that is,  $\displaystyle\bfgamma \in 
  \argmax_{\bfgamma' \in \uball[*]^\eta} \left\| \bfgamma' - \bfbeta \right\|_2^2$
  \BlankLine 
  \nlset{Step 1} Update active variables $\bfbeta_{\!\supp}$ assuming that $\supp$ and $\bfgamma_{\!\supp}$ are optimal \\
  $\bfbeta_{\!\supp}^\mathrm{old} \leftarrow \bfbeta_{\!\supp}$ \\
  $\bfbeta_{\!\supp} \leftarrow 
   \left(\mathbf{X}_{\centerdot\supp}^\intercal
      \mathbf{X}_{\centerdot\supp}^{\phantom{\intercal}} + \lambda \mathbf{I}_{|\supp|}\right)^{-1}
    \left(\mathbf{X}_{\centerdot\supp}^\intercal\mathbf{y} -
      \lambda \bfgamma_{\!\supp}\right)  
  $ \\ 
  \nlset{Step 2} Verify coherence of $\bfgamma_{\!\supp}$ with the updated $\bfbeta_{\!\supp}$\\
  \If(\tcp*[f]{if $\bfgamma_{\!\supp}$ is not worst-case})
   {$\displaystyle\norm{\bfbeta_{\!\supp} - \bfgamma_{\!\supp}}^2 < \max_{\bfg \in \uball[*]^\eta} \norm{\bfbeta_{\!\supp}-\bfg_{\!\supp}}^2$}
   {%
    \tcc{Backtrack towards the last $\bfgamma_{\!\supp}$-coherent solution:}
    $\bfbeta_{\!\supp} \leftarrow \bfbeta_{\!\supp}^\mathrm{old} + \rho (\bfbeta_{\!\supp} - \bfbeta_{\!\supp}^\mathrm{old})$ \\
    $\bfgamma_{\!\supp}$ is worst-case for $\bfbeta_{\!\supp}$, and there is another worst-case value $\widetilde\bfgamma_{\!\supp}$ \\
    \tcc{Check whether progress can be made with $\widetilde\bfgamma_{\!\supp}$}
    $\widetilde\bfbeta_{\!\supp} \leftarrow
    \left(\mathbf{X}_{\centerdot\supp}^\intercal
      \mathbf{X}_{\centerdot\supp}^{\phantom{\intercal}} + \lambda \mathbf{I}_{|\supp|}\right)^{-1}
    \left(\mathbf{X}_{\centerdot\supp}^\intercal\mathbf{y} -
      \lambda \widetilde\bfgamma_{\!\supp}\right)  
  $ \\ 
  \If(\tcp*[f]{if $\widetilde\bfgamma_{\!\supp}$ is worst-case\ldots})
   {$\displaystyle\norm{\widetilde\bfbeta_{\!\supp} - \widetilde\bfgamma_{\!\supp}}^2 = \max_{\bfg \in \uball[*]^\eta} \norm{\widetilde\bfbeta_{\!\supp}-\bfg_{\!\supp}}^2$}
     {%
%        $\bfbeta_{\!\supp}^\mathrm{old} \leftarrow \bfbeta_{\!\supp}$ \\
       $(\bfbeta_{\!\supp},\bfgamma_{\!\supp}) \leftarrow
          (\widetilde\bfbeta_{\!\supp},\widetilde\bfgamma_{\!\supp})$
        \tcp*[f]{$ (\widetilde\bfbeta_{\!\supp},\widetilde\bfgamma_{\!\supp})$ is better than $(\bfbeta_{\!\supp},\bfgamma_{\!\supp})$}
     }
    }
    \tcc{The current $\bfgamma_{\!\supp}$ is coherent with $\bfbeta_{\!\supp}$}

  \nlset{Step 3} Update active set $\supp$ \\
  $\displaystyle g_j \leftarrow \min_{\bfgamma\in\uball[*]^\eta} \left|
    \mathbf{x}_j^\intercal(\mathbf{X}_{\centerdot\supp}^{\phantom{\intercal}}\bfbeta_{\!\supp} - \mathbf{y})  + \lambda (\beta_j - \gamma_j) \right|
  \enspace j=1,\ldots,p$
  \tcp*[f]{worst-case gradient} \\
  \eIf{ $\exists\, j \in\supp:  \left\| \bfgamma - \bfbeta \right\|_2^2 = \left\| \bfgamma' - \bfbeta \right\|_2^2 \ and  \ \bfgamma_j' \neq \bfgamma_j  \ \text{and}\ g_j = 0$}{
       $\displaystyle \supp \leftarrow \supp\backslash\{j\}$ \tcp*[r]{Downgrade $j$} 
       \tcc{Go to Step 1}
   }{%
  \eIf{$\max_{j\in\supp^c} g_j \neq0$}{ 
   \tcc{Identify the greatest violation of optimality conditions}
    $\displaystyle j^\star \leftarrow \argmax_{j\in\supp^c} g_j \,,\enspace$
    $\supp \leftarrow \supp \cup \{j^\star\}$ \tcp*[r]{Upgrade $j^\star$} 
    \tcc{Go to Step 1}
  }{
   \tcc{Stop and return $\bfbeta$, which is optimal}
   }
  }
  \end{small} 
\caption{Worst-Case Active Set Algorithm}
\label{algo:active_set}
\end{algorithm}
%
Note that the structure is essentially identical to the one proposed by
\citet{2000_JCGS_Osborne} or \cite{2004_AS_Efron} for the Lasso, but that it 
applies to any penalty that can be decomposed as in 
Problem~\eqref{eq:general:dual}.
Our viewpoint is also radically different, as the global non-smooth problem
is dealt with subdifferentials by \citet{2000_JCGS_Osborne}, whereas we rely on
the maximum of smooth functions.
This approach suggests a new assessment of convergence, as detailed below.
%

% \iflong
  \subsection{Monitoring Convergence}

  At each iteration of the algorithm, the current $\bfbeta$ is computed assuming
  that the current active set $\supp$ and the current
  $\bfgamma_\supp$-value are optimal.
  When the current active set is not optimal, the current $\bfbeta$ (where
  $\bfbeta_\supp$ is completed by zeros on the complement $\supp^c$) is
  nevertheless optimal for a $\bfgamma$-value defined in $\Rset^p$ (where
  $\bfgamma_\supp$ is completed by ad hoc values on the complement
  $\supp^c$). However this $\bfgamma$ fails to belong to $\uball[*]^\eta$ 
  (otherwise, the problem would be solved: $\supp$, $\bfgamma$ and $\bfbeta$
  would indeed be optimal).
  The following proposition relates the current objective function, associated
  with an infeasible $\bfgamma$-value ($\bfgamma\notin\uball[*]^\eta$), to the
  global optimum of the optimization problem.

  \begin{proposition}\label{prop:monitoring}
    For any vectorial norm $\norm[*]{\cdot}$, 
%     when 
%     $\uball[*]^\eta$ is defined as 
%     $\uball[*]^\eta=\left\{\bfgamma \in \mathbb{R}^{p}:
%                         \norm[*]{\bfgamma} \leq \eta \right\}$, 
%     then, 
    $\forall \bfgamma \in \mathbb{R}^{p}:\norm[*]{\bfgamma} \geq \eta$, we have:
    \begin{equation*}
      \min_{\bfbeta\in\mathbb{R}^{p}} \max_{\bfgamma' \in \uball[*]} 
      J_\lambda(\bfbeta,\bfgamma') 
      \geq
      \frac{\eta}{\norm[*]{\bfgamma}} 
      J_\lambda\left(\bfbeta^\star\left(\bfgamma\right), \bfgamma \right) -
      \frac{\lambda\eta(\norm[*]{\bfgamma}-\eta)}{\norm[*]{\bfgamma}^2}\norm{\bfgamma}^2        
      \enspace,
    \end{equation*}
    where 
    \begin{equation*}
      J_\lambda(\bfbeta,\bfgamma) = \norm{\bfX \bfbeta - \bfy}^2 + 
        \lambda \norm{\bfbeta - \bfgamma}^2
      \enspace \text{and} \enspace
      \bfbeta^\star(\bfgamma) = \argmin_{\bfbeta\in\mathbb{R}^{p}} J_\lambda(\bfbeta,\bfgamma)
      \enspace.
    \end{equation*}
    See proof in  \ref{sec:proof:prop:monitoring}.
  \end{proposition}

  This proposition can be used to compute an optimality gap at Step 3 of
  Algorithm \ref{algo:active_set}, by picking a $\bfgamma$-value such that the
  current worst-case gradient $\bfg$ is null (the current $\bfbeta$-value then
  being the optimal $\bfbeta^\star(\bfgamma)$).
  Note that more precise upper bounds could be computed relying on significant extra computation.
  The generic optimality gap computed from Proposition \ref{prop:monitoring} differs
  from the Fenchel duality gap \citep[see][]{2012_FML_Bach}. 
  For the elastic-net expressed in \eqref{eq:elastic-net}, Fenchel inequality
  \citep[see details in][]{Mairal10} yields the following optimality gap:
   \begin{align*}
      \min_{\bfbeta\in\mathbb{R}^{p}} \max_{\bfgamma' \in \uball[*]^\eta} 
      J_\lambda(\bfbeta,\bfgamma') 
      \geq &
      J_\lambda\left(\bfbeta^\star\left(\bfgamma\right), \bfgamma \right) -
      \frac{\eta^{2}}{\norm[*]{\bfgamma}^{2}}
      \left(
        \norm{\bfX \bfbeta^\star\left(\bfgamma\right) - \bfy}^2 + 
        \lambda \norm{\bfbeta^\star\left(\bfgamma\right)}^{2}
      \right) \\
      & - \frac{2\eta}{\norm[*]{\bfgamma}}
      \left(
         \bfX \bfbeta^\star\left(\bfgamma\right) - \bfy
     \right)^\intercal \bfy
     \enspace.
   \end{align*}
  %
  \begin{figure}
    \centering 
    \xylabellarge{../figures/monitoring_bounds}{\# of iterations}{Optimality gap (log scale)}{$n=50$, $p=200$}
    \caption{Monitoring convergence: true optimality gap (solid black) versus
    our generic upper bound (dashed blue) and Fenchel's duality gap for
    elastic-net (dotted red) computed at each iteration of
    Algorithm~\ref{algo:active_set}.}
    \label{fig:monitoring}
  \end{figure}
  %
  The two optimality gaps are empirically compared in 
  Figure~\ref{fig:monitoring} for the elastic-net, along a short regularization 
  path with five values of the $\ell_{1}$-penalization parameter.  
  We see that the two options can be used to assess convergence, though
  Fenchel's duality gap is more accurate for the rougher solutions. 
  Note however that both upper bounds are fairly coarse until a very accurate
  solution is reached, which makes them both unsuitable for deriving loose stopping 
  criteria.
%   Fenchel's duality gap is more accurate here, especially for the rougher
%   solutions.  However, the practical use of these upper bounds is rather limited
%   as they are both fairly coarse unless a very accurate solution is reached.
%   These optimality gaps may thus be used to assess convergence, but they are
%   ineffective as stopping rules when fast and rough solutions are sought.
  Proposition~\ref{prop:monitoring} is thus of limited scope, but it illustrates
  that, besides its algorithmic consequences, our original view of sparse
  penalties opens new ways for analysis.
  As a final note on this topic, we provide a slightly tighter inequality for
  computing the optimality gap in~\ref{sec:proof:prop:monitoring}, and
  we conjecture that it could be further tightened (see in particular the
  derivation of inequality \eqref{eq:crude_inequality}).

  % \iffalse
  %   What is a good value for $\bfgamma$?
  %   Computationally, it is a value for which we already have computed 
  %   $\min_{\bfbeta\in\mathbb{R}^{p}} J_\lambda(\bfbeta,\bfgamma)$, regarding the 
  %   tightness of the bound, ${\norm[*]{\bfgamma}}$ should be as close to 
  %   $\eta_\gamma$ as possible, while $\min_{\bfbeta\in\mathbb{R}^{p}} 
  %   J_\lambda(\bfbeta,\bfgamma)$ should be as large as possible.
    
  %   Hence, a tight optimality gap can be computed provided one can guess a
  %   near-optimal $\bfgamma$-value.
  %   When entering Step \ref{item:algo:step3} of Algorithm~\ref{algo:active_set}, the current best guess
  %   $\widehat\bfgamma_{\supp}$ has to be completed on the complement $\supp^c$.
  %   We propose to do so by minimizing the magnitude of the gradient with respect to $\bfbeta$ at the current solution:
  %   \begin{align*}
  %     \widehat\bfgamma_{\supp^c} & = \argmin_{\bfgamma_{\supp^c}:\bfgamma \in \clD_{\bfgamma}}
  %                                \left|  \bfX_{\centerdot\supp^c}^\intercal(\bfX_{\centerdot\supp}\bfbeta_{\supp}-\bfy) - 
  %                                \lambda \bfgamma_{\supp^c} \right|
  %       \enspace.
  %   \end{align*}
  %   Finally, the full vector $\widehat\bfgamma$ can be used to compute the lower
  %   bound on the optimal objective function $J(\widehat{\bfbeta}
  %   (\widehat\bfgamma),\widehat\bfgamma)$. 
  %   This lower bound, which requires to solve a linear system of size $p$, is
  %   lighter to compute than the duality gap \citep[see details
  %   in][]{2012_FML_Bach}, and was observed to provide a tight bound. As a sanity
  %   check, note that when $\bfbeta$ is optimal, the corresponding optimal
  %   $\widehat\bfgamma$ is known to be null on $\supp^c$, so that the bound is exact.
  %   Two typical behaviors are displayed in Figure~\ref{fig:monitoring} for the
  %   elastic-net.
  %   The two plots show the true distance to the optimum, measured by the
  %   difference in objective function, and the value of the bound, both computed at each
  %   iteration of Algorithm~\ref{algo:active_set} at the entrance of Step
  %   \ref{item:algo:step3}.
  %   The left-hand side plot shows the unfavorable case where $p$ is larger than
  %   $n$, where the bound is not as precise as in the opposite setup, displayed in
  %   the right-hand side plot, where it is very close to the true optimality
  %   gap.
  
  %   \begin{figure}
  %     \centering 
  %     \begin{tabular}{c@{\hspace*{3em}}c}
  %       \xylabelsquare{../figures/monitoring_small_n}{\# of iterations}
  %                     {Optimality gap}{$n=50$, $p=200$}% 
  %       \xylabelsquare{../figures/monitoring_large_n}{\# of iterations}
  %                     {Optimality gap}{$n=400$, $p=200$}% 
  %     \end{tabular}
  %     \caption{Monitoring convergence: true optimality gap (solid blue) versus our
  %       upper bound (dashed red) computed at each iteration of
  %       Algorithm~\ref{algo:active_set}.}
  %     \label{fig:monitoring}
  %   \end{figure}
  %   % $\rho=0.9$, $\lambda_1=0.5$, $\lambda_2=0.�1$, $R^2\simeq0.8$.
  %   \begin{align*}
  %     \bfbeta_\supp & = \argmin_{\bfbeta\in\mathbb{R}^{|\supp|}} 
  %                     \norm{\bfX_{\centerdot\supp} \bfbeta - \bfy}^2 + \lambda_1 \norm[1]{\bfbeta}
  %     \\
  %     \widehat\bfbeta_{\supp^c} & = \bfzero \\ 
  %     \widehat\bfgamma_{\supp^c} & = \argmin_{\bfgamma_{\supp^c} \in [-1,1]^{|\supp^c|}}
  %                              \left|  \bfX_{\centerdot\supp^c}^\intercal(\bfX\widehat\bfbeta-\bfy) + 
  %                                \lambda_1 \bfgamma_{\supp^c} \right|
  %       \enspace. 
  %   \end{align*}
  %   We can then compute 
  %   \begin{align*}
  %     \widetilde{\bfbeta} & = \argmin_{\bfbeta\in\mathbb{R}^{p}} 
  %                     \norm{\bfX \bfbeta - \bfy}^2 + \lambda_1 \bfbeta^\intercal\widehat\bfgamma
  %       \enspace. 
  %   \end{align*}
  %   We then have:
  %   \begin{align*}
  %   J(\bfbeta^\star) & = \min_{\bfbeta\in\mathbb{R}^p} 
  %     \norm{\bfX \bfbeta - \bfy}^2 + \lambda_1  \norm[1]{\bfbeta} \\
  %     &= \min_{\bfbeta\in\mathbb{R}^p} \max_{\bfgamma \in \clD_{\bfgamma}}
  %         \Big\{ \norm{\bfX \bfbeta - \bfy}^2 + \lambda_1 \bfbeta^\intercal\bfgamma \Big\} \\
  %     &= \max_{\bfgamma \in \clD_{\bfgamma}} \min_{\bfbeta\in\mathbb{R}^p}
  %         \Big\{ \norm{\bfX \bfbeta - \bfy}^2 + \lambda_1 \bfbeta^\intercal\bfgamma \Big\} \\
  %     & \geq \min_{\bfbeta\in\mathbb{R}^p}
  %         \Big\{ \norm{\bfX \bfbeta - \bfy}^2 + \lambda_1 \bfbeta^\intercal\widehat\bfgamma \Big\}  \\
  %     & = \norm{\bfX \widetilde{\bfbeta} - \bfy}^2 + \lambda_1 \widetilde{\bfbeta}^{(t)^\intercal}\widehat\bfgamma
  %     \enspace,
  %   \end{align*}
  %   which provides a lower bound for $J(\bfbeta^\star)$.
  %   Meanwhile, we obviously have:
  %   \begin{align*}
  %   J(\bfbeta^\star) & = \min_{\bfbeta\in\mathbb{R}^p} 
  %     \norm{\bfX \bfbeta - \bfy}^2 + \lambda_1 \norm[1]{\bfbeta} \\
  %     &\leq \norm{\bfX \widehat\bfbeta - \bfy}^2 + \lambda_1  \norm[1]{\widehat\bfbeta} 
  %     \enspace,
  %   \end{align*}
  %   So that $J(\bfbeta^\star)$ is easily bracketed during the optimization process. 
  %   Note that when $\bfbeta_\supp=\bfbeta^\star$, we have that
  %   $\widetilde{\bfbeta}=\bfbeta_\supp$.

  %   The additional computational cost for monitoring (that is, the cost of evaluating
  %   the upper and lower bounds for an intermediate solution can be important because
  %   $\widetilde{\bfbeta}$ requires to solve a linear system of size $p$\ldots 
  %   We could do better than that if we can prove that $\big(\widetilde{\bfbeta}\big)_j=0$ if
  %   the subgradient at  $\widehat\bfbeta_{\supp^c}$ comprises zero. In this situation, we 
  %   should only consider adding the variables for which the optimality conditions
  %   are violated.

  %   \subsection{Monitoring Convergence: Lasso-Case}

  %   \begin{align*}
  %     \bfbeta_\supp & = \argmin_{\bfbeta\in\mathbb{R}^{|\supp|}} 
  %                     \norm{\bfX_{\centerdot\supp} \bfbeta - \bfy}^2 + \lambda_1 \norm[1]{\bfbeta}
  %     \\
  %     \widehat\bfbeta_{\supp^c} & = \bfzero \\ 
  %     \widehat\bfgamma_{\supp^c} & = \argmin_{\bfgamma_{\supp^c} \in [-1,1]^{|\supp^c|}}
  %                              \left|  \bfX_{\centerdot\supp^c}^\intercal(\bfX\widehat\bfbeta-\bfy) + 
  %                                \lambda_1 \bfgamma_{\supp^c} \right|
  %       \enspace. 
  %   \end{align*}
  %   We can then compute 
  %   \begin{align*}
  %     \widetilde{\bfbeta} & = \argmin_{\bfbeta\in\mathbb{R}^{p}} 
  %                     \norm{\bfX \bfbeta - \bfy}^2 + \lambda_1 \bfbeta^\intercal\widehat\bfgamma
  %       \enspace. 
  %   \end{align*}
  %   We then have:
  %   \begin{align*}
  %   J(\bfbeta^\star) & = \min_{\bfbeta\in\mathbb{R}^p} 
  %     \norm{\bfX \bfbeta - \bfy}^2 + \lambda_1  \norm[1]{\bfbeta} \\
  %     &= \min_{\bfbeta\in\mathbb{R}^p} \max_{\bfgamma \in \clD_{\bfgamma}}
  %         \Big\{ \norm{\bfX \bfbeta - \bfy}^2 + \lambda_1 \bfbeta^\intercal\bfgamma \Big\} \\
  %     &= \max_{\bfgamma \in \clD_{\bfgamma}} \min_{\bfbeta\in\mathbb{R}^p}
  %         \Big\{ \norm{\bfX \bfbeta - \bfy}^2 + \lambda_1 \bfbeta^\intercal\bfgamma \Big\} \\
  %     & \geq \min_{\bfbeta\in\mathbb{R}^p}
  %         \Big\{ \norm{\bfX \bfbeta - \bfy}^2 + \lambda_1 \bfbeta^\intercal\widehat\bfgamma \Big\}  \\
  %     & = \norm{\bfX \widetilde{\bfbeta} - \bfy}^2 + \lambda_1 \widetilde{\bfbeta}^{(t)^\intercal}\widehat\bfgamma
  %     \enspace,
  %   \end{align*}
  %   which provides a lower bound for $J(\bfbeta^\star)$.
  %   Meanwhile, we obviously have:
  %   \begin{align*}
  %   J(\bfbeta^\star) & = \min_{\bfbeta\in\mathbb{R}^p} 
  %     \norm{\bfX \bfbeta - \bfy}^2 + \lambda_1 \norm[1]{\bfbeta} \\
  %     &\leq \norm{\bfX \widehat\bfbeta - \bfy}^2 + \lambda_1  \norm[1]{\widehat\bfbeta} 
  %     \enspace,
  %   \end{align*}
  %   So that $J(\bfbeta^\star)$ is easily bracketed during the optimization process. 
  %   Note that when $\bfbeta_\supp=\bfbeta^\star$, we have that
  %   $\widetilde{\bfbeta}=\bfbeta_\supp$.
  %   \fi
% \else
%   a simple lower bound on the objective function. 

%   Let $J$ be the objective function of the generic
%   Problem~\eqref{eq:general:dual}, we trivially have that, for any
%   $\bfgamma$-value,
%   $\min_{\bfbeta\in\mathbb{R}^{p}} \max_{\bfgamma \in \clD_{\bfgamma}}
%   J(\bfbeta,\bfgamma) \geq \min_{\bfbeta\in\mathbb{R}^{p}} J(\bfbeta,\bfgamma)$.
%   %
%   Hence, a tight optimality gap can be computed provided one can guess a
%   near-optimal $\bfgamma$-value.
%   When entering Step \ref{item:algo:step3} of Algorithm~\ref{algo:active_set}, the current best guess
%   $\widehat\bfgamma_{\supp}$ has to be completed on the complement $\supp^c$.
%   We propose to do so by minimizing the magnitude of the gradient with respect to $\bfbeta$ at the current solution:
%   \begin{align*}
%     \widehat\bfgamma_{\supp^c} & = \argmin_{\bfgamma_{\supp^c}:\bfgamma \in \clD_{\bfgamma}}
%                                \left|  \bfX_{\centerdot\supp^c}^\intercal(\bfX_{\centerdot\supp}\bfbeta_{\supp}-\bfy) - 
%                                \lambda \bfgamma_{\supp^c} \right|
%       \enspace.
%   \end{align*}
%   The resulting vector $\widehat\bfgamma$ is then used to compute the lower
%   bound on the optimal objective function as $\min_{\bfbeta\in\mathbb{R}^{p}} J(\bfbeta,\widehat\bfgamma)$. 
%   This lower bound, which requires to solve a linear system of size $p$, is
%   much lighter to compute than the duality gap \citep[see details
%   in][]{2012_FML_Bach}, and was observed to provide a tight bound (as a sanity
%   check, note that when $\bfbeta$ is optimal, our estimated optimality gap is null).
% \fi

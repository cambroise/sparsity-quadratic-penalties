\documentclass[review,authoryear]{elsarticle}

%% Use the option review to obtain double line spacing
%% \documentclass[authoryear,preprint,review,12pt]{elsarticle}

%% Use the options 1p, twocolumn; 3p; 3p,twocolumn; 5p; or 5p,twocolumn
%% for a journal layout:
%% \documentclass[final,1p,times,authoryear]{elsarticle}
%% \documentclass[final,1p,times,twocolumn,authoryear]{elsarticle}
%% \documentclass[final,3p,times,authoryear]{elsarticle}
%% \documentclass[final,3p,times,twocolumn,authoryear]{elsarticle}
%% \documentclass[final,5p,times,authoryear]{elsarticle}
%% \documentclass[final,5p,times,twocolumn,authoryear]{elsarticle}

%% The amssymb package provides various useful mathematical symbols
\usepackage{amssymb,amsthm,amsmath}
\usepackage{array}
\usepackage{natbib}
\usepackage{hyperref}
\usepackage{xcolor}
\usepackage[ruled,vlined]{algorithm2e}

% Definitions of handy macros can go here
%\newtheorem{definition}{Definition}
%\newtheorem{theorem}{Theorem}
% \newtheorem{proposition}{Proposition}
% \newtheorem{assumption}{Assumption}
%\newtheorem{lemma}{Lemma}
%\newtheorem{remark}{Remark}

\newcommand\mediumxylabelsquare[4]{% #1 Image, #2=XText, #3=YText,, #4=Title,
  \setlength{\unitlength}{4cm}%
  \begin{picture}(1,1)%
    \put(0.10,0.10){\includegraphics[width=0.85\unitlength]{#1}}
    \put(0.5,0){\makebox[0cm]{\small#2}}
    \put(0,0.5){\rotatebox{90.0}{\makebox[0cm]{\small#3}}}
    \put(0.525,0.95){\makebox[0cm]{#4}}
  \end{picture}%
}

\newcommand{\argmax}{\mathop{\mathrm{arg\,max}}}
\newcommand{\argmin}{\mathop{\mathrm{arg\,min}}}
\newcommand{\minimize}{\mathop{\mathrm{minimize}}}
\newcommand{\set}[1]{\left\{#1\right\}}
\newcommand{\Rset}{\mathbb{R}}
% definitions related to groups
\newcommand{\norm}[2][2]{\left\|#2\right\|_{#1}}
\newcommand{\uball}[1][]{{\mathcal{B}}_{#1}}

\newcommand{\group}[1][k]{{\mathcal G}_{#1}}
\newcommand{\positive}{{\mathcal P}}
\newcommand{\negative}{{\mathcal N}}
\newcommand{\zero}{{\mathcal Z}}
%\renewcommand{\active}[1][k]{{\mathcal A}_{#1}}
\newcommand{\1}{\mathds{1}}

\newcommand{\bfzero}{\mathbf{0}}

\newcommand{\bfe}{\mathbf{e}}
\newcommand{\bfg}{\mathbf{g}}
\newcommand{\bfv}{\mathbf{v}}
\newcommand{\bfx}{\mathbf{x}}
\newcommand{\bfy}{\mathbf{y}}
\newcommand{\bfz}{\mathbf{z}}
\newcommand{\bfX}{\mathbf{X}}

\newcommand{\bfD}{\mathbf{D}}

\newcommand{\bfU}{\mathbf{U}}
\newcommand{\bfZ}{\boldsymbol{\Delta}_\bfX}%\mathbf{Z}}

\newcommand{\bfalpha}{\boldsymbol\alpha}
\newcommand{\bfbeta}{\boldsymbol\beta}
\newcommand{\bfkappa}{\boldsymbol\kappa}
\newcommand{\bfvarphi}{\boldsymbol\varphi}
\newcommand{\bftheta}{\boldsymbol\theta}
\newcommand{\bfgamma}{{\boldsymbol\gamma}}
\newcommand{\bfepsilon}{\boldsymbol\epsilon}
\newcommand{\bflambda}{\boldsymbol\lambda}

\newcommand{\bfvarepsilon}{\boldsymbol\varepsilon}

\newcommand{\bfTheta}{\boldsymbol\Theta}
\newcommand{\bfXi}{\boldsymbol\Xi}

\newcommand{\clA}{\mathcal{A}}
\newcommand{\clD}{\mathcal{D}}
\newcommand{\clH}{\mathcal{H}}
\newcommand{\clO}{\mathcal{O}}
\newcommand{\clS}{\mathcal{S}}

\newcommand{\ba}{\mathbf{a}}
\newcommand{\bb}{\mathbf{b}}



\newcommand{\sign}{\mathrm{sign}}
\newcommand{\weights}{\mathbf{w}}
\newcommand{\supp}{\mathcal{A}}
\newcommand{\prob}{\mathbb{P}}

%definitions related to convergences
\newcommand{\inprob}{\overset{P}{\longrightarrow}}
\newcommand{\inlaw}{\overset{D}{\longrightarrow}}
 
\newcommand{\lambdaclass}{\lambda^{\textrm{err}}}

\newcommand{\hatbeta}{\hat{\beta}}
\newcommand{\hatbbeta}{\,\hat{\!\bfbeta}}
\newcommand{\hatbetaenet}{\,\hat{\!\beta}^{\mathrm{enet}}}
\newcommand{\hatbetalasso}{\,\hat{\!\beta}^{\mathrm{lasso}}}
\newcommand{\hatbetacoop}{\,\hat{\!\beta}^{\mathrm{coop}}}
\newcommand{\hatbetagroup}{\,\hat{\!\beta}^{\mathrm{group}}}
\newcommand{\hatbetasgroup}{\,\hat{\!\beta}^{\mathrm{sgl}}}
\newcommand{\hatbetaridge}{\,\hat{\!\beta}^{\mathrm{ridge}}}
\newcommand{\hatbetaols}{\,\hat{\!\beta}^{\mathrm{ols}}}
\newcommand{\hatbbetaenet}{\,\hat{\!\bfbeta}^{\mathrm{enet}}}
\newcommand{\hatbbetalasso}{\,\hat{\!\bfbeta}^{\mathrm{lasso}}}
\newcommand{\hatbbetacoop}{\,\hat{\!\bfbeta}^{\mathrm{coop}}}
\newcommand{\hatbbetagroup}{\,\hat{\!\bfbeta}^{\mathrm{group}}}
\newcommand{\hatbbetasgroup}{\,\hat{\!\bfbeta}^{\mathrm{sgl}}}
\newcommand{\hatbbetaridge}{\,\hat{\!\bfbeta}^{\mathrm{ridge}}}
\newcommand{\hatbbetaols}{\,\hat{\!\bfbeta}^{\mathrm{ols}}}
\newcommand{\tildebbeta}{\,\tilde{\!\bfbeta}}
\newcommand{\bfbetaenet}{\bfbeta^{\mathrm{enet}}}
\newcommand{\bfbetaridge}{\bfbeta^{\mathrm{ridge}}}
\newcommand{\bfbetacoop}{\bfbeta^{\mathrm{coop}}}
\newcommand{\bfbetagroup}{\bfbeta^{\mathrm{group}}}
\newcommand{\bfbetaols}{\bfbeta^{\mathrm{ols}}}

\newcommand\mytexttt[1]{\texttt{\small #1}}
\newtheorem{assumption}{Assumption}
\newtheorem{lemma}{Lemma}
\newtheorem{proposition}{Proposition}
\newcommand\xylabellarge[4]{% #1 Image, #2=XText, #3=YText,, #4=Title,
  \setlength{\unitlength}{13cm}%
  \begin{picture}(1,0.6)%
    \put(0.04,0.025){\includegraphics[width=0.95\unitlength]{#1}}
    \put(0.54,0){\makebox[0cm]{\small#2}}
    \put(0,0.3){\rotatebox{90.0}{\makebox[0cm]{\small#3}}}
    \put(0.525,0.6){\makebox[0cm]{#4}}
  \end{picture}%
}
\newcommand\xylabelsquare[4]{% #1 Image, #2=XText, #3=YText,, #4=Title,
  \setlength{\unitlength}{6.5cm}%
  \begin{picture}(1,1)%
    \put(0.075,0.075){\includegraphics[width=0.90\unitlength]{#1}}
    \put(0.575,0){\makebox[0cm]{\small#2}}
    \put(0,0.525){\rotatebox{90.0}{\makebox[0cm]{\small#3}}}
    \put(0.525,1){\makebox[0cm]{#4}}
  \end{picture}%
}
\newcommand\medxylabelsquare[4]{% #1 Image, #2=XText, #3=YText,, #4=Title,
  \setlength{\unitlength}{0.33\linewidth}%
  \begin{picture}(1,1)%
    \put(0.075,0.075){\includegraphics[width=0.90\unitlength]{#1}}
    \put(0.525,0){\makebox[0cm]{\small#2}}
    \put(0,0.525){\rotatebox{90.0}{\makebox[0cm]{\small#3}}}
    \put(0.525,1){\makebox[0cm]{#4}}
  \end{picture}%
}
\newcommand\smallxylabelsquare[4]{% #1 Image, #2=XText, #3=YText,, #4=Title,
  \setlength{\unitlength}{0.25\linewidth}%
  \begin{picture}(1,1)%
    \put(0.10,0.10){\includegraphics[width=0.80\unitlength]{#1}}
    \put(0.5,0){\makebox[0cm]{\small#2}}
    \put(0,0.5){\rotatebox{90.0}{\makebox[0cm]{\small#3}}}
    \put(0.525,0.95){\makebox[0cm]{#4}}
  \end{picture}%
}

\newif\ifverylong\verylongtrue
\newif\iflong\longtrue

%% The lineno packages adds line numbers. Start line numbering with
%% \begin{linenumbers}, end it with \end{linenumbers}. Or switch it on
%% for the whole article with \linenumbers.
\usepackage{lineno}
\linenumbers

\journal{CSDA}

\begin{document}

\begin{frontmatter}

%% Title, authors and addresses

%% use the tnoteref command within \title for footnotes;
%% use the tnotetext command for theassociated footnote;
%% use the fnref command within \author or \address for footnotes;
%% use the fntext command for theassociated footnote;
%% use the corref command within \author for corresponding author footnotes;
%% use the cortext command for theassociated footnote;
%% use the ead command for the email address,
%% and the form \ead[url] for the home page:
%% \title{Title\tnoteref{label1}}
%% \tnotetext[label1]{}
%% \author{Name\corref{cor1}\fnref{label2}}
%% \ead{email address}
%% \ead[url]{home page}
%% \fntext[label2]{}
%% \cortext[cor1]{}
%% \address{Address\fnref{label3}}
%% \fntext[label3]{}


%% use optional labels to link authors explicitly to addresses:
%% \author[label1,label2]{}
%% \address[label1]{}
%% \address[label2]{}

\title{Sparsity by Worst-Case Penalties}

\author[label1]{Yves Grandvalet}
\ead{yves.grandvalet@utc.fr}
\author[label2]{Julien Chiquet}
\ead{julien.chiquet@inra.fr}
\author[label3]{Christophe Ambroise}
\ead{christophe.ambroise@genopole.cnrs.fr}
\address[label1]{Sorbonne universit\'es, Universit\'e de technologie de Compi\`egne, CNRS, \\ Heudiasyc UMR 7253, CS 60 319, 60 203 Compi\`egne cedex, France}
\address[label2]{AgroParisTech, INRA, Universit\'e Paris-Saclay\\UMR MIA-Paris, 16 rue Claude Bernard, 75005, Paris, France}
\address[label3]{Universit\'e d'\'Evry Val d'Essonne, Universit\'e Paris-Saclay,  ENSIIE, USC INRA\\ UMR CNRS 8071, LaMME, 91000 \'Evry, France}

\begin{abstract}% 
  This paper proposes a new interpretation of sparse penalties such as
  the elastic-net and the group-lasso.  Beyond providing a new
  viewpoint on these penalization schemes, our approach results in a
  unified optimization strategy.  Our experiments
  demonstrate that this strategy, implemented on the elastic-net, is
  computationally extremely efficient for small to medium size
  problems.  Our accompanying software solves problems very accurately, at machine
  precision, in the time required to get a rough estimate with
  competing state-of-the-art algorithms. 
  We illustrate on real and artificial datasets that this accuracy is required to 
  for the correctness of the support of the solution, which is an important element 
  for the interpretability of sparsity-inducing penalties. 
\end{abstract} 

\begin{keyword}
  sparsity, adaptive penalty, dual norm, optimality gap
\end{keyword}

\end{frontmatter}

\section{Introduction}

% # Quadrupen

% ## Introduction

% - Interprétation géométrique et aspect algorithmique, pourquoi c’est bien:
%     - c’est générique
%     - c’est  précis
%     - dans quel cadre c’est bien
%         - problèmes de taille intermédiaire

% ## Arguments

% - résolution exacte
%     - intérêt pour l’interprétabilité
%     - voir la conclusion
% - point de vue unifiant
% - dualité
%     - qui permet le calcul de la borne
%     - on remplace le sous gradient par le worst case
% - créativité (attention surtout valable en 2d
% - stabilité
% - linéaire par morceau
% - ...
% - code
% - expériences
% - borne
%     - mais moins bonne que Fenchel
% - plus de sous gradient

% ## Contre-Arguments

% - rien de vraiment neuf
% - coût quadratique

% ## Conclusion

% - c’est un package qui devrait être utilisé pour interpréter les variables sélectionnées
%     - parcque c’est plus précis
%     - parce qu’en conséquences on gagne en précision sur le support, mais pas en terme de MSE

% ## Consigne CSDA

% - The journal requires that the manuscript contain either computational or data analysis component.  Papers which are purely theoretical are not appropriate.  Manuscripts describing simulation studies must
%     - 1. be thorough with regard to the choice of parameter settings,
%     - 2. not over-generalize the conclusions,
%     - 3. carefully describe the limitations of the simulation studies, and
%     - 4. should guide the user regarding when the recommended methods areappropriated.It is recommended that the author indicate why comparisons cannot be made theoretically.

% ## TODO

% - Reprendre l’intro
%     - une vue unifiante
%     - fondée sur une dualité alternative
%     - qui permet
%         - une vision géométrique
%             - créativité
%         - une analyse sans sous gradient
%         - solution exacte
%             - précision
%             - stabilité
%             - le calcul d’une borne
%     - analyse de la vitesse de convergence difficile
%         - donc nous faisons des simulations

Inferential statistics aim at drawing conclusions from data and from some kind
of assumption or prior information about the underlying distribution.  It is
well known that processes where data guide the choice of assumptions can lead to
paradoxes, resulting from overfitting issues, in particular in the large
dimensional setup when data are used to select explicative variables: the
seeming explanatory power of weakly relevant variables can be important when the
number of variables is similar to the number of data
points~\citep{Freedman83b,Ambroise02}.
In this context, we present here an unusual reformulation of variable selection
methods based on sparsity-inducing penalties: we show that these methods can be
interpreted as adaptive penalties, where adaptivity refers to the choice of the
eventual penalty from data.
However, contrary to the available reformulations so far, ours shows that the adaptation
of the sparsity-inducing penalty to data follows a disagreement principle, where
the least-favorable penalty is selected from data.

We believe that our new viewpoint can be instrumental in unified analyses of
algorithms and their derived estimators, and we show here two such examples:
first, we provide a generic algorithm for solving the data fitting problem;
then, we derive the general form of an optimality gap that may be used to
monitor convergence.
Our experimental section illustrates the effectiveness of the generic algorithm
motivated by our interpretation, when instanced on the elastic-net estimator:
the algorithm, which relies on solving linear systems
is accurate, and computationally efficient up to medium scale problems (thousands of
variables).
As a side experimental result, we show that solving problems with high
precision, as with the proposed approach, benefits to the performances, either
measured in terms of prediction accuracy or in terms of support error rate.


% % Coordinate descent as implemented in \mytexttt{glmnet} \citep{2009_JSS_Friedman}
% % is a versatile approach allowing to solve various sparse regression
% % problems in a simple and fast procedure. In a nutshell, it
% % consists in cycling through the parameters and updating each in turn
% % until convergence. Proximal methods are popular in the machine learning community since
% % they  are optimal  among the  first-order  techniques.   These methods
% % minimize  objective  functions with  a  nonsmooth  part by  successive
% % quadratic approximations. \mytexttt{SPAMS} (SPArse Modeling
% % Software) \citep{2012_FML_Bach} represents an effective  implementation  of  this
% % approach for sparse regression problems.  


% Section \ref{sec:robust} introduces our general robust regression formalization,
% which allows numerous variants that follow from the definition of the
% uncertainty set on the adversarial noise, thereby leading to different sparse
% regression problems.  
% Section \ref{sec:quadra} fully details the derivations for the  and the
% group-Lasso (using the $\ell_{\infty,1}$ mixed-norm) problems, applied together
% with an $\ell_2$ ridge penalty (leading to what is known as the elastic net for
% the Lasso).
% %
% A description of the general-purpose active set algorithm derived from this
% formalism is given in Section \ref{sec:algo}, which also introduces a new bound
% on the optimality gap stemming from the new formalization and the resolution scheme.
% Finally, Section \ref{sec:experiments} demonstrates that our solver is highly
% efficient compared to existing algorithms and popular implementations.




\section{Adaptive Penalties \label{sec:adaptquadra}}

\subsection{Background}

We consider the linear regression model 
\begin{equation}
  \label{eq:linear_reg_group}
  Y = X \bfbeta^\star + \varepsilon
  \enspace,
\end{equation}
where $Y$ is a continuous response variable, $X=(X_1,\dots,X_p)$ is a vector of
$p$ predictor variables, $\bfbeta^\star$ is the vector of unknown parameters and
$\varepsilon$ is a zero-mean Gaussian error variable with variance $\sigma^2$.
We will assume throughout this paper that $\bfbeta^\star$ has few non-zero
coefficients.

The estimation and inference of $\bfbeta^\star$ is based on training data,
consisting of a vector
$\mathbf{y}=(y_1,\dots,y_n)^\intercal$ for responses and a
$n\times  p$ design  matrix $\mathbf{X}$  whose $j$th  column contains
$\mathbf{x}_j  = (x_j^1,\dots,x_j^n)^\intercal$, the  $n$ observations
for variable $X_j$.  For  clarity, we assume that both $\mathbf{y}$
and $\{\mathbf{x}_j\}_{j=1,\dots,p}$ are centered so as to eliminate the
intercept from fitting criteria.

Penalization methods that build on the $\ell_1$-norm, referred to as
\emph{Lasso} procedures (Least Absolute Shrinkage and Selection Operator), are
now widely used to tackle simultaneously variable estimation and selection in
sparse problems.  They define a shrinkage estimator of the form
\begin{equation}\label{eq:general:original}
  \hat{\bfbeta} = \argmin_{\bfbeta\in\mathbb{R}^p} 
    \frac{1}{2} \norm{\bfX \bfbeta - \bfy}^2 + 
    \lambda \norm[]{\bfbeta}
  \enspace, 
\end{equation}
where $\norm[2]{\cdot}$ is the Euclidean norm and $\norm[]{\cdot}$ is an
arbitrary norm, chosen to induce some assumed sparsity pattern (typically
$\ell_1$ or $\ell_{c,1}$ norms, where $c \in (1,\infty]$).
%
% The tuning parameter $\lambda\geq  0$ controls the overall amount  of penalty.

The existence of computationally efficient optimization procedures plays an
important role in the popularity of these methods.
Though various general-purpose convex optimization solvers could be used
\citep{boyd2004convex}, exploiting the structure of the regularization problem,
and especially the sparsity of solutions, is essential in terms of computational
efficiency.
%
\citet{2012_FML_Bach} provided an overview of the families of techniques 
specifically  designed for solving this type of problems:
proximal methods, coordinate descent algorithms, reweighted-$\ell_{2}$
algorithms, working-set methods.
Stochastic gradient methods \citep{moulines2011non}, the
Frank-Wolfe algorithm \citep{lacoste2012block} or ADMM \citep[Alternating Direction
Method of Multipliers,][]{boyd2011distributed} have also recently gained in
popularity to the resolution of sparse problems.

We present below a new formulation of Problem~\eqref{eq:general:original} that
motivates an algorithm that may seem reminiscent of reweighted-$\ell_{2}$
algorithms, but which is in fact more closely related to working-set methods.
%
As for reweighted-$\ell_{2}$ algorithms, our proposal is based on the
reformulation of the sparsity-inducing penalty in terms of penalties that are
simpler to handle (linear or quadratic).  However, whereas
reweighted-$\ell_{2}$ algorithms rely on a
variational formulation of the sparsity-inducing norm that ends up in an
augmented minimization problem, our proposal is rooted in the duality principle,
eventually leading to a minimax problem that lends itself to a working-set 
algorithm that will be presented in Section~\ref{sec:algo}.

\subsection{Dual Norms}

When the sparsity-inducing penalty is a norm, its sublevel sets can always be
defined as the intersection of linear or quadratic sublevel sets.  In other
terms, if the optimization problem is written in the form of a constrained
optimization problem with inequality constraints pertaining to the penalty,
then, the feasible region can be defined as the intersection of linear or
quadratic regions. 
This fact, which is illustrated in Figures~\ref{fig:en-penalty} and
\ref{fig:group-penalty}, stems from the definition of dual norms:
%
\begin{equation*}%\label{eq:lasso_generic}
  \norm[]{\bfbeta} = \max_{\bfgamma\in\uball[*]} \bfgamma^\intercal \bfbeta
  \enspace,
\end{equation*}
where $\uball[*]$ is the unit ball centered at the origin defined from the dual
norm $\norm[*]{\cdot}$,
$\uball[*]=\left\{\bfgamma\in\mathbb{R}^p:\norm[*]{\bfgamma}\leq 1\right\}$.
Using this definition, Problem~\eqref{eq:general:original} can be reformulated
as
%
\begin{equation}\label{eq:general:primal}
  \hat{\bfbeta} = \argmin_{\bfbeta\in\mathbb{R}^p} 
  \max_{\bfgamma\in\uball[*]}
    \frac{1}{2} \norm{\bfX \bfbeta - \bfy}^2 + 
    \lambda \bfgamma^\intercal \bfbeta
  \enspace. 
\end{equation}
%
Technically, this formulation is the primal form of the original 
Problem~\eqref{eq:general:original} using the coupling function defined by the 
dual norm \citep[see e.g.][]{Gilbert16, Bonnans06}. 
It is interesting in the sense that the problem
%
\begin{equation*}
  \min_{\bfbeta\in\mathbb{R}^p} 
  \frac{1}{2} \norm{\bfX \bfbeta - \bfy}^2 + 
  \lambda \bfgamma^\intercal \bfbeta
\end{equation*}
%
is simple to solve for any value of $\bfgamma$, since it only requires solving 
a linear system.
The problem
%
\begin{eqnarray}
  \hat{\bfgamma} & = & \argmax_{\bfgamma\in\uball[*]}
    \frac{1}{2} \norm{\bfX \bfbeta - \bfy}^2 + 
    \lambda \bfgamma^\intercal \bfbeta \nonumber \\
     & = & \argmax_{\bfgamma\in\uball[*]}
    \bfgamma^\intercal \bfbeta \label{eq:optimal_gamma}
%   \enspace,
\end{eqnarray}
%
% which defines ``the worst case penalty'' in $\bfbeta$, 
is usually straightforward to solve.
Besides the sparsity of $\hat{\bfbeta}$, the overall efficiency of our algorithm
relies also on the invariance of $\hat{\bfgamma}$ with respect to
large changes in $\bfbeta$. 
For the penalties we are interested in, $\hat{\bfgamma}$ takes its value in a
finite set, defined by the extreme points of the convex polytope $\uball[*]$.
This number of points typically increases exponentially in $p$, but, with the working-set
strategy, the number of configuration actually visited typically grows linearly
with the number of non-zero coefficients in the solution $\hat{\bfbeta}$.

\subsection{Relations with Other Methods}

The expansion in dual norm expressed in Problem \eqref{eq:general:primal} bears
some similarities with the first step of the derivation of very general duality
schemes, such as Fenchel's duality or Lagrangian duality.
It is however dedicated to the category of problems expressed as in
\eqref{eq:general:original}, thereby offering an interesting novel view of this
category of problems.
In particular, it provides geometrical insights on these methods and a generic
algorithm for computing solutions.  The associated algorithm, that relies on
solving linear systems is accurate, and efficient up to medium scale problems
(thousands of variables).

\subsection{Geometrical Interpretation}

% Penalized approaches can be formulated as constrained optimization, 
% where the penalty $\lambda \Omega(\bfbeta)$ is replaced by the hard constraint  
% % problems of the form minimize $f (\bfbeta; data)$, such that 
% $\Omega(\bfbeta)\leq c$. 
% This constrained formulation has, among other things, a geometric interpretation: 
% the solution belongs to the geometrical shape defined by  $\Omega(\bfbeta)\leq c$.

Geometrical insights are easier to gain from a slightly different formulation of
Problem \eqref{eq:general:primal}:~\footnote{%
The quadratic formulation corresponds to Problem \eqref{eq:general:primal} in 
the limit of $\eta\rightarrow+\infty$ if the feasible set for $\bfgamma$ is
defined as $\left\{\bfgamma\in\mathbb{R}^p:\norm[*]{\bfgamma}\leq
\eta\right\}$
}
% can rewritten as
% will be solved by
% considering an equivalent form amenable to a simpler resolution in
%$ \bfbeta$ for any $\bfgamma$, that is:
%
\begin{equation}\label{eq:general:dual}
  \min_{\bfbeta\in\mathbb{R}^p} \max_{\bfgamma\in\uball[*]}
    \frac{1}{2} \norm{\bfX \bfbeta - \bfy}^2 + \frac{\lambda}{2} \norm{\bfbeta - \bfgamma}^2
  \enspace.
\end{equation}
%where there is a one-to-one mapping between $\eta_X$ and $\lambda$.
%if the norm of $\bfgamma$ is finite.  In this later case, the penalty
The penalty $\lambda\norm{\bfbeta - \bfgamma}^2$ corresponds to a hard
constraint $\norm{\bfbeta - \bfgamma}^2\leq c$, which states that the 
solution in $\bfbeta$ belongs to a $\ell_{2}$ ball centered in $\bfgamma$.
Then, as $\hat{\bfgamma}$ maximizes $\norm{\hat{\bfbeta} - \bfgamma}^2$, the
solution $\hat{\bfbeta}$ belongs to the intersection of all the balls
centered in $\bfgamma\in\uball[*]$.
Eventually, the active constraints will be defined by the $\bfgamma$ values for
which $\norm{\hat{\bfbeta} - {\bfgamma}^2}$ is maximal, that is for the
worst-case $\hat{\bfgamma}$ values. 
For the penalties we are interested in, $\hat{\bfgamma}$ takes its value in a
finite set, defined by the extreme points of the convex polytope $\uball[*]$.
% 
% 
% Various classical sparse problems such as may be expressed by means of
% such a quadratic penalty. 
This is the case for the Lasso,
the $\ell_{1,\infty}$ version of the group-lasso 
(where the magnitude of regression coefficients are assumed to be equal within
groups, either zero or non-zero), 
and for OSCAR (Octagonal Shrinkage and Clustering Algorithm for Regression)
which is based on a penalizer encouraging the sparsity of the regression
coefficients and the equality of the non-zero entries \citep{Bondell08}.
%
Figure \ref{fig:penalties} illustrates  those three  sparse problems
with their  associated worst case quadratic penalty.


\ifverylong
% \subsection{More General Assumptions}
% 
% We now consider the more general assumption where the regularity of $\bfbeta^\star$ 
% is measured by the following norm:
% \begin{eqnarray*}
%  \left\| \bfbeta \right\| & = & \min_{\bftheta\in\clO} \min(-\bftheta^\intercal\bfbeta,\bftheta^\intercal\bfbeta)  \\
%  \text{with} \enspace 
%  \clO & = & \left\{ \{\bftheta_1,\ldots,\bftheta_r\}: \bftheta_i \in \Rset^p \ 
%    \text{and} \ 
%    \bfTheta = (\bftheta_1,\ldots,\bftheta_r) 
%    \ \text{is of rank $p$} \right\}
% \end{eqnarray*}
% 
\begin{figure}
  \begin{center} 
    \xylabelsquare{../figures/en_decomposition}{$\beta_1$}{$\beta_2$}{Elastic Net}% 
    \xylabelsquare{../figures/linf_decomposition}{$\beta_1$}{$\beta_2$}{$\ell_\infty$} 
    \xylabelsquare{../figures/oscar_decomposition}{$\beta_1$}{$\beta_2$}{OSCAR}%
    \caption{Penalty shapes (patches) built from the quadratic functions whose
             isocontour are displayed in white.}
    \label{fig:penalties}
    \end{center} 
\end{figure}
\fi



% We do consider  regression problems where $\hat{\bfbeta}$ minimizes 
% \begin{equation*}
%       \hat{\bfbeta} = \argmin_{\bfbeta\in\mathbb{R}^p} \left\{ \max_{\bfgamma \in \clD_{\bfgamma}} 
%       \norm{\boldmath{X} \bfbeta - \boldmath{y}}^2 + \lambda \norm{\bfbeta - \bfgamma}^2 \right\},
% \end{equation*}
% where 
%     \begin{itemize}
%     \item $\clD_\bfgamma$ describes an uncertainty set for the parameters,
%     \item $\bfgamma$ acts as a spurious \emph{adversary} over the true $\bfbeta^\star$.
%     \end{itemize}
%  Maximizing over $\clD_\bfgamma$ leads to the worst-case formulation. 
%  Choosing  $\clD_\bfgamma$  allows  to  recover many known $\ell_1$ penalizer via $\bfgamma^\intercal \bfbeta$,
%  
% The  $\norm{\bfgamma}^2$  does not change the minimization and may be
% discarded leading to 
%      \begin{equation*}
%       \minimize_{\bfbeta\in\mathbb{R}^p} \left\{  \norm{\boldmath{X} \bfbeta - \boldmath{y}}^2
%         + \lambda \norm{\bfbeta}^2 -  \lambda \max_{\bfgamma \in \clD_{\bfgamma}} \bfgamma^\intercal \bfbeta
%        \right\},
%     \end{equation*}


\section{Assumptions on the Spurious Regression Coefficients \label{sec:quadra}}
\label{sec:gammaperturb}

Our framework is amenable to many variations.
Here, we simply present two examples following the same pattern:
assuming a given regularity on the regression coefficients $\bfbeta^\star$, we
consider the adversarial dual assumption on the spurious coefficients
$\bfgamma$.
When the initial regularity conditions on $\bfbeta^\star$ are expressed by
$\ell_1$ or $\ell_\infty$ norms, this process results in uncertainty sets
$\mathcal{D}_{\bfgamma}$ which are convex polytopes that are
easy to manage when solving Problem~\eqref{eq:general:dual}, since they
can be defined as the convex hulls of a finite number possible perturbations.

% In this paper, we restrict our examples to 
The two sparsity-inducing penalizers presented below have a grouping effect.
The elastic net implements this grouping without predefining the group
structure: strongly correlated predictors tend to be in or out of the model
together \citep{2005_JRSS_Zou}.  
The $\ell_{\infty,1}$ group-Lasso that is presented subsequently is based on a
prescribed group structure and favors regression coefficients with identical
magnitude within activated groups.

\subsection{Elastic Net} \label{sec:elasticnet}

As an introductory example, let us consider the assumption stating that the
$\ell_1$-norm of $\bfbeta^\star$ should be small. 
The dual norm is the $\ell_\infty$-norm:
%
\begin{align*}
  \uball[*]^\eta & = \left\{ \bfgamma \in \mathbb{R}^p :
\sup_{\norm[1]{\bfbeta}\leq1} \bfgamma^\intercal\bfbeta \leq \eta \right\} \\
    & = \left\{ \bfgamma \in \mathbb{R}^p : \norm[\infty]{\bfgamma} \leq \eta \right\} \\
    & = \mathbf{conv} \big\{ \left\{ -\eta, \eta \right\}^p \big\}
  \enspace,
\end{align*}
where $\mathbf{conv}$ denotes convex hull, so that Problem
\eqref{eq:general:dual} reads:
%
\begin{align}
  & \min_{\bfbeta\in\mathbb{R}^p} \max_{\bfgamma \in\uball[*]^\eta}
      \Big\{ \frac{1}{2} \norm{\bfX \bfbeta - \bfy}^2 + \frac{\lambda}{2} \norm{\bfbeta - \bfgamma}^2 
      \Big\} \nonumber \\
  \Leftrightarrow
    & \min_{\bfbeta\in\mathbb{R}^p}
       \frac{1}{2} \norm{\bfX \bfbeta - \bfy}^2 + \lambda \eta
       \norm[1]{\bfbeta} + \frac{\lambda}{2} \norm{\bfbeta}^2
  \enspace, \label{eq:elastic-net}
\end{align}
%
which is recognized as an elastic net problem.
When $\eta$ is null, we recover ridge regression, and when $\eta$ goes to 
infinity, the problem approaches a Lasso problem.
A 2D pictorial illustration of this evolution is given in
Figure~\ref{fig:en-penalty}, where the shape of the uncertainty set
$\uball[*]^\eta$ is the convex hull of the points located at 
$(\pm \eta, \pm \eta)^\intercal$, which are identified by the cross markers.
Then, the sublevel set 
$\{\bfbeta : \max_{\bfgamma \in \uball[*]^\eta} \norm{\bfbeta-\bfgamma}^2 \leq t\}$
is simply defined as the intersection of the four sublevel sets
$\{\bfbeta : \norm{\bfbeta - \bfgamma}^2 \leq t\}$ for $\bfgamma=(\pm
1, \pm 1)^\intercal$, which are Euclidean balls centered at
these $\bfgamma$ values.
%
\begin{figure}
  \begin{center} 
    \smallxylabelsquare{../figures/en_decomposition1}{$\beta_1$}{$\beta_2$}{}%
    \smallxylabelsquare{../figures/en_decomposition2}{$\beta_1$}{$\beta_2$}{}%
    \smallxylabelsquare{../figures/en_decomposition3}{$\beta_1$}{$\beta_2$}{}%
    \smallxylabelsquare{../figures/en_decomposition4}{$\beta_1$}{$\beta_2$}{}%
    \caption{Sublevel sets for elastic net penalties (represented by the darker
             colored patches).  
             Each set is defined as the intersection of the the Euclidean balls
             (represented by the lighter color patches) whose centers are
             represented by crosses.}
    \label{fig:en-penalty}
    \end{center} 
\end{figure}

 

% \subsection{Pairwise Fused Lasso}

The sparsity inducing penalties can be adapted to pursue different goals, such
as having equal coefficients.  This was first implemented for ordered features
with the fused Lasso \citep{Tibshirani05}, which encourages sparse and locally
constant solutions by penalizing the $\ell_1$-norm of both the coefficients and
their successive differences.

The pairwise fused Lasso \citep{Petry11} does not assume that predictors are
ordered.  It selects features and favors some grouping by penalizing the
$\ell_1$-norm of both the coefficients and the differences between all pairs,
thus considering the following hypothesis space for
$\bfbeta^*$:
%
\begin{equation*}
  \mathcal{\clH}^\mathrm{PFL}_{\bfbeta^*} = \left\{ \bfbeta \in \mathbb{R}^p : 
    \norm[1]{\bfbeta}  + c
    \sum_{j=1}^{p} \sum_{k<j} \left| \beta_{j} - \beta_k \right| \leq \eta_\beta
  \right\}
  \enspace,
\end{equation*}
%
whose dual set is:
%
\begin{align*}
  \mathcal{D}^\mathrm{PFL}_{\bfgamma} & = 
    \left\{ \bfgamma \in \mathbb{R}^p : 
            \sup_{\bfbeta \in \mathcal{\clH}^\mathrm{PFL}_{\bfbeta^*}}
            \bfgamma^\intercal\bfbeta \leq 1 
    \right\}
    \\
    & = \left\{ ? \right\}
  \enspace,
\end{align*}
{\color{red}{YG: I believe that the definition of $\bfgamma$ is wrong for the fused Lasso}}
\begin{equation*}
  \bfD = 
  \begin{pmatrix}
      1      &  0     & 0      & \cdots & 0\\
     -1      &  1     & 0      & \ddots & \vdots  \\
      0      & -1     & 1      & \ddots & 0 \\
     \vdots  & \ddots & \ddots & \ddots & 0 \\
      0      & \cdots & 0      & -1     & -1 
  \end{pmatrix}
  \enspace,
\end{equation*}
%
so that Problem \eqref{eq:robust:general:form3} reads:
%
\begin{align*}
  & \min_{\bfbeta\in\mathbb{R}^p} \max_{\bfgamma_1 \in \left\{ -\eta_\gamma, \eta_\gamma \right\}^p}
    \max_{\bfgamma_2 \in \left\{ -\nu_\gamma, \nu_\gamma \right\}^{p-1}}
      \Big\{ \norm{\bfX \bfbeta - \bfy}^2 + \lambda \norm{\bfbeta - \bfgamma_1 - \bfD \bfgamma_2}^2 \Big\} \\
  \Leftrightarrow
    & \min_{\bfbeta\in\mathbb{R}^p}
      \norm{\bfX \bfbeta - \bfy}^2 + \lambda \eta_\gamma \norm[1]{\bfbeta} + \lambda \norm{\bfbeta}^2 
  \enspace,
\end{align*}
%


The Lagrangian formulation of the fused Lasso optimization problem is expressed
as:
\begin{equation}
  \min_{\bfbeta\in\mathbb{R}^p} \norm{\bfX \bfbeta - \bfy}^2 + 
    \lambda \sum_{j=1}^p \left|\beta_{j}\right| + \theta \sum_{j=1}^{p-1} \left| \beta_{j+1} - \beta_j \right|
  \enspace, 
\end{equation}


\subsection{Group-Lasso}

We consider here the $\ell_{\infty,1}$ variant of the group-Lasso, which was
first proposed by \citet{Turlach05} to perform variable selection in the
multiple response setup.
%
A group structure is defined on
the set of variables by setting a partition of the index set
$\mathcal{I}=\{1,\ldots,p\}$, that is,
$
  \mathcal{I}=\bigcup_{k=1}^K\group \enspace,\, \text{with}\enspace 
  \group \cap \group[\ell]=\emptyset \enspace
  \text{for}\enspace k\neq\ell \enspace.
$
%
Let $p_k$ denote the cardinality of group $k$, and $\bfbeta_{\group} \in
\Rset^{p_k}$ be the vector $(\beta_j)_{j\in \group}$.
%


The $\ell_{\infty,1}$ mixed-norm of $\bfbeta$ (that is, its groupwise 
$\ell_\infty$-norm) is defined as
%
\begin{equation*}
  \uball = \left\{ 
    \bfbeta \in \mathbb{R}^p :\sum_{k=1}^K \norm[\infty]{\bfbeta_{\group}} \leq 1
  \right\}
  \enspace.
\end{equation*}
%
The dual norm is the groupwise $\ell_1$-norm:
%
\begin{align*}
  \uball[*]^\eta & = \left\{ \bfgamma \in \mathbb{R}^p :
\sup_{\bfbeta\in\uball} \bfgamma^\intercal\bfbeta \leq \eta \right\} \\
    & = \left\{ \bfgamma \in \mathbb{R}^p : \max_{k\in\{1,...,K\}}  \norm[1]{\bfgamma_{\group}} \leq \eta \right\} \\
    & = \mathbf{conv} \big\{ 
                        \left\{\eta\bfe^{p_1}_1, \ldots, \eta\bfe^{p_1}_{p_1},-\eta\bfe^{p_1}_1, \ldots, -\eta\bfe^{p_1}_{p_1} \right\} 
                        \times \ldots \\
    & \hspace*{4em} \times 
                        \left\{\eta\bfe^{p_K}_1, \ldots, \eta\bfe^{p_K}_{p_K},-\eta\bfe^{p_K}_1, \ldots, -\eta\bfe^{p_K}_{p_K} \right\} 
                      \big\}
  \enspace,
\end{align*}
where $\bfe^p_j$ is the $j$th element of the canonical basis of $\Rset^p$.
Problem \eqref{eq:general:dual} becomes:
%
\begin{align*}
  & \min_{\bfbeta\in\mathbb{R}^p} \max_{\bfgamma \in \uball[*]^\eta}
      \Big\{ \frac{1}{2} \norm{\bfX \bfbeta - \bfy}^2 + \frac{\lambda}{2} \norm{\bfbeta - \bfgamma}^2 \Big\} \\
  \Leftrightarrow
    & \min_{\bfbeta\in\mathbb{R}^p}
      \frac{1}{2} \norm{\bfX \bfbeta - \bfy}^2 + \lambda \eta \sum_{k=1}^K \norm[\infty]{\bfbeta_{\group}} + \frac{\lambda}{2} \norm{\bfbeta}^2 
  \enspace,
\end{align*}
%

Notice that the limiting cases of this penalty are two classical problems: ridge
regression and the $\ell_{\infty,1}$ group-Lasso.
A 2D pictorial illustration of this evolution is given in
Figure~\ref{fig:group-penalty}, where $\uball[*]^\eta$ is the convex hull of the
points located on the axes at $\pm \eta$, which are identified by the cross
markers.
Then, the sublevel set 
$\{\bfbeta : \max_{\bfgamma \in \uball[*]^\eta} \norm{\bfbeta-\bfgamma}^2 \leq t\}$
is simply defined as the intersection of the four sublevel sets
$\{\bfbeta : \norm{\bfbeta - \bfgamma}^2 \leq t\}$ for 
$\bfgamma=\pm \eta\bfe^{2}_1$ and $\bfgamma=\pm \eta\bfe^{2}_2$,
which are Euclidean balls centered at these $\bfgamma$ values.
%
\begin{figure}
  \begin{center} 
    \smallxylabelsquare{../figures/linf_decomposition1}{$\beta_1$}{$\beta_2$}{}%
    \smallxylabelsquare{../figures/linf_decomposition2}{$\beta_1$}{$\beta_2$}{}%
    \smallxylabelsquare{../figures/linf_decomposition3}{$\beta_1$}{$\beta_2$}{}%
    \smallxylabelsquare{../figures/linf_decomposition4}{$\beta_1$}{$\beta_2$}{}%
    \caption{Sublevel sets for the $\ell_{\infty,1}$ group-Lasso penalties in $\mathbb{R}^2$.
             In each plot, the darkest colored patch corresponds to a sublevel set $\norm[]{\bfbeta} \leq c$.
             Each set is defined as the intersection of the four Euclidean balls
             (represented as light color layers) whose centers are
             represented by crosses (not visible because out of frame on the 
             rightmost example).}
    \label{fig:group-penalty}
    \end{center} 
\end{figure}

\iffalse
The  lagrangian formulation  of the  $\ell_{\infty,1}$ version  of the
group-Lasso as a constrained optimization can be expressed as
$$
 \min_{\bfbeta}     \norm{\bfX  \bfbeta  - \bfy  }^2+\lambda \sum_{g=1}^G \|\bfbeta_g\|_\infty,
$$
with $\bfbeta_g=(\beta_{gk})$ the coefficients of $\bfbeta$ corresponding to group $g$.


The penalty term can be expressed in a form close to our adverse quadratic penalty. Let us consider  the adverse vector domain to be
$$
 \clD_{\bfgamma}=\left\{ \bfgamma \in \mathbb{R}^p | \bfgamma= 
(\alpha_j \mathbb{I}_{(\mbox{rank}\left(  \max_k |\beta|_{gk}\right)=j) })_{j=1
      \cdots p } 
\norm[\infty]{(\alpha_1,\cdots,\alpha_p)} \leq \eta_\alpha \right\} .
$$

We can reformulate the previous lagrangian as 
\begin{equation}
    \min_{\bfbeta\in\mathbb{R}^p} \max_{\bfgamma \in \clD_{\bfgamma}}
    \norm{\bfX \bfbeta - \bfy } + \lambda \norm{\bfbeta +
    \bfgamma } \enspace.
\end{equation}

This  rewriting  of the  problem  allows to  see  that  the very  same
optimization adaptive  constraint algorithm  used for the  elastic-net
can be used to solve the Group  $\ell_{\infty,1}$ problem.


{\color{red}{YG : Could'nt we derive the usual group-Lasso from the robust
optimization viewpoint, simply by changing the definition of $\clD_{\bfX}$, with
groupwise Frobenius norms?}

Christophe: Si  tu fais cela tu écris  bien le group Lasso  mais je ne
vois pas comment cela permet de faciliter la résolution du problème avec les mêmes
techniques que pour les autres....
}

{\color{blue}{Christophe:  une version group  oscar devrait  donner de
    meilleurs résultats}}
\fi

%\input{oscar}




%
% LASSO
% RIDGE
% ELASTIC NET
% STRUCTURE ELASTIC NET
% FUSED
% OSCAR

\section{Algorithm}\label{sec:algo}

The unified derivation for the problems presented in Section
\ref{sec:gammaperturb} suggests a unified processing based on the iterative
resolution of quadratic problems. 
This general algorithm is summarized in this section.
We then show that the new derivation can also be used in the analysis of this 
algorithm by describing an alternative to Fenchel duality 
\citep[used for example by][]{2012_FML_Bach} to assess convergence.
% by computing an upper-bound for the gap between the current solution and the
% optimal one.

\iflong
  \subsection{Active Set Approach}
\fi

The efficient approaches developed for sparse regression take advantage of the sparsity
of the solution by solving a series of small linear systems, the sizes of which are
incrementally increased/decreased.  
Here, as for the Lasso~\citep{2000_JCGS_Osborne,2004_AS_Efron}, this process boils down to an
iterative optimization scheme involving the resolution of quadratic problems.

The algorithm is based on the iterative update of the set of ``active''
variables, $\supp$, indexing the coefficients $\bfbeta_{\!\supp}$. A
variable $j$ is part of the active set when a single value $\gamma_j$ maximizes the criterion.
In other words a variable is active if it does not generate a singular point of the criterion.
Optimization with respect to all the active variables boils then down to a simple resolution of a
quadratic problem. 
For the elastic-net, a non-zero coefficient defines an active variable; more
generally, a zero coefficient defines an inactive variable.

The active set is typically started from a sparse initial guess %, say $\supp=\emptyset$ ($\bfbeta=0$),
and iterates the three following steps:
\begin{enumerate}
\item \label{item:algo:step1} the first step solves
  Problem~\eqref{eq:general:dual} considering that $\supp$, the set of
  ``active'' variables, is correct; that is, the
  objective~\eqref{eq:general:dual} is optimized with respect to
  $\bfbeta_{\!\supp}$.  This penalized least squares problem is defined
  from $\mathbf{X}_{\centerdot\supp}$, which is the submatrix of
  $\mathbf{X}$ comprising all rows and the columns indexed by $\supp\,$
  and $\bfgamma_{\!\supp}$, which is set to
  its current most adversarial value.~\footnote{%
    When several $\bfgamma_{\!\supp}$ are equally unfavorable to
    $\bfbeta_{\!\supp}$, we use gradient information to pick the worst one
    among those when $\bfbeta_{\!\supp}$ moves along the steepest descent
    direction.
  }

\item \label{item:algo:step2} the second step updates $\bfbeta_{\!\supp}$
  if necessary (and possibly $\bfgamma_{\!\supp}$), so that
  $\bfgamma_{\!\supp}$ is indeed (one of) the most adversarial value of
  the current $\bfbeta_{\!\supp}$.
  This is easily checked with the problems given in
  Section~\ref{sec:gammaperturb}, where $\uball[*]^\eta$ is a convex polytope
  whose vertices (that is, extreme $\bfgamma$-values) are associated with a
  cone of coherent $\bfbeta$-value.
%  defined by the intersection of the hyperplanes that cut the middle of
%  the neighboring edges, normally to these edges.

\item  \label{item:algo:step3} the last step updates the active set $\supp$. 
  It relies on the ``worst-case gradient'' with respect to
  $\bfbeta$, where $\bfgamma$ is chosen so as to minimize infinitesimal
  improvements of the current solution.
  Again picking the right $\bfgamma$ is easy for the problems given in
  Section~\ref{sec:gammaperturb}.
  Once this is done, we first check whether some variables should quit the
  active set, and if this is not the case, we assess the completeness of
  $\supp$, by checking the optimality conditions with respect to inactive
  variables.  We add the variable, or the group of variables that most violates
  the worst-case optimality conditions.  When no such violation exists, the
  current solution is optimal, since, at this stage, the problem is solved
  exactly within the active set $\supp$.
\end{enumerate}
Algorithm~\ref{algo:active_set} provides a more comprehensive technical
description.

%
\begin{algorithm}[htbp]
  \begin{small}
 \SetSideCommentLeft
%  \DontPrintSemicolon
 \nlset{Init.} $\bfbeta \leftarrow \bfbeta^0$\\
  Pick a worst admissible $\bfgamma$ in $\boldsymbol\Gamma = \left\{\argmax_{\bfgamma' \in \uball[*]^\eta} \left\| \bfgamma' - \bfbeta \right\|_2^2 \right\}$ \\ 
  Determine the active set:  
%   $\supp \leftarrow \{j: \exists \bfgamma, \bfgamma' \in \boldsymbol\Gamma \left\| \bfgamma - \bfbeta \right\|_2^2 = \left\| \bfgamma' - \bfbeta \right\|_2^2 \Rightarrow \bfgamma_j' = \bfgamma_j \}$\\  
  $\supp \leftarrow \{j: \forall (\bfgamma, \bfgamma') \in \boldsymbol\Gamma, \enspace \gamma_j= \gamma_j' \}$\\  
  \BlankLine 
  \nlset{Step 1} Update active variables $\bfbeta_{\!\supp}$ assuming that $\supp$ and $\bfgamma_{\!\supp}$ are optimal \\
  $\bfbeta_{\!\supp}^\mathrm{old} \leftarrow \bfbeta_{\!\supp}$ \\
  $\mathbf{r} \leftarrow \mathbf{y} - \mathbf{X}_{\centerdot\supp^c}^{\phantom{\intercal}} \bfbeta_{\!\supp^c}$ \\  
  $\bfbeta_{\!\supp} \leftarrow 
   \left(\mathbf{X}_{\centerdot\supp}^\intercal
      \mathbf{X}_{\centerdot\supp}^{\phantom{\intercal}} + \lambda \mathbf{I}_{|\supp|}\right)^{-1}
    \left(\mathbf{X}_{\centerdot\supp}^\intercal\mathbf{r} +
      \lambda \bfgamma_{\!\supp}\right)  
  $ \\ 
  \nlset{Step 2} Verify coherence of $\bfgamma_{\!\supp}$ with the updated $\bfbeta_{\!\supp}$\\
  \If(\tcp*[f]{if $\bfgamma_{\!\supp}$ is not worst-case})
   {$\displaystyle\norm{\bfbeta_{\!\supp} - \bfgamma_{\!\supp}}^2 < \max_{\bfg \in \uball[*]^\eta} \norm{\bfbeta_{\!\supp}-\bfg_{\!\supp}}^2$}
   {%
    \tcc{Backtrack towards the last $\bfgamma_{\!\supp}$-coherent solution:}
    $\bfbeta_{\!\supp} \leftarrow \bfbeta_{\!\supp}^\mathrm{old} + \rho (\bfbeta_{\!\supp} - \bfbeta_{\!\supp}^\mathrm{old})$ \\
    $\bfgamma_{\!\supp}$ is worst-case for $\bfbeta_{\!\supp}$, and there is another worst-case value $\widetilde\bfgamma_{\!\supp}$ \\
    \tcc{Check whether progress can be made with $\widetilde\bfgamma_{\!\supp}$}
    $\widetilde\bfbeta_{\!\supp} \leftarrow
    \left(\mathbf{X}_{\centerdot\supp}^\intercal
      \mathbf{X}_{\centerdot\supp}^{\phantom{\intercal}} + \lambda \mathbf{I}_{|\supp|}\right)^{-1}
    \left(\mathbf{X}_{\centerdot\supp}^\intercal\mathbf{r} +
      \lambda \widetilde\bfgamma_{\!\supp}\right)  
  $ \\ 
  \If(\tcp*[f]{if $\widetilde\bfgamma_{\!\supp}$ is worst-case\ldots})
   {$\displaystyle\norm{\widetilde\bfbeta_{\!\supp} - \widetilde\bfgamma_{\!\supp}}^2 = \max_{\bfg \in \uball[*]^\eta} \norm{\widetilde\bfbeta_{\!\supp}-\bfg_{\!\supp}}^2$}
     {%
%        $\bfbeta_{\!\supp}^\mathrm{old} \leftarrow \bfbeta_{\!\supp}$ \\
       $(\bfbeta_{\!\supp},\bfgamma_{\!\supp}) \leftarrow
          (\widetilde\bfbeta_{\!\supp},\widetilde\bfgamma_{\!\supp})$
        \tcp*[f]{$ (\widetilde\bfbeta_{\!\supp},\widetilde\bfgamma_{\!\supp})$ is better than $(\bfbeta_{\!\supp},\bfgamma_{\!\supp})$}
     }
    }
    \tcc{The current $\bfgamma_{\!\supp}$ is coherent with $\bfbeta_{\!\supp}$}

  \nlset{Step 3} Update active set $\supp$ with worst-case $\bfgamma_{\!\supp}$ \\
  $\displaystyle g_j \leftarrow \min_{\bfgamma \in \uball[*]^\eta}\left|
    \mathbf{x}_j^\intercal(\mathbf{X}_{\centerdot\supp}^{\phantom{\intercal}}\bfbeta_{\!\supp} - \mathbf{r})  + \lambda (\beta_j - \gamma_j) \right|
  \enspace j=1,\ldots,p$
  \tcp*[f]{worst-case gradient} \\
  \eIf{ $\exists\, j \in\supp:  \left\| \bfgamma - \bfbeta \right\|_2^2 = \left\| \bfgamma' - \bfbeta \right\|_2^2 \ and  \ \gamma_j \neq \gamma_j'  \ \text{and}\ g_j = 0$}{
       $\displaystyle \supp \leftarrow \supp\backslash\{j\}$ \tcp*[r]{Downgrade $j$} 
       \tcc{Go to Step 1}
   }{%
  \eIf{$\max_{j\in\supp^c} g_j \neq0$}{ 
   \tcc{Identify the greatest violation of optimality conditions}
    $\displaystyle j^\star \leftarrow \argmax_{j\in\supp^c} g_j \,,\enspace$
    $\supp \leftarrow \supp \cup \{j^\star\}$ \tcp*[r]{Upgrade $j^\star$} 
    \tcc{Go to Step 1}
  }{
   \tcc{Stop and return $\bfbeta$, which is optimal}
   }
  }
  \end{small} 
\caption{Worst-Case Active Set Algorithm}
\label{algo:active_set}
\end{algorithm}
%
Note that the structure is essentially identical to the one proposed by
\citet{2000_JCGS_Osborne} or \cite{2004_AS_Efron} for the Lasso, but that it 
applies to any penalty that can be decomposed as in 
Problem~\eqref{eq:general:dual}.
Our viewpoint is also radically different, as the global non-smooth problem
is dealt with subdifferentials by \citet{2000_JCGS_Osborne}, whereas we rely on
the maximum of smooth functions.
This approach suggests a new assessment of convergence, as detailed below.
%

% \iflong
  \subsection{Monitoring Convergence}

  At each iteration of the algorithm, the current $\bfbeta$ is computed assuming
  that the current active set $\supp$ and the current
  $\bfgamma_\supp$-value are optimal.
  When the current active set is not optimal, the current $\bfbeta$ (where
  $\bfbeta_\supp$ is completed by zeros on the complement $\supp^c$) is
  nevertheless optimal for a $\bfgamma$-value defined in $\Rset^p$ (where
  $\bfgamma_\supp$ is completed by ad hoc values on the complement
  $\supp^c$). However this $\bfgamma$ fails to belong to $\uball[*]^\eta$ 
  (otherwise, the problem would be solved: $\supp$, $\bfgamma$ and $\bfbeta$
  would indeed be optimal).
  The following proposition relates the current objective function, associated
  with an infeasible $\bfgamma$-value ($\bfgamma\notin\uball[*]^\eta$), to the
  global optimum of the optimization problem.

  \begin{proposition}\label{prop:monitoring}
    For any vectorial norm $\norm[*]{\cdot}$, 
%     when 
%     $\uball[*]^\eta$ is defined as 
%     $\uball[*]^\eta=\left\{\bfgamma \in \mathbb{R}^{p}:
%                         \norm[*]{\bfgamma} \leq \eta \right\}$, 
%     then, 
    $\forall \bfgamma \in \mathbb{R}^{p}:\norm[*]{\bfgamma} \geq \eta$, we have:
    \begin{equation*}
      \min_{\bfbeta\in\mathbb{R}^{p}} \max_{\bfgamma' \in \uball[*]^\eta} 
      J_\lambda(\bfbeta,\bfgamma') 
      \geq
      \frac{\eta}{\norm[*]{\bfgamma}} 
      J_\lambda\left(\bfbeta^\star\left(\bfgamma\right), \bfgamma \right) -
      \frac{\lambda\eta(\norm[*]{\bfgamma}-\eta)}{\norm[*]{\bfgamma}^2}\norm{\bfgamma}^2        
      \enspace,
    \end{equation*}
    where 
    \begin{equation*}
      J_\lambda(\bfbeta,\bfgamma) = \norm{\bfX \bfbeta - \bfy}^2 + 
        \lambda \norm{\bfbeta - \bfgamma}^2
      \enspace \text{and} \enspace
      \bfbeta^\star(\bfgamma) = \argmin_{\bfbeta\in\mathbb{R}^{p}} J_\lambda(\bfbeta,\bfgamma)
      \enspace.
    \end{equation*}
    See proof in  \ref{sec:proof:prop:monitoring}.
  \end{proposition}

  This proposition can be used to compute an optimality gap at Step 3 of
  Algorithm \ref{algo:active_set}, by picking a $\bfgamma$-value such that the
  current worst-case gradient $\bfg$ is null (the current $\bfbeta$-value then
  being the optimal $\bfbeta^\star(\bfgamma)$).
  Note that more precise upper bounds could be computed relying on significant extra computation.
  The generic optimality gap computed from Proposition \ref{prop:monitoring} differs
  from the Fenchel duality gap \citep[see][]{2012_FML_Bach}. 
  For the elastic-net expressed in \eqref{eq:elastic-net}, Fenchel inequality
  \citep[see details in][]{Mairal10} yields the following optimality gap:
   \begin{align*}
      \min_{\bfbeta\in\mathbb{R}^{p}} \max_{\bfgamma' \in \uball[*]^\eta} 
      J_\lambda(\bfbeta,\bfgamma') 
      \geq &
      J_\lambda\left(\bfbeta^\star\left(\bfgamma\right), \bfgamma \right) -
      \frac{\eta^{2}}{\norm[*]{\bfgamma}^{2}}
      \left(
        \norm{\bfX \bfbeta^\star\left(\bfgamma\right) - \bfy}^2 + 
        \lambda \norm{\bfbeta^\star\left(\bfgamma\right)}^{2}
      \right) \\
      & - \frac{2\eta}{\norm[*]{\bfgamma}}
      \left(
         \bfX \bfbeta^\star\left(\bfgamma\right) - \bfy
     \right)^\intercal \bfy
     \enspace.
   \end{align*}
  %
  \begin{figure}
    \centering 
    \xylabellarge{../figures/monitoring_bounds}{\# of iterations}{Optimality gap (log scale)}{$n=50$, $p=200$}
    \caption{Monitoring convergence: true optimality gap (solid black) versus
    our generic upper bound (dashed blue) and Fenchel's duality gap for
    elastic-net (dotted red) computed at each iteration of
    Algorithm~\ref{algo:active_set}.}
    \label{fig:monitoring}
  \end{figure}
  %
  The two optimality gaps are empirically compared in 
  Figure~\ref{fig:monitoring} for the elastic-net, along a short regularization 
  path with five values of the $\ell_{1}$-penalization parameter.  
  We see that the two options can be used to assess convergence, though
  Fenchel's duality gap is more accurate for the rougher solutions. 
  Note however that both upper bounds are fairly coarse until a very accurate
  solution is reached, which makes them both unsuitable for deriving loose stopping 
  criteria.
%   Fenchel's duality gap is more accurate here, especially for the rougher
%   solutions.  However, the practical use of these upper bounds is rather limited
%   as they are both fairly coarse unless a very accurate solution is reached.
%   These optimality gaps may thus be used to assess convergence, but they are
%   ineffective as stopping rules when fast and rough solutions are sought.
  Proposition~\ref{prop:monitoring} is thus of limited scope, but it illustrates
  that, besides its algorithmic consequences, our original view of sparse
  penalties opens new ways for analysis.
  As a final note on this topic, we provide a slightly tighter inequality for
  computing the optimality gap in~\ref{sec:proof:prop:monitoring}, and
  we conjecture that it could be further tightened (see in particular the
  derivation of inequality \eqref{eq:crude_inequality}).

  % \iffalse
  %   What is a good value for $\bfgamma$?
  %   Computationally, it is a value for which we already have computed 
  %   $\min_{\bfbeta\in\mathbb{R}^{p}} J_\lambda(\bfbeta,\bfgamma)$, regarding the 
  %   tightness of the bound, ${\norm[*]{\bfgamma}}$ should be as close to 
  %   $\eta_\gamma$ as possible, while $\min_{\bfbeta\in\mathbb{R}^{p}} 
  %   J_\lambda(\bfbeta,\bfgamma)$ should be as large as possible.
    
  %   Hence, a tight optimality gap can be computed provided one can guess a
  %   near-optimal $\bfgamma$-value.
  %   When entering Step \ref{item:algo:step3} of Algorithm~\ref{algo:active_set}, the current best guess
  %   $\widehat\bfgamma_{\supp}$ has to be completed on the complement $\supp^c$.
  %   We propose to do so by minimizing the magnitude of the gradient with respect to $\bfbeta$ at the current solution:
  %   \begin{align*}
  %     \widehat\bfgamma_{\supp^c} & = \argmin_{\bfgamma_{\supp^c}:\bfgamma \in \clD_{\bfgamma}}
  %                                \left|  \bfX_{\centerdot\supp^c}^\intercal(\bfX_{\centerdot\supp}\bfbeta_{\supp}-\bfy) - 
  %                                \lambda \bfgamma_{\supp^c} \right|
  %       \enspace.
  %   \end{align*}
  %   Finally, the full vector $\widehat\bfgamma$ can be used to compute the lower
  %   bound on the optimal objective function $J(\widehat{\bfbeta}
  %   (\widehat\bfgamma),\widehat\bfgamma)$. 
  %   This lower bound, which requires to solve a linear system of size $p$, is
  %   lighter to compute than the duality gap \citep[see details
  %   in][]{2012_FML_Bach}, and was observed to provide a tight bound. As a sanity
  %   check, note that when $\bfbeta$ is optimal, the corresponding optimal
  %   $\widehat\bfgamma$ is known to be null on $\supp^c$, so that the bound is exact.
  %   Two typical behaviors are displayed in Figure~\ref{fig:monitoring} for the
  %   elastic-net.
  %   The two plots show the true distance to the optimum, measured by the
  %   difference in objective function, and the value of the bound, both computed at each
  %   iteration of Algorithm~\ref{algo:active_set} at the entrance of Step
  %   \ref{item:algo:step3}.
  %   The left-hand side plot shows the unfavorable case where $p$ is larger than
  %   $n$, where the bound is not as precise as in the opposite setup, displayed in
  %   the right-hand side plot, where it is very close to the true optimality
  %   gap.
  
  %   \begin{figure}
  %     \centering 
  %     \begin{tabular}{c@{\hspace*{3em}}c}
  %       \xylabelsquare{../figures/monitoring_small_n}{\# of iterations}
  %                     {Optimality gap}{$n=50$, $p=200$}% 
  %       \xylabelsquare{../figures/monitoring_large_n}{\# of iterations}
  %                     {Optimality gap}{$n=400$, $p=200$}% 
  %     \end{tabular}
  %     \caption{Monitoring convergence: true optimality gap (solid blue) versus our
  %       upper bound (dashed red) computed at each iteration of
  %       Algorithm~\ref{algo:active_set}.}
  %     \label{fig:monitoring}
  %   \end{figure}
  %   % $\rho=0.9$, $\lambda_1=0.5$, $\lambda_2=0.�1$, $R^2\simeq0.8$.
  %   \begin{align*}
  %     \bfbeta_\supp & = \argmin_{\bfbeta\in\mathbb{R}^{|\supp|}} 
  %                     \norm{\bfX_{\centerdot\supp} \bfbeta - \bfy}^2 + \lambda_1 \norm[1]{\bfbeta}
  %     \\
  %     \widehat\bfbeta_{\supp^c} & = \bfzero \\ 
  %     \widehat\bfgamma_{\supp^c} & = \argmin_{\bfgamma_{\supp^c} \in [-1,1]^{|\supp^c|}}
  %                              \left|  \bfX_{\centerdot\supp^c}^\intercal(\bfX\widehat\bfbeta-\bfy) + 
  %                                \lambda_1 \bfgamma_{\supp^c} \right|
  %       \enspace. 
  %   \end{align*}
  %   We can then compute 
  %   \begin{align*}
  %     \widetilde{\bfbeta} & = \argmin_{\bfbeta\in\mathbb{R}^{p}} 
  %                     \norm{\bfX \bfbeta - \bfy}^2 + \lambda_1 \bfbeta^\intercal\widehat\bfgamma
  %       \enspace. 
  %   \end{align*}
  %   We then have:
  %   \begin{align*}
  %   J(\bfbeta^\star) & = \min_{\bfbeta\in\mathbb{R}^p} 
  %     \norm{\bfX \bfbeta - \bfy}^2 + \lambda_1  \norm[1]{\bfbeta} \\
  %     &= \min_{\bfbeta\in\mathbb{R}^p} \max_{\bfgamma \in \clD_{\bfgamma}}
  %         \Big\{ \norm{\bfX \bfbeta - \bfy}^2 + \lambda_1 \bfbeta^\intercal\bfgamma \Big\} \\
  %     &= \max_{\bfgamma \in \clD_{\bfgamma}} \min_{\bfbeta\in\mathbb{R}^p}
  %         \Big\{ \norm{\bfX \bfbeta - \bfy}^2 + \lambda_1 \bfbeta^\intercal\bfgamma \Big\} \\
  %     & \geq \min_{\bfbeta\in\mathbb{R}^p}
  %         \Big\{ \norm{\bfX \bfbeta - \bfy}^2 + \lambda_1 \bfbeta^\intercal\widehat\bfgamma \Big\}  \\
  %     & = \norm{\bfX \widetilde{\bfbeta} - \bfy}^2 + \lambda_1 \widetilde{\bfbeta}^{(t)^\intercal}\widehat\bfgamma
  %     \enspace,
  %   \end{align*}
  %   which provides a lower bound for $J(\bfbeta^\star)$.
  %   Meanwhile, we obviously have:
  %   \begin{align*}
  %   J(\bfbeta^\star) & = \min_{\bfbeta\in\mathbb{R}^p} 
  %     \norm{\bfX \bfbeta - \bfy}^2 + \lambda_1 \norm[1]{\bfbeta} \\
  %     &\leq \norm{\bfX \widehat\bfbeta - \bfy}^2 + \lambda_1  \norm[1]{\widehat\bfbeta} 
  %     \enspace,
  %   \end{align*}
  %   So that $J(\bfbeta^\star)$ is easily bracketed during the optimization process. 
  %   Note that when $\bfbeta_\supp=\bfbeta^\star$, we have that
  %   $\widetilde{\bfbeta}=\bfbeta_\supp$.

  %   The additional computational cost for monitoring (that is, the cost of evaluating
  %   the upper and lower bounds for an intermediate solution can be important because
  %   $\widetilde{\bfbeta}$ requires to solve a linear system of size $p$\ldots 
  %   We could do better than that if we can prove that $\big(\widetilde{\bfbeta}\big)_j=0$ if
  %   the subgradient at  $\widehat\bfbeta_{\supp^c}$ comprises zero. In this situation, we 
  %   should only consider adding the variables for which the optimality conditions
  %   are violated.

  %   \subsection{Monitoring Convergence: Lasso-Case}

  %   \begin{align*}
  %     \bfbeta_\supp & = \argmin_{\bfbeta\in\mathbb{R}^{|\supp|}} 
  %                     \norm{\bfX_{\centerdot\supp} \bfbeta - \bfy}^2 + \lambda_1 \norm[1]{\bfbeta}
  %     \\
  %     \widehat\bfbeta_{\supp^c} & = \bfzero \\ 
  %     \widehat\bfgamma_{\supp^c} & = \argmin_{\bfgamma_{\supp^c} \in [-1,1]^{|\supp^c|}}
  %                              \left|  \bfX_{\centerdot\supp^c}^\intercal(\bfX\widehat\bfbeta-\bfy) + 
  %                                \lambda_1 \bfgamma_{\supp^c} \right|
  %       \enspace. 
  %   \end{align*}
  %   We can then compute 
  %   \begin{align*}
  %     \widetilde{\bfbeta} & = \argmin_{\bfbeta\in\mathbb{R}^{p}} 
  %                     \norm{\bfX \bfbeta - \bfy}^2 + \lambda_1 \bfbeta^\intercal\widehat\bfgamma
  %       \enspace. 
  %   \end{align*}
  %   We then have:
  %   \begin{align*}
  %   J(\bfbeta^\star) & = \min_{\bfbeta\in\mathbb{R}^p} 
  %     \norm{\bfX \bfbeta - \bfy}^2 + \lambda_1  \norm[1]{\bfbeta} \\
  %     &= \min_{\bfbeta\in\mathbb{R}^p} \max_{\bfgamma \in \clD_{\bfgamma}}
  %         \Big\{ \norm{\bfX \bfbeta - \bfy}^2 + \lambda_1 \bfbeta^\intercal\bfgamma \Big\} \\
  %     &= \max_{\bfgamma \in \clD_{\bfgamma}} \min_{\bfbeta\in\mathbb{R}^p}
  %         \Big\{ \norm{\bfX \bfbeta - \bfy}^2 + \lambda_1 \bfbeta^\intercal\bfgamma \Big\} \\
  %     & \geq \min_{\bfbeta\in\mathbb{R}^p}
  %         \Big\{ \norm{\bfX \bfbeta - \bfy}^2 + \lambda_1 \bfbeta^\intercal\widehat\bfgamma \Big\}  \\
  %     & = \norm{\bfX \widetilde{\bfbeta} - \bfy}^2 + \lambda_1 \widetilde{\bfbeta}^{(t)^\intercal}\widehat\bfgamma
  %     \enspace,
  %   \end{align*}
  %   which provides a lower bound for $J(\bfbeta^\star)$.
  %   Meanwhile, we obviously have:
  %   \begin{align*}
  %   J(\bfbeta^\star) & = \min_{\bfbeta\in\mathbb{R}^p} 
  %     \norm{\bfX \bfbeta - \bfy}^2 + \lambda_1 \norm[1]{\bfbeta} \\
  %     &\leq \norm{\bfX \widehat\bfbeta - \bfy}^2 + \lambda_1  \norm[1]{\widehat\bfbeta} 
  %     \enspace,
  %   \end{align*}
  %   So that $J(\bfbeta^\star)$ is easily bracketed during the optimization process. 
  %   Note that when $\bfbeta_\supp=\bfbeta^\star$, we have that
  %   $\widetilde{\bfbeta}=\bfbeta_\supp$.
  %   \fi
% \else
%   a simple lower bound on the objective function. 

%   Let $J$ be the objective function of the generic
%   Problem~\eqref{eq:general:dual}, we trivially have that, for any
%   $\bfgamma$-value,
%   $\min_{\bfbeta\in\mathbb{R}^{p}} \max_{\bfgamma \in \clD_{\bfgamma}}
%   J(\bfbeta,\bfgamma) \geq \min_{\bfbeta\in\mathbb{R}^{p}} J(\bfbeta,\bfgamma)$.
%   %
%   Hence, a tight optimality gap can be computed provided one can guess a
%   near-optimal $\bfgamma$-value.
%   When entering Step \ref{item:algo:step3} of Algorithm~\ref{algo:active_set}, the current best guess
%   $\widehat\bfgamma_{\supp}$ has to be completed on the complement $\supp^c$.
%   We propose to do so by minimizing the magnitude of the gradient with respect to $\bfbeta$ at the current solution:
%   \begin{align*}
%     \widehat\bfgamma_{\supp^c} & = \argmin_{\bfgamma_{\supp^c}:\bfgamma \in \clD_{\bfgamma}}
%                                \left|  \bfX_{\centerdot\supp^c}^\intercal(\bfX_{\centerdot\supp}\bfbeta_{\supp}-\bfy) - 
%                                \lambda \bfgamma_{\supp^c} \right|
%       \enspace.
%   \end{align*}
%   The resulting vector $\widehat\bfgamma$ is then used to compute the lower
%   bound on the optimal objective function as $\min_{\bfbeta\in\mathbb{R}^{p}} J(\bfbeta,\widehat\bfgamma)$. 
%   This lower bound, which requires to solve a linear system of size $p$, is
%   much lighter to compute than the duality gap \citep[see details
%   in][]{2012_FML_Bach}, and was observed to provide a tight bound (as a sanity
%   check, note that when $\bfbeta$ is optimal, our estimated optimality gap is null).
% \fi


\section{Numerical Experiments}\label{sec:experiments}

This section compares the performances of our algorithm to its state-of-the-art
competitors from an optimization viewpoint.  Efficiency may then be assessed by
accuracy and speed:  accuracy is the difference between the optimum of the
objective function and its value at the solution returned by the algorithm;
speed is the computing time required for returning this solution.
Obviously, the timing of two algorithms/packages has to be compared at similar
precision requirements, which are rather crude in statistical learning, far from
machine precision \citep{Bottou08}.

 We provide an in-depth analysis on simulated data, thereby covering
a wide range of well-controlled situations, where our conclusions are
derived from numerous simulations.  Finally, the available ground
truths will also be useful to show that optimization issues during
training have notable effects in terms of prediction accuracy and
support recovery.  These optimization issues are eventually illustrated
on  a real dataset in genomics.
%, where it is shown  that different supports
%are recovered by \texttt{glmnet} and our proposed approach
%optimization is an actual issue from the data analysis viewpoint.
%%%%%%%%%%%%%%


\subsection{Synthetic data}

We compare the performance of the proposed quadratic solver with
representatives of the most successful competing algorithms.
First, we use our own implementations of all competitors, so as to provide
comparisons without implementation biases (language, library, etc.).
We use \mytexttt{R} with most of the matrix calculus done in \mytexttt{C++} using
the \mytexttt{RcppArmadillo} package \citep{2011_JSS_rcpp,armadillo} that relies
on \mytexttt{BLAS/LAPACK} libraries for linear algebra operations.
Second, we compare our code to the leading standalone packages that are
available today, so as to provide comparisons avoiding a possible competence
bias.

We use simulated data to obtain representative average results. 
Their generation covers the typical attributes of the real data encountered in
post genomic and signal processing.
In these domains, the main optimization difficulties result from
ill-conditioning, which is either due to the high correlation between
predictors, or to underdetermination when the number of variables
exceeds the sample 
\iflong 
  size (also known as the high-dimensional or the ``large $p$ small $n$'' 
  setup).
\else
  size.
\fi
For the optimization algorithms based on active set strategies, bad
conditioning is somehow alleviated when the objective function has a regular
behavior when restricted to the subspace containing the solution.  All other
things being equal, this local conditioning is thus governed by the
sparsity of the unknown true parameter (which affects the sparsity of
the solution), which also heavily impacts the running times of most
optimization algorithms available today.

\subsubsection{Data Generation}

The above-mentioned characteristics are explored without difficulty in the
framework of linear regression. We generate samples of size $n$ from the model
\begin{equation*}
  \mathbf{y} = \mathbf{X} \bfbeta^\star + \varepsilon, \qquad
  \varepsilon \sim\mathcal{N}(\mathbf{0},\sigma^2\mathbf{I})
  \enspace,
\end{equation*}
with $\sigma$ chosen so as to reach a rather strong  coefficient of
determination ($R^2\approx 0.8$).  The design matrix $\mathbf{X}$ is drawn from
a multivariate normal distribution in $\Rset^p$, and the conditioning of
$\mathbf{X}^\intercal\mathbf{X}$ is ruled by the correlation between variables. 
We use the same correlation coefficient $\rho$ for all pairs of variables.
The sparsity of the true regression coefficients is controlled by a parameter
$s$, with
\begin{equation*}
  \bfbeta^\star = \big(\underbrace{2,\dots,2}_{s/2} , \underbrace{-2,\dots,-2}_{s/2} , \underbrace{0,\dots,0}_{p-s} \big)
  \enspace.
\end{equation*}
% where $s$ controls the level of 
% with either 10\%, 30\%
%or 60\% of relevant parameters. 
Finally, the ratio $p/n$
% \in \{2, 1, 0.5\}$ finally
quantifies the well/ill-posedness of the problem.
%These values are not as extreme as the ones typically encountered in genomic
%applications, but they enable to explore a wider range of hyper-parameter values.


% % with the \mytexttt{glmnet} package
% %\citep{2009_JSS_Friedman} which is a reference tool in genomic studies and more
% %generally in the statistics community.
 
\subsubsection{Comparing Optimization Strategies} 

We compare here the performance of three state-of-the-art optimization
strategies implemented in our own computational framework:
accelerated proximal method \citep[see, e.g.,][]{2009_SIAM_Beck},
coordinate descent \citep[popularized by][]{2007_AAS_Friedman},
and our algorithm, that will respectively be named hereafter \mytexttt{proximal},
\mytexttt{coordinate} and \mytexttt{quadratic}.
Our implementations estimate the solution to the elastic-net problem
\begin{equation}
  \label{eq:enet}
%   \hatbbetaenet_{\lambda_1, \lambda_2} = \argmin_{\bfbeta \in
%     \mathbf{R}^p}J_{\lambda_1, \lambda_2}^{\text{enet}}(\bfbeta)
%     \enspace,
%     \ \text{with} \enspace
    J_{\lambda_1,
    \lambda_2}^{\text{enet}}(\boldsymbol\beta)= \frac{1}{2}
  \norm{\mathbf{X}\boldsymbol\beta - \mathbf{y}}^2 +
  \lambda_1 \norm[1]{\boldsymbol\beta} + \frac{\lambda_2}{2} \norm{\boldsymbol\beta}^2
  \enspace,
\end{equation}
which is strictly convex when $\lambda_2>0$ and thus admits a unique
solution even if $n<p$.

The three implementations are embedded in the same active set routine, which
approximately solves the optimization problem with respect to a limited number
of variables as in Algorithm~\ref{algo:active_set}.
They only differ regarding the inner optimization problem with respect to the
current active variables, which is performed by an
accelerated proximal gradient method for \mytexttt{proximal}, by coordinate descent
for  \mytexttt{coordinate}, and by the resolution of the worst-case quadratic
problem for \mytexttt{quadratic}.
We followed the practical recommendations of \citet{2012_FML_Bach} for
accelerating the proximal and coordinate descent implementations, and we used
the same halting condition for the three implementations, based on the
approximate satisfaction of the first-order optimality conditions:
\begin{equation}
  \label{eq:tau_conv}
%   \max_{j\{\in 1\dots p\}} \left| \mathbf{x}_j^\intercal\left(\mathbf{y}
%       -\mathbf{X}\hatbbetaenet_{\lambda_1,\lambda_2}                 
%     \right) + \lambda_2\hatbbetaenet_{\lambda_1,\lambda_2} \right| <
%   \lambda_1 + \tau,
  \max_{j\{\in 1\dots p\}} \left| \mathbf{x}_j^\intercal\left(\mathbf{y}
      -\mathbf{X}\hatbbeta                 
    \right) + \lambda_2\hatbbeta \right| <
  \lambda_1 + \tau,
\end{equation}
where the threshold $\tau$ was fixed to $\tau=10^{-2}$ in our simulations.~\footnote{%
  The rather loose threshold is favorable to \mytexttt{coordinate} and
  \mytexttt{proximal}, which reach the threshold, while \mytexttt{quadratic}
  ends up with a much smaller value, due to the exact resolution, up to
  machine precision, of the inner quadratic problem.
}
Finally, the active set algorithm is itself wrapped in a warm-start routine,
where the approximate solution to $J^{\text{enet}}_{\lambda_1,\lambda_2}$ is
used as the starting point for the resolution of
$J^{\text{enet}}_{\lambda_1',\lambda_2}$ for $\lambda_1' < \lambda_1$.

Our benchmark considers small-scale problems, with size $p=100$, and the nine
situations stemming from the choice of three following parameters:
\iflong
\begin{itemize}
\item low,  medium  and  high correlation between predictors ($\rho \in \{0.1, 0.4, 0.8\}$),
\item low, medium and  high-dimensional setting ($p/n \in \{2, 1,
  0.5\}$),
\item low, medium and high levels of sparsity ($s/p\in\{0.6 ,0.3,0.1\}$).
\end{itemize}
\else
1) low,  medium  and  high levels of correlation between predictors ($\rho \in \{0.1, 0.4, 0.8\}$),
2) low, medium and  high-dimensional setting ($p/n \in \{2, 1, 0.5\}$),
3) low, medium and high levels of sparsity ($s/p\in\{10\% ,30\%,60\%\}$).
\fi
Each solver computes the elastic-net for the tuning parameters $\lambda_1$ and
$\lambda_2$ on a 2D-grid of $50 \times 50$ values, and their running 
times have been averaged over 100 runs.

All results are qualitatively similar regarding the dimension and sparsity
settings.
Figure~\ref{fig:timing_all} displays the high-dimensional case ($p=2n$) with
a medium level of sparsity ($s=30$) for the three levels of correlation.
\iflong
  \begin{figure}%[htbp!]
    \centering
    \setlength{\unitlength}{0.575\linewidth}%
    \begin{picture}(1.5,2.2)%
      \put(0,0){\includegraphics[angle=90,width=1.45\unitlength]{../figures/timing_all}}
      \put(1.425,1.575){\rotatebox{90.0}{\makebox[0cm]{$\log_{10}{\lambda_1}$}}}
      \put(1.425,1.0){\rotatebox{90.0}{\makebox[0cm]{$\log_{10}{\lambda_1}$}}}
      \put(1.425,0.425){\rotatebox{90.0}{\makebox[0cm]{$\log_{10}{\lambda_1}$}}}
      \put(0.4,0.025){\makebox[0cm]{$\log_{10}{\lambda_2}$}}
      \put(1.025,0.025){\makebox[0cm]{$\log_{10}{\lambda_2}$}}
%       \put(0,2.2){\line(1,0){1.5}}
    \end{picture} 
     \caption{Surfaces of the log-ratio of computation times (brighter shades
     indicate higher ratio, see colorbar) according to the
     $(\lambda_1,\lambda_2)$ penalty parameters for \mytexttt{coordinate} versus
     \mytexttt{quadratic} (left), and \mytexttt{proximal} versus
     \mytexttt{quadratic} (right), for $(p,n)=(100,50),\, s=30$ and
     correlation $\rho \in \{0.1, 0.4, 0.8\}$ (top, middle and bottom
     respectively).}
    \label{fig:timing_all}
  \end{figure}
\else
  \begin{figure}
    \centering
    \setlength{\unitlength}{0.5\linewidth}%
    \begin{picture}(2,1.2)%
      \put(0.025,0.025){\includegraphics[angle=0,width=0.9\textwidth]{../figures/timing_all}}
      \put(0,0.4){\rotatebox{90.0}{\makebox[0cm]{$\log_{10}{\lambda_2}$}}}
      \put(0,0.9){\rotatebox{90.0}{\makebox[0cm]{$\log_{10}{\lambda_2}$}}}
      \put(0.45,0){\makebox[0cm]{$\log_{10}{\lambda_1}$}}
      \put(1,0){\makebox[0cm]{$\log_{10}{\lambda_1}$}}
      \put(1.45,0){\makebox[0cm]{$\log_{10}{\lambda_1}$}}
    \end{picture} 
     \caption{Log-ratio of computation times for \mytexttt{coordinate} (top) and
     \mytexttt{proximal} (bottom) versus \mytexttt{quadratic}, for high, medium
     and low variable correlation (left, center and right respectively).}
    \label{fig:timing_all}
  \end{figure}
\fi
Each map represents the log-ratio between the timing of either
\mytexttt{coordinate} or \mytexttt{proximal} versus \mytexttt{quadratic},
according to $(\lambda_1, \lambda_2)$ for a given correlation level.  
Dark regions with a value of 1 indicate identical running times while lighter
regions with a value of 10 indicate that \mytexttt{quadratic} is 10 times faster.
Figure~\ref{fig:timing_all} illustrates that \mytexttt{quadratic} outperforms both
\mytexttt{coordinate} and \mytexttt{proximal}, by running much faster in most
cases, even reaching 300-fold speed increases.  
The largest gains are observed for small $(\lambda_1,\lambda_2)$ penalty
parameters for which the problem is ill-conditioned, including many active
variables, resulting in a huge
slowdown of the first-order methods \mytexttt{coordinate} and
\mytexttt{proximal}.
As the penalty parameters increase, smaller gains are observed, especially when
$\lambda_2$, attached to the quadratic penalty, reaches high values for which all
problems are well-conditioned, and where the elastic-net is leaning towards
univariate soft thresholding, in which case all algorithms behave similarly.

\subsubsection{Comparing Stand-Alone Implementations}

We now proceed to the evaluation of our code with three other
stand-alone programs for solving the Lasso, which are publicly available as \mytexttt{R} packages.  We
chose three leading state-of-the-art packages, namely
\mytexttt{glmnet} \citep[Generalized Linear Models regularized by
Lasso and elastic-net,][]{2009_JSS_Friedman}, \mytexttt{lars}
\citep[Least Angle Regression, Lasso and Forward
Stagewise,][]{2004_AS_Efron} and \mytexttt{SPAMS} \citep[SPArse
Modeling Software,][]{2012_FML_Bach}, with two options
\mytexttt{SPAMS-FISTA}, which implements an accelerated proximal
method, and \mytexttt{SPAMS-LARS} which is a \mytexttt{lars}
substitute.  Note that \mytexttt{glmnet} does most of its internal
computations in \mytexttt{Fortran}, \mytexttt{lars} in \mytexttt{R},
and \mytexttt{SPAMS} in \mytexttt{C++}.  Our own implementation, by
resolution of the worst-case quadratic problem, is shipped within an
\mytexttt{R} package \mytexttt{quadrupen} publicly available on github
{\url{https://github.com/jchiquet/quadrupen}}.

We benchmark these packages by computing  regularization paths for the
Lasso\footnote{%
  We benchmark the packages on a Lasso problem since the parametrization of the
  elastic-net problem differs among packages, hindering fair comparisons.},
that is, the elastic-net Problem~\eqref{eq:enet} with $\lambda_2=0$.   
%
The inaccuracy of the solutions produced is measured by the gap in the objective
function compared to a reference solution, considered as being the true optimum.  
\iflong
  We use the \mytexttt{lars} solution as a reference, since it solves the Lasso problem
  up to the machine precision, relying on the \mytexttt{LAPACK/BLAS} routines. 
  % accessed via the \mytexttt{R} framework.
  Furthermore, \mytexttt{lars} provides the solution path for the Lasso,
  that is, the set of solutions computed for each penalty parameter value
  for which variable activation or deletion occurs, from the empty model to the
  least-mean squares model.
  This set of reference penalty parameters is used here to define a sensible reproducible choice.

  In high dimensional setups, the computational cost of returning the solutions
  for the largest models may be overwhelming compared to the one necessary
  for exploring the interesting part of the regularization path
  \citep{2011_JSS_Simon,2009_JSS_Friedman}.  This is mostly due to numerical
  instability problems that may be encountered in these extreme settings, where
  the Lasso solution is overfitting as it approaches the set of solutions to the
  underdetermined least squares problem.  We avoid a
  comparison mostly relying on these spurious cases by restricting the set
  of reference penalty parameters to the first $\min(n,p)$ steps of
  \mytexttt{lars} \citep[similar settings are used by][]{2009_JSS_Friedman}.

  Henceforth, the  distance $\mathrm{D}$  of a given  \mytexttt{method} to
  the optimum is evaluated on  the whole set of penalties $\Lambda$ used
  along the path, by
  \begin{equation*}
    \mathrm{D}(\mytexttt{method}) = \left( \frac{1}{|\Lambda|}\sum_{\lambda\in\Lambda}
      \left(J_{\lambda}^{\text{lasso}}\left(\hatbbeta_\lambda^{\mytexttt{lars}}\right)
        -J_{\lambda}^{\text{lasso}}\left(\hatbbeta_\lambda^{\mytexttt{method}}\right)\right)^2
       \right)^{1/2} 
    \enspace,
  \end{equation*}
  where $J_{\lambda}^{\text{lasso}}(\bfbeta) = J_{\lambda,0}^{\text{enet}}(\bfbeta)$ 
  is the objective function of the Lasso evaluated at $\bfbeta$, and
  $\hatbbeta_\lambda^{\mytexttt{method}}$ is the estimated optimal solution
  provided by the \mytexttt{method} package currently tested.
\fi

The data  sets are  generated according to  the linear  model described
above, in three different high-dimensional settings and small to medium number 
of variables: 
$(p,n)=(100,40)$, $(p,n)=(1\,000,200)$ and $(p,n)=(10\,000,400)$.  
The sparsity of the true underlying $\boldsymbol\beta^\star$ is governed by $s =
0.25\min(n,p)$, and the correlation between predictors is set by $\rho\in\{0.1, 0.4,  0.8\}$. For  each  value of
$\rho$, we averaged the timings over $50$ simulations, ensuring that each package
computes the solutions at identical $\lambda$ values, as defined above.

\iflong

  We pool together the runtimes obtained for the three levels of correlation for
  \mytexttt{quadrupen}, \mytexttt{SPAMS-LARS} and \mytexttt{lars}: all these 
  second-order methods are not sensitive to the correlation between features
  In each plot of Figure~\ref{fig:timing_glmnet}, each of these methods is thus 
  represented by a single point marking the average precision and the average distance to the
  optimum over the 150 runs (50 runs for each $\rho \in \{0.1, 0.4, 0.8\}$).
  Note that for \mytexttt{lars} only the abscissa is meaningful since
  $\mathrm{D}(\mytexttt{lars})$ is zero by definition.
  Besides, \mytexttt{quadrupen}, which solves each quadratic problem up to the
  machine precision, tends to be within this precision of the \mytexttt{lars}
  solution.%, as the required precision $\tau$ in \eqref{eq:tau_conv} was set here to $10^{-7}$.
  The \mytexttt{SPAMS-LARS} is also very precise, up to $10^{-6}$,
  which is the typical precision of the approximate resolution of linear systems.
  It is the fastest alternative for solving the Lasso when the problem is
  high-dimensional with a large number of variables (Figure
  \ref{fig:timing_glmnet}, bottom-left).

  \begin{figure}
    \centering
    \begin{tabular}{cc}
      \xylabelsquare{../figures/timing_others_low}{CPU time (in seconds, $\log_{10}$)}
                    {$\mathrm{D}(\mytexttt{method})$ ($\log_{10}$)}
                    {$n=100$, $p=40$}% 
      & \xylabelsquare{../figures/timing_others_med}{CPU time (in seconds, $\log_{10}$)}
                    {$\mathrm{D}(\mytexttt{method})$ ($\log_{10}$)}
                    {$n=200$, $p=1\,000$}% 
      \\[4ex] % 
      \xylabelsquare{../figures/timing_others_hig}{CPU time (in seconds, $\log_{10}$)}
                    {$\mathrm{D}(\mytexttt{method})$ ($\log_{10}$)}
                    {$n=400$, $p=10\,000$}% 
      & \xylabelsquare{../figures/timing_others_legend}{}{}{} \\
    \end{tabular}
    \caption{Distance  to optimum  versus CPU  time for  three different
     high-dimensional settings: $(p,n)=(100,40)$ (top left), $(p,n)=(1\,000,200)$
     (top right) and $p=(10\,000,400)$ (bottom left).  }
    \label{fig:timing_glmnet}
  \end{figure}

  In contrast, the (precision,timing)-values of \mytexttt{glmnet} and
  \mytexttt{SPAMS-FISTA} are highly affected by the threshold
  parameters\footnote{%
    In \mytexttt{glmnet}, convergence is monitored by the stability of the
    objective function, measured between two optimization steps, and
    optimization is halted when changes fall below the specified threshold
    (scaled by the null deviance).
    In \mytexttt{SPAMS-FISTA}, the stopping condition of the algorithm is based
    on the relative change of parameters between two iterations.}
  that control their stopping conditions.
  The computational burden to reach a given precision is also affected by the
  level of correlation, as illustrated in Figure~\ref{fig:timing_glmnet}.
  Obviously, a precise solution is difficult to reach with first-order descent
  algorithms in a high correlation setup, which corresponds to an
  ill-conditioned linear system.
  It may be surprising to observe that \mytexttt{SPAMS-FISTA} is about ten time
  slower than \mytexttt{glmnet}, as proximal and coordinate descent
  methods were experimentally shown to be roughly equivalent in our preceding
  analysis and by \citet{2012_FML_Bach}.
  However, these two comparisons were carried out with the same active set 
  strategy \citep[that is, \emph{with} active set for ours and \emph{without} 
  active set for][]{2012_FML_Bach}.
  We believe that this difference in the handling of active variables explains
  the relative bad performance of \mytexttt{SPAMS-FISTA}, which optimizes all
  variables along the regularization path, while  \mytexttt{glmnet} uses a 
  greedy active set strategy. 
%   Indeed, we observed a better match
%   between the performances of coordinate descent and accelerated proximal method
%   in our own implementations relying on the same active set strategy for all
%   approaches.

 Overall, our implementation is highly competitive, that is, very accurate, at
 the \mytexttt{lars} level, and much faster.  The speed improvements of
 \mytexttt{glmnet} are only observed for very rough approximate solutions and
 \mytexttt{SPAMS-FISTA} is dominated by \mytexttt{glmnet}. 
 Our experiments, in the framework of active set methods, agree with the results of
 \citet{2012_FML_Bach}: indeed, they observed that first-order methods are
 competitive with second-order ones only for low correlation levels and
 small penalties (which entails a large number of active variables).
 Conversely, our results may appear to contradict some of the experimental
 findings of \citet{2009_JSS_Friedman}: first, we observe that \mytexttt{glmnet}
 is quite sensitive to correlations, and second, the optimized second-order
 methods are competitive with \mytexttt{glmnet}.
 These differences in conclusions arise from the differences in experimental
 protocols: while we compare running times at a given accuracy, they are compared
 at a given threshold on the stopping criterion by \citet{2009_JSS_Friedman}.
 Regarding the influence of correlations, the stability-based criterion can be
 fooled due to the tiny step size that typically occurs for ill-conditioned
 problems, leading to a sizable early stopping.
 Regarding the second point, even though the \mytexttt{R} implementation of
 \mytexttt{lars} may indeed be slow compared to \mytexttt{glmnet}, considerable
 improvements can be obtained using optimized second-order methods such as 
 \mytexttt{quadrupen} as soon as a sensible accuracy is required, especially
 when correlation increases.
 
 Finally, among the accurate solvers, \mytexttt{SPAMS-LARS} is insignificantly
 less accurate than \mytexttt{quadrupen} or \mytexttt{lars} in a statistical
 context.  It is always faster than \mytexttt{lars} and slightly faster than
 \mytexttt{quadrupen} for the largest problem sizes (Figure
 \ref{fig:timing_glmnet}, bottom-left) and much slower for the smallest problem
 (Figure \ref{fig:timing_glmnet}, top-left).

%
\else
 The results, displayed in Figure  \ref{fig:timing_glmnet}, show that our
 implementation is highly competitive, that is, very accurate, at the \mytexttt{lars} 
 level, and much faster.  The speed improvements of \mytexttt{glmnet} 
 are only observed for very rough approximate solutions and
 \mytexttt{SPAMS} FISTA is dominated by \mytexttt{glmnet}.
  Finally, \mytexttt{SPAMS} homotopy is slightly less accurate than
  \mytexttt{quadrupen}  and \mytexttt{LARS} but  it is the fastest accurate
  alternative for the largest problem sizes (Figure  \ref{fig:timing_glmnet}, right). 

  \begin{figure}
    \centering 
    \begin{tabular}{@{}c@{}c@{}c@{}c@{}}
      \xylabelsquare{../figures/timing_others_low}{CPU     time
        ($\log_{10}$)}{optimization gap ($\log_{10}$)}{}% 
      & \xylabelsquare{../figures/timing_others_med}{CPU time
        ($\log_{10}$)}{}{} % 
      \xylabelsquare{../figures/timing_others_hig}{CPU     time
        ($\log_{10}$)}{}{}%
      & \xylabelsquare{../figures/timing_others_legend}{}{}{} \\
    \end{tabular}
    \caption{Distance  to optimum  versus CPU  time for  three different
     high-dimensional settings: $p=100,\ n=40$ (left), $p=1\,000,\
     n=200$ (center) and
     $p=10\,000,\ n=400$ (right). }
    \label{fig:timing_glmnet}
  \end{figure}
\fi

\subsubsection{Link between accuracy of solutions and prediction performances}
\label{sec:fromatop}

When the ``irrepresentable condition'' \citep{2006_JMLR_Zhao} holds, the
Lasso should  select the true model consistently.   However, even when
this  rather  restrictive  condition  is  fulfilled,  perfect  support
recovery  obviously requires numerical  accuracy:
rough estimates may speed up the procedure, but whatever optimization strategy
is used, stopping an algorithm is likely to prevent either the removal of all
irrelevant coefficients or the insertion of all relevant ones.  The support of 
the solution may then be far from the optimal one. 

We  advocate here  that our  quadratic solver  is very  competitive in
computation time when support recovery matters, that is, when high level  of accuracy is needed, in small (few
hundreds of variables) and  medium sized problems (few thousands).  As
an  illustration,  we  
% highlight  a  simple  situation  where  such  a
% desideratum arises:  we 
generate 100 data sets under the linear model described above, with a rather
strong coefficient of determination ($R^2
\approx 0.8$ on  average), a rather high level  of correlation between
predictors ($\rho=0.8$) and a medium level of sparsity ($s/p = 30\%$).
The number of  variable is kept low ($p=100$) and  the difficulty of the
estimation problem  is tuned  by the  $n/p$  ratio. For  each data  set, we  also
generate a  test set sufficiently  large (say, $10n$) to  evaluate the
quality of the prediction without  depending on any sampling fluctuation.  
We compare the Lasso solutions computed by \mytexttt{quadrupen} to the ones returned by
\mytexttt{glmnet} with
various   level   of    accuracy\footnote{This   is   done   via   the
  \mytexttt{thresh}  argument of  the  \mytexttt{glmnet} procedure,  whose
  default  value  is \mytexttt{1e-7}.   In  our  experiments,
  \mytexttt{low}, \mytexttt{med}  and \mytexttt{high}  level of  accuracy for
  \mytexttt{glmnet} respectively correspond to \mytexttt{thresh} set to
  \mytexttt{1e-1}, \mytexttt{1e-4}, and \mytexttt{1e-9}.}.  

% \begin{figure}[htbp]
%   \centering
%   \begin{tabular}{@{}l@{}c@{}} 
%     \rotatebox{90.0}{\makebox[.6\textwidth]{\hspace{.3\textwidth} Support Error 
%     Rate \hspace{.35\textwidth} MSE}}
%     &
%     \includegraphics[width=.95\textwidth]{../figures/accuracy}
%     % \includegraphics[trim={0.5cm 0.5cm 0.5cm 0.5cm},clip,angle=0,width=\textwidth]{../figures/accuracy}
%   \end{tabular}
%   \caption{Test performances according to the penalty parameter for the
%     Lasso estimates returned by \mytexttt{quadrupen} and \mytexttt{glmnet} at various level
%     of accuracy. Three high-dimensional setups are illustrated: from left to 
%     right $n/p=1/2$, $n/p=1$ and $n/p=2$;
%     top: mean squared test error;
%     bottom: support error rate.\label{fig:accuracy}}
% \end{figure} 

\begin{figure}[htbp]
  \centering
  \begin{tabular}{@{}l@{}c@{}}
    & $n = p/2$ \hspace{2.2cm} $n = p$ \hspace{2.2cm} $n = 2p$ \\
    \rotatebox{90.0}{\makebox[.2\textwidth]{\hspace{7ex} RMSE }}
    & \includegraphics[width=.9\textwidth]{../figures/rmse} \\[-4ex]
    & ~\hspace{1.1cm} $\log_{10}(\lambda)$ \hspace{1.9cm} $\log_{10}(\lambda)$ 
    \hspace{1.9cm} $\log_{10}(\lambda)$ \hspace{1.1cm}~ \\
    \rotatebox{90.0}{\makebox[.2\textwidth]{\hspace{2ex}Precision}}
    & \includegraphics[width=.9\textwidth]{../figures/precision} \\[-4ex]
    & ~\hspace{1.1cm} $\log_{10}(\lambda)$ \hspace{1.9cm} $\log_{10}(\lambda)$ 
    \hspace{1.9cm} $\log_{10}(\lambda)$ \hspace{1.1cm}~ \\
    \rotatebox{90.0}{\makebox[.2\textwidth]{\hspace{5ex}Recall}}
    & \includegraphics[width=.9\textwidth]{../figures/recall} \\[-4ex]
    & ~\hspace{1.1cm} $\log_{10}(\lambda)$ \hspace{1.9cm} $\log_{10}(\lambda)$ 
    \hspace{1.9cm} $\log_{10}(\lambda)$ \hspace{1.1cm}~ \\
    \rotatebox{90.0}{\makebox[.2\textwidth]{\hspace{7ex}Precision}}
    & \includegraphics[width=.9\textwidth]{../figures/PR_curves} \\[-4ex]
    & ~\hspace{1.1cm} Recall \hspace{2.3cm} Recall \hspace{2.3cm} Recall \hspace{1.1cm}~ \\
  \end{tabular}
  \caption{Test RMSE, precision and recall for the
    Lasso estimates returned by \mytexttt{quadrupen} and \mytexttt{glmnet} at 
    the low and medium levels of accuracy. Three high-dimensional setups are illustrated: from left to 
    right $n/p=1/2$, $n/p=1$ and $n/p=2$.\label{fig:accuracy}}
\end{figure} 

Figure \ref{fig:accuracy} reports performances, as measured by the root mean 
squared error (RMSE) and the precision and the recall regarding the selected 
coefficients.  
In the top row, one sees that one can reach about the same prediction 
performance for all accuracies. However, these best performances are obtained 
for different penalty strengths, with more stringent penalties (that is, higher values 
of $\lambda$) for the more accurate estimates. 
The central rows show that, for any penalty strength, being more accurate 
brings better precision and identical or worse recall.
As expected, the curves show that there is a trade-off in selecting truly 
relevant variables and discarding the irrelevant ones. 
Combined with the top row, the precision and recall vs $\lambda$ curves show that there are noticeable differences in terms of support for the best predictors.
Finally, in the bottom row, the precision vs recall curves show that the 
accurate solutions dominate the inaccurate ones. 
Here, a solution dominates another one if it is displayed more to the right and higher. 
It can be seen that any point of the green curve 
(\mytexttt{glmnet} with \mytexttt{low} accuracy) is dominated by a point of the blue curve (\mytexttt{glmnet} with \mytexttt{med} accuracy), 
which is itself dominated by a point on the red curve (\mytexttt{quadrupen}).
Hence, there is always a solution of \mytexttt{quadrupen} having both better precision and better recall that the solutions returned by \mytexttt{glmnet}  \mytexttt{low} or \mytexttt{glmnet} \mytexttt{med}.
%
Note that this dominating solution needs not to be obtained for the same penalty parameter.
% Regarding  the MSE though (upper
% part of  the Figure), an interesting  point concerns the  shape of the
% error  curve: a  high level  of accuracy  tends to  produce  less flat
% curves with clearer  minimums, which may have an  important impact for
% model selection.   This question is of particular  interest for penalized
% procedures such as the Lasso, where the choice of the tuning parameter
% remains a bottleneck commonly dealt with cross-validation.

\begin{table}[htbp]
  \caption{Median timings and solution accuracies% for the experiments
                                % of Section~\ref{sec:fromatop}.
  }
  \label{tab:accuracy}
  \begin{tabular}{@{}l|cccc@{}}
    methods & \mytexttt{quadrupen} & \mytexttt{glmnet\,low} & \mytexttt{glmnet\,med} & \mytexttt{glmnet\,high} \\
    \hline
    timing (ms) & 8 & 7 & 8 & 64 \\
    accuracy (dist.  to opt.)  & $5.9\times 10^{-14}$ & $7.2 \times 10^{0}$ & $6.04 \times 10^{0}$ & $1.47 \times 10^{-2}$\\
  \end{tabular}
\end{table} 

%Now focusing on \mytexttt{glmnet} prediction performances, the better the accuracy, the  smaller the  MSE  and the  support  error rate.  But  the better  the accuracy,  the  slower  the  algorithm  becomes.   Using  the  default settings allows to have a result very close to our quadratic solver, and 
\sloppypar
Figure \ref{fig:accuracy} does not display the performances of \mytexttt{glmnet} with a \mytexttt{high} level of  accuracy, because they are indistinguishable from those of\mytexttt{quadrupen}. Table~\ref{tab:accuracy} shows that this accuracy is achieved at 
a high computational cost: to be at par with \mytexttt{quadrupen} with regards to performances, \mytexttt{glmnet} is about ten times slower than our solver.

\subsection{Quantitative Trait Loci and Association Mapping}

Many quantitative traits in plants and animals are heritable.  When
considering Mendelian traits, the Quantitative Trait Loci (QTL) are sections of
DNA that explain the trait variability.
As traits are typically controlled by several QTL, the association 
between traits and QTL involves several limited effects that are difficult to detect individually.

A possible method for mapping genotype to trait consists in regressing
the trait (that is, phenotype) of interest  against the QTL (that is, genotype), using penalized regression.
%
In this introductory example, we consider the maize association
mapping panel described in \citep{RincentEtAl2014},where 269
individuals were genotyped with a 50k single-nucleotide polymorphisms (SNPs) array. 
After classical data
cleaning, 261 individuals with 29849 markers (that is, SNPs) were kept.
%

We focus here on the tasseling time of maize, whose heritability is
explained by many genes.  We used two different implementations of
Lasso for detecting the relevant markers: \mytexttt{glmnet} and
\mytexttt{quadrupen}. The two implementations were run using their
default parameters, with the same penalty strength $\lambda$, selected
by leave-one-out cross-validation with \mytexttt{glmnet}.  The latter
selected 169 markers among the 29849 SNPs, whereas
\mytexttt{quadrupen} was more stringent, selecting only a subset of
156 markers.
% Where does this $10\%$ difference in the number of SNPs comes from since there is theoretically one unique solution to our convex Lasso problem ? 
This notable difference is unexpected considering that the two implementations attempt to solve the very same convex optimization problem.

In the following, we show how the speed and precision of \mytexttt{glmnet} are affected by the threshold controlling the stopping condition. 
Regarding precision, a first hint is provided in 
Figure \ref{fig:gradient}, which displays the absolute value of the derivatives of the objective function with respect to the non-zero coefficients of the parameter vectors returned by \mytexttt{glmnet} and \mytexttt{quadrupen}, respectively. 
The departure from zero for \mytexttt{glmnet} illustrates its relative imprecision, which may be  responsible for the difference in the set of selected markers.
%
\begin{figure}
  \centering
  \begin{tabular}{l@{}c}
  \rotatebox{90}{\small \hspace{2.25cm} count}
    &     \includegraphics[width=0.6\textwidth,trim=5mm 5mm 50.5mm 0mm,clip=true]{../figures/gradient} \\[-1.5ex]
    & \small{$\left| \mathbf{x}_j^\intercal\left(\mathbf{y}  -\mathbf{X}\hatbbeta\right) \right| -  \lambda$} \hspace{0cm} \\
  \end{tabular}
  \caption{Histogram  (log scale) of the derivatives w.r.t. the non-zero coefficients of the parameter vectors returned by \mytexttt{glmnet} and \mytexttt{quadrupen}}.
  \label{fig:gradient}
\end{figure}

This last conjecture fits perfectly with the simulations of the previous section (see Figure \ref{fig:accuracy}),
where the difference in support recovery between \texttt{quadrupen} and \texttt{glmnet}
is   noticeable with the \texttt{glmnet} default value of precision. 


\section{Discussion}

%% SUMMARY

% PRINCIPLE
This paper presents a new viewpoint on sparsity-inducing penalties
where the dual norms associated with these penalties play a central role. 
Technically, our formulation is a simple dual form of the original problem. 
However, we do not follow a very general principle such as Fenchel duality: 
we specifically tailor our formulation to optimisation problems involving a 
sparsity-inducing norm.
The dual variables define a series of linear or quadratic penalties whose 
sublevel sets define the feasible set of the original problem through
intersection.

This  viewpoint  enables  to  encompass in the same framework  several
well-known penalties. In particular, we detailed how the solutions to the Lasso and the
group-Lasso (with the $\ell_{\infty,1}$  mixed norm), possibly applied
together with an  $\ell_2$ ridge penalty (leading to what  is known as
the elastic net for the Lasso) can be derived. 

% ALGORITHM AND MAGIC BOUND
We derived a  general-purpose algorithm that computes the solution to
the penalized regression problem,  and proposed a new lower
bound  on the  minimum  of  the objective  function to
assess convergence.  The proposed  algorithm solves a series of
quadratic  problems on a working set defined  by the dual  variables.   
It has  been
thoroughly tested  and compared with  state-of-the-art implementations
for the elastic net and the Lasso, and prevails over its competitors for 
the problems tested, involving  up to a few thousands of variables.

% DISCUSSION
From a  practical viewpoint, an  important feature of our  approach is
that it solves the original problem  up to machine precision: we shown
that when  variable selection  is involved,  optimization with  a high
level of precision is mandatory to recover the true model.
% Moreover, our algorithm  is well suited for computing  a solution path
% rather than  for just one  value of $\lambda_1, (\lambda_2)$  since we
% are  generally interested  in cross-validating  these paths  for model
% selection purpose (at least in genomics).
\\

% FUTURE DEV
Regarding  future  development,  the   algorithm  can  be  adapted  to
non-quadratic loss  functions for  addressing other  learning problems
such  as  classification,  but  this  generalization,  which  requires
solving non-quadratic  problems, may not  be as efficient  compared to
the  existing alternatives.   We are  now examining  how to  address a
wider range of penalties by extending the framework in two directions:
first,  to accommodate  additional general  $\ell_2$ penalties  in the
form of  arbitrary symmetric  positive semidefinite matrix  instead of
the simple ridge, in particular to provide an efficient implementation
of the  structured elastic  net \citep{2010_AOS_Slawski} ;  second, we
plan to  derive similar  views on a  wider range  of sparsity-inducing
penalties, such as the fused-Lasso or the OSCAR \citep{Bondell08}.


\bibliographystyle{elsarticle-harv} 
\bibliography{biblio_crafter}


\appendix

\section{Proof of Proposition~\ref{prop:monitoring}}~\label{sec:proof:prop:monitoring}

We detail here a proof yielding a slightly tighter bound. 
Proposition~\ref{prop:monitoring} is simply a corollary of
Proposition~\ref{prop:monitoring:appendix:v2} stated and proved below.

The following Lemma relates the penalty
associated with a infeasible $\bfgamma$-value 
($\norm[*]{\bfgamma} > \eta$) to the one obtained by shrinking this
$\bfgamma$-value to reach the boundary of $\uball[*]^\eta$.

\begin{lemma}\label{prop:lemma1:v2}
  Let $\clS \subseteq \{1,...,p\}$, $\alpha\in(0,1)$, $\norm[*]{\cdot}$ be a vectorial norm and $\varphi^{*}(\cdot,\clS,\alpha) :
  \mathbb{R}^{p} \rightarrow \mathbb{R}^{p}$ be defined as follows:
  \begin{equation*} 
    \left\{
    \begin{array}{l}
      \varphi_{\clS}^{*}(\bfgamma,\clS,\alpha) = \bfgamma_{\clS} \\
      \varphi_{\clS^c}^{*}(\bfgamma,\clS,\alpha) = \alpha \bfgamma_{\clS^c} 
    \end{array}\right.\enspace.
   \end{equation*} 
%    For any $0\leq \alpha \leq 1$,
   Then,
   \begin{equation}\label{eq:lemma1:v2}
     \norm{\bfbeta-\varphi^{*}(\bfgamma,\clS,\alpha)}^2 \geq 
      \alpha \norm{\bfbeta-\bfgamma}^2  -
      \alpha (1-\alpha) \norm{\bfgamma_{\clS^c}}^{2} 
     \enspace. 
  \end{equation}
  \begin{proof}
    \begin{align*}
      \norm{\bfbeta-\varphi^{*}(\bfgamma,\clS,\alpha)}^2 & =
      \norm{\bfbeta_{\clS}-\bfgamma_{\clS}}^2 +
      \alpha \norm{\bfbeta_{\clS^c}-\bfgamma_{\clS^c}}^2 +
      (1-\alpha) \norm{\bfbeta_{\clS^c}}^{2} -
      \alpha (1-\alpha) \norm{\bfgamma_{\clS^c}}^{2} \\
      & \geq
      \norm{\bfbeta_{\clS}-\bfgamma_{\clS}}^2 +
      \alpha \norm{\bfbeta_{\clS^c}-\bfgamma_{\clS^c}}^2  -
      \alpha (1-\alpha) \norm{\bfgamma_{\clS^c}}^{2}  \\
      & \geq
      \alpha \norm{\bfbeta-\bfgamma}^2  -
      \alpha (1-\alpha) \norm{\bfgamma_{\clS^c}}^{2} 
      \enspace.
    \end{align*}
  \end{proof}
%   Note that the gap in the last inequality is 
%   $\norm{\bfbeta}^2({\norm[*]{\bfgamma}-\norm[*]{\bfgamma'}})/{\norm[*]{\bfgamma}}$,
%   so that the gap in Inequality \eqref{eq:lemma1:v2} is
%   $\norm{\bfbeta}^2({\norm[*]{\bfgamma}-\norm[*]{\bfgamma'}})/{\norm[*]{\bfgamma'}}$.
\end{lemma}
%
\begin{proposition}\label{prop:monitoring:appendix:v2}
  For any vectorial norm $\norm[*]{\cdot}$, 
%   when 
%   $\uball[*]^\eta$ is defined as $\uball[*]^\eta=\left\{\bfgamma \in \mathbb{R}^{p}:
%                       \norm[*]{\bfgamma} \leq \eta \right\}$, then, 
  $\forall \bfgamma \in \mathbb{R}^{p}:\norm[*]{\bfgamma} \geq \eta$, and 
  $\forall (\clS,\alpha)\in2^{\{1,\ldots,p\}}\times(0,1)$ such that 
  $\norm[*]{(\bfgamma_{\clS},\alpha\bfgamma_{\clS^c })}\leq \eta$, we   
  have:
  \begin{equation*}
    \min_{\bfbeta\in\mathbb{R}^{p}} \max_{\bfgamma' \in \uball[*]^\eta} 
    J_\lambda(\bfbeta,\bfgamma') 
    \geq
    \alpha J_\lambda\left(\bfbeta^\star\left(\bfgamma\right),\bfgamma\right) -
    \lambda \alpha (1-\alpha) \norm{\bfgamma_{\clS^c}}^{2}
    \enspace,
  \end{equation*}
  where 
  \begin{equation*}
    J_\lambda(\bfbeta,\bfgamma) = \norm{\bfX \bfbeta - \bfy}^2 + 
      \lambda \norm{\bfbeta - \bfgamma}^2
    \enspace \text{and} \enspace
    \bfbeta^\star(\bfgamma) = \argmin_{\bfbeta\in\mathbb{R}^{p}} J_\lambda(\bfbeta,\bfgamma)
    \enspace.
  \end{equation*}
  \begin{proof} 
  For all $\bfbeta \in \mathbb{R}^{p}$ and for any $(\bfgamma,\clS,\alpha) \in
  \mathbb{R}^{p}\times2^{\{1,\ldots,p\}}\times(0,1)$ such that
  $\varphi^{*}(\bfgamma,\clS,\alpha) \in \uball[*]^\eta$, with $\varphi^{*}$ 
  defined as in Lemma \ref{prop:lemma1:v2} we have
  \begin{equation}\label{eq:maxgreater:v2}
    \max_{\bfgamma' \in \uball[*]^\eta} 
    J_\lambda(\bfbeta,\bfgamma') 
    \geq
    J_\lambda\left(\bfbeta,\varphi^{*}(\bfgamma,\clS,\alpha)\right)
    \enspace,
  \end{equation}
  since $\varphi^{*}(\bfgamma,\clS,\alpha)$ belongs to $\uball[*]^\eta$. 
  We now compute a lower bound of the right-hand-side for $\bfgamma$ such that
  $\norm[*]{\bfgamma} \geq \eta$:
    \begin{align}       
      J_\lambda\left(\bfbeta,\varphi^{*}(\bfgamma,\clS,\alpha)\right) 
        & = \alpha \left( 
              \frac{1}{\alpha} \norm{\bfX \bfbeta - \bfy}^2 + 
              \frac{\lambda}{\alpha} 
              \norm{\bfbeta - \varphi^{*}(\bfgamma,\clS,\alpha)}^2
            \right)  
            \nonumber \\
        & \geq \alpha \left( 
              \norm{\bfX \bfbeta - \bfy}^2 + 
              \frac{\lambda}{\alpha} 
              \norm{\bfbeta - \varphi^{*}(\bfgamma,\clS,\alpha)}^2
            \right)  
            \label{eq:crude_inequality} \\
        & \geq \alpha \left( 
              \norm{\bfX \bfbeta - \bfy}^2 + 
              \lambda \norm{\bfbeta - \bfgamma}^2
           \right) -
           \lambda\alpha (1-\alpha) \norm{\bfgamma_{\clS^c}}^{2} 
           \enspace, \nonumber
    \end{align}
    where the last inequality stems from Lemma~\ref{prop:lemma1:v2}.
    This inequality holds for any given $\bfbeta$-value, in particular for
    $\bfbeta^\star(\varphi^{*}(\bfgamma,\clS,\alpha))=\argmin_{\bfbeta \in \mathbb{R}^{p}}
    J_\lambda\left(\bfbeta,\varphi^{*}(\bfgamma,\clS,\alpha)\right)$:
    \begin{align}       
      \min_{\bfbeta\in\mathbb{R}^{p}} 
      J_\lambda\left(\bfbeta,
                     \varphi^{*}(\bfgamma,\clS,\alpha)
               \right) 
      & \geq \alpha
          J_\lambda\left(\bfbeta^\star\left(\varphi^{*}(\bfgamma,\clS,\alpha)\right)\right) -
          \lambda\alpha (1-\alpha) \norm{\bfgamma_{\clS^c}}^{2} 
          \nonumber \\
      & \geq \alpha
          J_\lambda\left(\bfbeta^\star\left(\bfgamma\right),\bfgamma\right) -
          \lambda\alpha (1-\alpha) \norm{\bfgamma_{\clS^c}}^{2} 
          \label{eq:inequality:particular:appendix:v2} 
          \enspace,
    \end{align}
    where the second inequality follows from the definition of
    $\bfbeta^\star(\bfgamma)$.
    Inequality \eqref{eq:inequality:particular:appendix:v2} can be restated as:
     \begin{align*}       
       \min_{\bfbeta\in\mathbb{R}^{p}} 
       J_\lambda\left(\bfbeta,\varphi^{*}(\bfgamma,\clS,\alpha)\right)
       & \geq
         \alpha \min_{\bfbeta\in\mathbb{R}^{p}} 
         J_\lambda(\bfbeta,\bfgamma)  -
         \lambda\alpha (1-\alpha) \norm{\bfgamma_{\clS^c}}^{2} 
       \enspace.
    \end{align*}
    We finally remark that,  
    since $\varphi^{*}(\bfgamma,\clS,\alpha) \in \uball[*]^\eta$,
    we trivially have:
    \begin{align*}
      \min_{\bfbeta\in\mathbb{R}^{p}} \max_{\bfgamma \in \uball[*]^\eta} J_\lambda(\bfbeta,\bfgamma) 
      & \geq 
       \min_{\bfbeta\in\mathbb{R}^{p}} 
        J_\lambda\left(\bfbeta,\varphi^{*}(\bfgamma,\clS,\alpha)\right)
     \enspace,
    \end{align*}
    which concludes the proof.
  \end{proof}
\end{proposition}
Proposition~\ref{prop:monitoring} follows by choosing $\alpha={\eta}/{\norm[*]{\bfgamma}}$.

% \iffalse
%   Let $J$ be the objective function of the generic
%   Problem~\eqref{eq:robust:general:form3}, we trivially have that, for any
%   $\bfgamma$-value,
%   $\min_{\bfbeta\in\mathbb{R}^{p}} \max_{\bfgamma \in \uball[*]^\eta}
%   J(\bfbeta,\bfgamma) \geq \min_{\bfbeta\in\mathbb{R}^{p}} J(\bfbeta,\bfgamma)$.
%   %
%   Hence, a tight optimality gap can be computed provided one can guess a
%   near-optimal $\bfgamma$-value.
%   When entering Step \ref{item:algo:step3} of Algorithm~\ref{algo:active_set}, the current best guess
%   $\widehat\bfgamma_{\clS}$ has to be completed on the complement $\clS^c$.
%   We propose to do so by minimizing the magnitude of the gradient with respect to $\bfbeta$ at the current solution:
%   \begin{align*}
%     \widehat\bfgamma_{\clS^c} & = \argmin_{\bfgamma_{\clS^c}:\bfgamma \in \uball[*]^\eta}
%                                \left|  \bfX_{\centerdot\supp^c}^\intercal(\bfX_{\centerdot\supp}\bfbeta_{\supp}-\bfy) - 
%                                \lambda \bfgamma_{\supp^c} \right|
%       \enspace.
%   \end{align*}
%   The resulting vector $\widehat\bfgamma$ is then used to compute the lower
%   bound on the optimal objective function as $\min_{\bfbeta\in\mathbb{R}^{p}} J(\bfbeta,\widehat\bfgamma)$. 
%   This lower bound, which requires to solve a linear system of size $p$, is
%   much lighter to compute than the duality gap \citep[see details
%   in][]{2012_FML_Bach}, and was observed to provide a tight bound (as a sanity
%   check, note that when $\bfbeta$ is optimal, our estimated optimality gap is null).

% %

% \begin{align*}
%   \bfbeta_\supp & = \argmin_{\bfbeta\in\mathbb{R}^{|\supp|}} 
%                   \norm{\bfX_{\centerdot\supp} \bfbeta - \bfy}^2 + \lambda_1 \norm[1]{\bfbeta}
%   \\
%   \widehat\bfbeta_{\supp^c} & = \bfzero \\ 
%   \widehat\bfgamma_{\supp^c} & = \argmin_{\bfgamma_{\supp^c} \in [-1,1]^{|\supp^c|}}
%                            \left|  \bfX_{\centerdot\supp^c}^\intercal(\bfX\widehat\bfbeta-\bfy) + 
%                              \lambda_1 \bfgamma_{\supp^c} \right|
%     \enspace. 
% \end{align*}
% We can then compute 
% \begin{align*}
%   \widetilde{\bfbeta} & = \argmin_{\bfbeta\in\mathbb{R}^{p}} 
%                   \norm{\bfX \bfbeta - \bfy}^2 + \lambda_1 \bfbeta^\intercal\widehat\bfgamma
%     \enspace. 
% \end{align*}
% We then have:
% \begin{align*}
% J(\bfbeta^\star) & = \min_{\bfbeta\in\mathbb{R}^p} 
%   \norm{\bfX \bfbeta - \bfy}^2 + \lambda_1  \norm[1]{\bfbeta} \\
%   &= \min_{\bfbeta\in\mathbb{R}^p} \max_{\bfgamma \in \uball[*]^\eta}
%       \Big\{ \norm{\bfX \bfbeta - \bfy}^2 + \lambda_1 \bfbeta^\intercal\bfgamma \Big\} \\
%   &= \max_{\bfgamma \in \uball[*]^\eta} \min_{\bfbeta\in\mathbb{R}^p}
%       \Big\{ \norm{\bfX \bfbeta - \bfy}^2 + \lambda_1 \bfbeta^\intercal\bfgamma \Big\} \\
%   & \geq \min_{\bfbeta\in\mathbb{R}^p}
%       \Big\{ \norm{\bfX \bfbeta - \bfy}^2 + \lambda_1 \bfbeta^\intercal\widehat\bfgamma \Big\}  \\
%   & = \norm{\bfX \widetilde{\bfbeta} - \bfy}^2 + \lambda_1 \widetilde{\bfbeta}^{(t)^\intercal}\widehat\bfgamma
%   \enspace,
% \end{align*}
% which provides a lower bound for $J(\bfbeta^\star)$.
% Meanwhile, we obviously have:
% \begin{align*}
% J(\bfbeta^\star) & = \min_{\bfbeta\in\mathbb{R}^p} 
%   \norm{\bfX \bfbeta - \bfy}^2 + \lambda_1 \norm[1]{\bfbeta} \\
%   &\leq \norm{\bfX \widehat\bfbeta - \bfy}^2 + \lambda_1  \norm[1]{\widehat\bfbeta} 
%   \enspace,
% \end{align*}
% So that $J(\bfbeta^\star)$ is easily bracketed during the optimization process. 
% Note that when $\bfbeta_\supp=\bfbeta^\star$, we have that
% $\widetilde{\bfbeta}=\bfbeta_\supp$.



% % \ifverylong

% The additional computational cost for monitoring (that is, the cost of evaluating
% the upper and lower bounds for an intermediate solution can be important because
% $\widetilde{\bfbeta}$ requires to solve a linear system of size $p$\ldots 
% We could do better than that if we can prove that $\big(\widetilde{\bfbeta}\big)_j=0$ if
% the subgradient at  $\widehat\bfbeta_{\supp^c}$ comprises zero. In this situation, we 
% should only consider adding the variables for which the optimality conditions
% are violated.

% \subsection{Monitoring Convergence: Lasso-Case}

% \begin{align*}
%   \bfbeta_\supp & = \argmin_{\bfbeta\in\mathbb{R}^{|\supp|}} 
%                   \norm{\bfX_{\centerdot\supp} \bfbeta - \bfy}^2 + \lambda_1 \norm[1]{\bfbeta}
%   \\
%   \widehat\bfbeta_{\supp^c} & = \bfzero \\ 
%   \widehat\bfgamma_{\supp^c} & = \argmin_{\bfgamma_{\supp^c} \in [-1,1]^{|\supp^c|}}
%                            \left|  \bfX_{\centerdot\supp^c}^\intercal(\bfX\widehat\bfbeta-\bfy) + 
%                              \lambda_1 \bfgamma_{\supp^c} \right|
%     \enspace. 
% \end{align*}
% We can then compute 
% \begin{align*}
%   \widetilde{\bfbeta} & = \argmin_{\bfbeta\in\mathbb{R}^{p}} 
%                   \norm{\bfX \bfbeta - \bfy}^2 + \lambda_1 \bfbeta^\intercal\widehat\bfgamma
%     \enspace. 
% \end{align*}
% We then have:
% \begin{align*}
% J(\bfbeta^\star) & = \min_{\bfbeta\in\mathbb{R}^p} 
%   \norm{\bfX \bfbeta - \bfy}^2 + \lambda_1  \norm[1]{\bfbeta} \\
%   &= \min_{\bfbeta\in\mathbb{R}^p} \max_{\bfgamma \in \uball[*]^\eta}
%       \Big\{ \norm{\bfX \bfbeta - \bfy}^2 + \lambda_1 \bfbeta^\intercal\bfgamma \Big\} \\
%   &= \max_{\bfgamma \in \uball[*]^\eta} \min_{\bfbeta\in\mathbb{R}^p}
%       \Big\{ \norm{\bfX \bfbeta - \bfy}^2 + \lambda_1 \bfbeta^\intercal\bfgamma \Big\} \\
%   & \geq \min_{\bfbeta\in\mathbb{R}^p}
%       \Big\{ \norm{\bfX \bfbeta - \bfy}^2 + \lambda_1 \bfbeta^\intercal\widehat\bfgamma \Big\}  \\
%   & = \norm{\bfX \widetilde{\bfbeta} - \bfy}^2 + \lambda_1 \widetilde{\bfbeta}^{(t)^\intercal}\widehat\bfgamma
%   \enspace,
% \end{align*}
% which provides a lower bound for $J(\bfbeta^\star)$.
% Meanwhile, we obviously have:
% \begin{align*}
% J(\bfbeta^\star) & = \min_{\bfbeta\in\mathbb{R}^p} 
%   \norm{\bfX \bfbeta - \bfy}^2 + \lambda_1 \norm[1]{\bfbeta} \\
%   &\leq \norm{\bfX \widehat\bfbeta - \bfy}^2 + \lambda_1  \norm[1]{\widehat\bfbeta} 
%   \enspace,
% \end{align*}
% So that $J(\bfbeta^\star)$ is easily bracketed during the optimization process. 
% Note that when $\bfbeta_\supp=\bfbeta^\star$, we have that
% $\widetilde{\bfbeta}=\bfbeta_\supp$.
% % \fi
% \fi

% \section{Code-like proof}


First, rewrite code-like expression like paper-like expression:

Code-like expression:
\[
 \tilde J(\bfbeta,\tilde\bfgamma) = \frac{1}{2} \norm{\bfX \bfbeta - \bfy}^2 + 
    \frac{\lambda_2}{2} \norm{\bfbeta}^2 - 
    \bfbeta^\intercal\tilde\bfgamma 
\]
with $\norm[\infty]{\tilde\bfgamma} \leq \lambda_1$.

Paper-like expression:
\begin{align*}
  \frac{1}{2} J_{\lambda_2}(\bfbeta,\bfgamma) & = 
    \frac{1}{2} \norm{\bfX \bfbeta - \bfy}^2 + 
    \frac{\lambda_2}{2} \norm{\bfbeta - \bfgamma}^2 \\
     & = 
    \frac{1}{2} \norm{\bfX \bfbeta - \bfy}^2 + 
    \frac{\lambda_2}{2} \norm{\bfbeta}^2 - 
    \lambda_2 \bfbeta^\intercal\bfgamma + 
    \frac{\lambda_2}{2} \norm{\bfgamma}^2 
\end{align*}
with $\norm[\infty]{\bfgamma} \leq \eta_\gamma$.


We want, 
\begin{align*}
   \lambda_2 \bfgamma = \tilde\bfgamma 
   & \Rightarrow
   \eta_\gamma = \frac{\lambda_1}{\lambda_2} \\
   & \Rightarrow
   \frac{1}{2} J_{\lambda_2}(\bfbeta,\bfgamma) = \frac{1}{2} \norm{\bfX \bfbeta - \bfy}^2 + 
    \frac{\lambda_2}{2} \norm{\bfbeta}^2 - 
    \bfbeta^\intercal\tilde\bfgamma + 
    \frac{1}{2\lambda_2} \norm{\tilde\bfgamma}^2 \\
   & \Rightarrow
   \frac{1}{2} J_{\lambda_2}(\bfbeta,\bfgamma) = \tilde J(\bfbeta,\tilde\bfgamma) + \frac{1}{2\lambda_2} \norm{\tilde\bfgamma}^2
\end{align*}

Let's apply the proposition to these settings, that is: $\forall \bfgamma: \norm[\infty]{\bfgamma} \geq \eta_\gamma$,

\begin{equation*}
  \min_{\bfbeta\in\mathbb{R}^{p}} \max_{\bfgamma' \in \clD_{\bfgamma}} 
  \frac{1}{2} J_\lambda(\bfbeta,\bfgamma') 
  \geq
  \frac{1}{2} \frac{\eta_\gamma}{\norm[\infty]{\bfgamma}} 
  J_\lambda\left(\bfbeta^\star\left(\bfgamma\right), \bfgamma \right) -
  \frac{1}{2} \frac{\lambda\eta_\gamma(\norm[\infty]{\bfgamma}-\eta_\gamma)}{\norm[\infty]{\bfgamma}^2}\norm{\bfgamma}^2        
  \enspace.
\end{equation*}
It now becomes: $\forall \tilde\bfgamma: \norm[\infty]{\tilde\bfgamma} \geq \lambda_1$
\begin{align*}
  \min_{\bfbeta\in\mathbb{R}^{p}} \max_{\bfgamma\in \clD_{\bfgamma}} 
  \frac{1}{2} J_{\lambda_2}(\bfbeta,\bfgamma) 
  & \geq
  \frac{\lambda_1}{2 \lambda_2\norm[\infty]{\bfgamma}} 
  J_{\lambda_2}\left(\bfbeta^\star\left(\bfgamma\right), \bfgamma \right) -
  \frac{\lambda_1(\norm[\infty]{\bfgamma}-\lambda_1/\lambda_2)}{2\norm[\infty]{\bfgamma}^2}\norm{\bfgamma}^2
  \\
  & \geq
  \frac{\lambda_1}{\lambda_2\norm[\infty]{\bfgamma}} 
  \frac{1}{2} 
  J_{\lambda_2}\left(\bfbeta^\star\left(\bfgamma\right), \bfgamma \right) -
  \frac{\lambda_1(\lambda_2\norm[\infty]{\bfgamma}-\lambda_1)}{2 \lambda_2}\frac{\norm{\bfgamma}^2}{\norm[\infty]{\bfgamma}^2}
  \\
  & \geq
  \frac{\lambda_1}{\norm[\infty]{\tilde\bfgamma}} 
  \left( \tilde J(\bfbeta^\star\left(\tilde\bfgamma\right),\tilde\bfgamma) + \frac{1}{2\lambda_2} \norm{\tilde\bfgamma}^2 \right)
  - \frac{\lambda_1(\norm[\infty]{\tilde\bfgamma}-\lambda_1)\norm{\tilde\bfgamma}^2}{2 \lambda_2 \norm[\infty]{\tilde\bfgamma}^2}
  \\
  & \geq
  \frac{\lambda_1}{\norm[\infty]{\tilde\bfgamma}} \tilde J(\bfbeta^\star\left(\tilde\bfgamma\right),\tilde\bfgamma) + 
  \frac{\lambda_1 \norm{\tilde\bfgamma}^2}{2\lambda_2\norm[\infty]{\tilde\bfgamma}}
  - \frac{\lambda_1(\norm[\infty]{\tilde\bfgamma}-\lambda_1)\norm{\tilde\bfgamma}^2}{2 \lambda_2 \norm[\infty]{\tilde\bfgamma}^2}
  \\
  & \geq
  \frac{\lambda_1}{\norm[\infty]{\tilde\bfgamma}} \tilde J(\bfbeta^\star\left(\tilde\bfgamma\right),\tilde\bfgamma) + 
  \frac{\lambda_1 \norm{\tilde\bfgamma}^2}{2\lambda_2 \norm[\infty]{\tilde\bfgamma}}
  \left( 1 - \frac{(\norm[\infty]{\tilde\bfgamma}-\lambda_1)}{\norm[\infty]{\tilde\bfgamma}} \right)  
  \\
  \tilde J(\bfbeta^\star\left(\tilde\bfgamma^\star\right),\tilde\bfgamma^\star) 
  + \frac{1}{2\lambda_2} \norm{\tilde\bfgamma^\star}^2
  & \geq
  \frac{\lambda_1}{\norm[\infty]{\tilde\bfgamma}} \tilde J(\bfbeta^\star\left(\tilde\bfgamma\right),\tilde\bfgamma) + 
  \frac{\lambda_1^2\norm{\tilde\bfgamma}^2}{2 \lambda_2 \norm[\infty]{\tilde\bfgamma}^2}
\end{align*}
Note that $\tilde\bfgamma^\star$ has to be at bound everywhere, so that we
finally have:
\begin{align*}
  \tilde J(\bfbeta^\star\left(\tilde\bfgamma^\star\right),\tilde\bfgamma^\star) 
  + \frac{\lambda_1^{2}}{2\lambda_2} p
  & \geq
  \frac{\lambda_1}{\norm[\infty]{\tilde\bfgamma}} \tilde J(\bfbeta^\star\left(\tilde\bfgamma\right),\tilde\bfgamma) + 
  \frac{\lambda_1^2\norm{\tilde\bfgamma}^2}{2 \lambda_2 \norm[\infty]{\tilde\bfgamma}^2}
  \\
  \tilde J(\bfbeta^\star\left(\tilde\bfgamma^\star\right),\tilde\bfgamma^\star) 
  & \geq
  \frac{\lambda_1}{\norm[\infty]{\tilde\bfgamma}} \tilde J(\bfbeta^\star\left(\tilde\bfgamma\right),\tilde\bfgamma) + 
  \frac{\lambda_1^2}{2 \lambda_2} \left( \frac{\norm{\tilde\bfgamma}^2}{\norm[\infty]{\tilde\bfgamma}^2}-p\right).
\end{align*}
Note (for the Lasso/Elastic net) that when $\tilde\bfgamma\to\tilde\bfgamma^{\star}$ then
$\norm[\infty]{\tilde\bfgamma}\to\lambda_1$  and $\norm{\tilde\bfgamma}^2\to
\lambda_1^2p$, and the bound is  met with equality in that case, which
mean convergence of the algorithm.
\begin{align*}
  \tilde J(\bfbeta^\star\left(\tilde\bfgamma^\star\right),\tilde\bfgamma^\star) 
  & \geq
  \frac{\lambda_1}{\norm[\infty]{\tilde\bfgamma}} \tilde J(\bfbeta^\star\left(\tilde\bfgamma\right),\tilde\bfgamma) + 
  \frac{\lambda_1^2}{2 \lambda_2} \left(
    \frac{\lambda_{1}|\supp|}{\norm[\infty]{\tilde\bfgamma}} +
    \frac{\norm{\tilde\bfgamma_{\!\supp^c}}^2}{\norm[\infty]{\tilde\bfgamma}^2} -p
  \right).
\end{align*}





\end{document}

\endinput
%%
%% End of file `elsarticle-template-harv.tex'.

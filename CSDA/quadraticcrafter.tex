\section{Adaptive Penalties \label{sec:adaptquadra}}

\subsection{Background}

We consider the linear regression model 
\begin{equation}
  \label{eq:linear_reg_group}
  Y = X \bfbeta^\star + \varepsilon
  \enspace,
\end{equation}
where $Y$ is a continuous response variable, $X=(X_1,\dots,X_p)$ is a vector of
$p$ predictor variables, $\bfbeta^\star$ is the vector of unknown parameters and
$\varepsilon$ is a zero-mean Gaussian error variable with variance $\sigma^2$.
We will assume throughout this paper that $\bfbeta^\star$ has few non-zero
coefficients.

The estimation and inference of $\bfbeta^\star$ is based on training data,
consisting of a vector
$\mathbf{y}=(y_1,\dots,y_n)^\intercal$ for responses and a
$n\times  p$ design  matrix $\mathbf{X}$  whose $j$th  column contains
$\mathbf{x}_j  = (x_j^1,\dots,x_j^n)^\intercal$, the  $n$ observations
for variable $X_j$.  For  clarity, we assume that both $\mathbf{y}$
and $\{\mathbf{x}_j\}_{j=1,\dots,p}$ are centered so as to eliminate the
intercept from fitting criteria.

Penalization methods that build on the $\ell_1$-norm, referred to as
\emph{Lasso} procedures (Least Absolute Shrinkage and Selection Operator), are
now widely used to tackle simultaneously variable estimation and selection in
sparse problems.  They define a shrinkage estimator of the form
\begin{equation}\label{eq:general:original}
  \hat{\bfbeta} = \argmin_{\bfbeta\in\mathbb{R}^p} 
    \frac{1}{2} \norm{\bfX \bfbeta - \bfy}^2 + 
    \lambda \norm[]{\bfbeta}
  \enspace, 
\end{equation}
where $\norm[2]{\cdot}$ is the Euclidean norm and $\norm[]{\cdot}$ is an
arbitrary norm, chosen to induce some assumed sparsity pattern (typically
$\ell_1$ or $\ell_{c,1}$ norms, where $c \in (1,\infty]$).
%
% The tuning parameter $\lambda\geq  0$ controls the overall amount  of penalty.

The existence of computationally efficient optimization procedures plays an
important role in the popularity of these methods.
Though various general-purpose convex optimization solvers could be used
\citep{boyd2004convex}, exploiting the structure of the regularization problem,
and especially the sparsity of solutions, is essential in terms of computational
efficiency.
%
\citet{2012_FML_Bach} provided an overview of the families of techniques 
specifically  designed for solving this type of problems:
proximal methods, coordinate descent algorithms, reweighted-$\ell_{2}$
algorithms, working-set methods.
Stochastic gradient methods \citep{moulines2011non}, the
Frank-Wolfe algorithm \citep{lacoste2012block} or ADMM \citep[Alternating Direction
Method of Multipliers,][]{boyd2011distributed} have also recently gained in
popularity to the resolution of sparse problems.

We present below a new formulation of Problem~\eqref{eq:general:original} that
motivates an algorithm that may seem reminiscent of reweighted-$\ell_{2}$
algorithms, but which is in fact more closely related to working-set methods.
%
As for reweighted-$\ell_{2}$ algorithms, our proposal is based on the
reformulation of the sparsity-inducing penalty in terms of penalties that are
simpler to handle (linear or quadratic).  However, whereas
reweighted-$\ell_{2}$ algorithms rely on a
variational formulation of the sparsity-inducing norm that ends up in an
augmented minimization problem, our proposal is rooted in the duality principle,
eventually leading to a minimax problem that lends itself to a working-set 
algorithm that will be presented in Section~\ref{sec:algo}.

\subsection{Dual Norms}

When the sparsity-inducing penalty is a norm, its sublevel sets can always be
defined as the intersection of linear or quadratic sublevel sets.  In other
terms, if the optimization problem is written in the form of a constrained
optimization problem with inequality constraints pertaining to the penalty,
then, the feasible region can be defined as the intersection of linear or
quadratic regions. 
This fact, which is illustrated in Figures~\ref{fig:en-penalty} and
\ref{fig:group-penalty}, stems from the definition of dual norms:
%
\begin{equation*}%\label{eq:lasso_generic}
  \norm[]{\bfbeta} = \max_{\bfgamma\in\uball[*]} \bfgamma^\intercal \bfbeta
  \enspace,
\end{equation*}
where $\uball[*]$ is the unit ball centered at the origin defined from the dual
norm $\norm[*]{\cdot}$,
$\uball[*]=\left\{\bfgamma\in\mathbb{R}^p:\norm[*]{\bfgamma}\leq 1\right\}$.
Using this definition, Problem~\eqref{eq:general:original} can be reformulated
as
%
\begin{equation}\label{eq:general:primal}
  \hat{\bfbeta} = \argmin_{\bfbeta\in\mathbb{R}^p} 
  \max_{\bfgamma\in\uball[*]}
    \frac{1}{2} \norm{\bfX \bfbeta - \bfy}^2 + 
    \lambda \bfgamma^\intercal \bfbeta
  \enspace. 
\end{equation}
%
Technically, this formulation is the primal form of the original 
Problem~\eqref{eq:general:original} using the coupling function defined by the 
dual norm \citep[see e.g.][]{Gilbert16, Bonnans06}. 
It is interesting in the sense that the problem
%
\begin{equation*}
  \min_{\bfbeta\in\mathbb{R}^p} 
  \frac{1}{2} \norm{\bfX \bfbeta - \bfy}^2 + 
  \lambda \bfgamma^\intercal \bfbeta
\end{equation*}
%
is simple to solve for any value of $\bfgamma$, since it only requires solving 
a linear system.
The problem
%
\begin{eqnarray}
  \hat{\bfgamma} & = & \argmax_{\bfgamma\in\uball[*]}
    \frac{1}{2} \norm{\bfX \bfbeta - \bfy}^2 + 
    \lambda \bfgamma^\intercal \bfbeta \nonumber \\
     & = & \argmax_{\bfgamma\in\uball[*]}
    \bfgamma^\intercal \bfbeta \label{eq:optimal_gamma}
%   \enspace,
\end{eqnarray}
%
% which defines ``the worst case penalty'' in $\bfbeta$, 
is usually straightforward to solve.
Besides the sparsity of $\hat{\bfbeta}$, the overall efficiency of our algorithm
relies also on the invariance of $\hat{\bfgamma}$ with respect to
large changes in $\bfbeta$. 
For the penalties we are interested in, $\hat{\bfgamma}$ takes its value in a
finite set, defined by the extreme points of the convex polytope $\uball[*]$.
This number of points typically increases exponentially in $p$, but, with the working-set
strategy, the number of configuration actually visited typically grows linearly
with the number of non-zero coefficients in the solution $\hat{\bfbeta}$.




\subsection{Interpretation in terms of quadratic penalties}


Penalised approaches in machine learning are related to constrained optimization
problems of the form minimize $f (\bfbeta; data)$, such that $\Omega(\bfbeta)
\leq c$. This last constraint has a geometric intrepretation. The optimal solution 
belongs to the geometrical volume defined by  $\Omega(\bfbeta)\leq c$ while minimizing 
$f()$. 

In practice, Problem \eqref{eq:robust:general:primal} can rewritten as
% will be solved by
% considering an equivalent form amenable to a simpler resolution in
%$ \bfbeta$ for any $\bfgamma$, that is:
%
\begin{equation}\label{eq:general:dual}
  \min_{\bfbeta\in\mathbb{R}^p} \max_{\bfgamma \in \clD_{\bfgamma}}
    \frac{1}{2} \norm{\bfX \bfbeta - \bfy}^2 + \lambda \norm{\bfbeta - \bfgamma}^2
  \enspace,
\end{equation}
%where there is a one-to-one mapping between $\eta_X$ and $\lambda$.
%
if the norm of $\bfgamma$ is finite.  In this later case, the penalty
is quadratic, which constrains the optimal solution to belong to a ball
centered in the worst $\bfgamma$. 

This view of the geometry of the constraint as a union of balls 
illustrates the basic idea of our approach: for a fixed worst case $\bfgamma$ 
our optimization problem boils down to a simple quadratic optimization problem. 

Various classical sparse problems such as may be expressed by means of
such a quadratic penalty. The is the case for the classical Lasso,
the $\ell_{1,\infty}$ version of the group-lasso, where the
magnitude of regression coefficients are assumed to be equal within groups,
either zero or non-zero, and for OSCAR (Octagonal Shrinkage and Clustering
Algorithm for Regression) which is based on a penalizer encouraging the
sparsity of the regression coefficients and the equality of the
non-zero entries \citep{Bondell08}.  


Figure \ref{fig:penalties} illustrates  those three  sparse problems
with their  associated worst case quadratic penalty.


\ifverylong
% \subsection{More General Assumptions}
% 
% We now consider the more general assumption where the regularity of $\bfbeta^\star$ 
% is measured by the following norm:
% \begin{eqnarray*}
%  \left\| \bfbeta \right\| & = & \min_{\bftheta\in\clO} \min(-\bftheta^\intercal\bfbeta,\bftheta^\intercal\bfbeta)  \\
%  \text{with} \enspace 
%  \clO & = & \left\{ \{\bftheta_1,\ldots,\bftheta_r\}: \bftheta_i \in \Rset^p \ 
%    \text{and} \ 
%    \bfTheta = (\bftheta_1,\ldots,\bftheta_r) 
%    \ \text{is of rank $p$} \right\}
% \end{eqnarray*}
% 
\begin{figure}
  \begin{center} 
    \xylabelsquare{../figures/en_decomposition}{$\beta_1$}{$\beta_2$}{Elastic Net}% 
    \xylabelsquare{../figures/linf_decomposition}{$\beta_1$}{$\beta_2$}{$\ell_\infty$} 
    \xylabelsquare{../figures/oscar_decomposition}{$\beta_1$}{$\beta_2$}{OSCAR}%
    \caption{Penalty shapes (patches) built from the quadratic functions whose
             isocontour are displayed in white.}
    \label{fig:penalties}
    \end{center} 
\end{figure}
\fi


\subsection{Relations with Other Methods}

The expansion in dual norm expressed in Problem \eqref{eq:general:primal} bears
some similarities with the first step of the derivation of very general duality
schemes, such as Fenchel's duality or Lagrangian duality.
It is however dedicated to the category of problems expressed as in
\eqref{eq:general:original}, thereby offering an interesting novel view of this
category of problems.
In particular, it provides geometrical insights on these methods and a generic
algorithm for computing solutions.  The associated algorithm, that relies on
solving linear systems is accurate, and efficient up to medium scale problems
(thousands of variables).


% We do consider  regression problems where $\hat{\bfbeta}$ minimizes 
% \begin{equation*}
%       \hat{\bfbeta} = \argmin_{\bfbeta\in\mathbb{R}^p} \left\{ \max_{\bfgamma \in \clD_{\bfgamma}} 
%       \norm{\boldmath{X} \bfbeta - \boldmath{y}}^2 + \lambda \norm{\bfbeta - \bfgamma}^2 \right\},
% \end{equation*}
% where 
%     \begin{itemize}
%     \item $\clD_\bfgamma$ describes an uncertainty set for the parameters,
%     \item $\bfgamma$ acts as a spurious \emph{adversary} over the true $\bfbeta^\star$.
%     \end{itemize}
%  Maximizing over $\clD_\bfgamma$ leads to the worst-case formulation. 
%  Choosing  $\clD_\bfgamma$  allows  to  recover many known $\ell_1$ penalizer via $\bfgamma^\intercal \bfbeta$,
%  
% The  $\norm{\bfgamma}^2$  does not change the minimization and may be
% discarded leading to 
%      \begin{equation*}
%       \minimize_{\bfbeta\in\mathbb{R}^p} \left\{  \norm{\boldmath{X} \bfbeta - \boldmath{y}}^2
%         + \lambda \norm{\bfbeta}^2 -  \lambda \max_{\bfgamma \in \clD_{\bfgamma}} \bfgamma^\intercal \bfbeta
%        \right\},
%     \end{equation*}


\section{Assumptions on the Spurious Regression Coefficients \label{sec:quadra}}
\label{sec:gammaperturb}

Our framework is amenable to many variations.
Here, we simply present two examples following the same pattern:
assuming a given regularity on the regression coefficients $\bfbeta^\star$, we
consider the adversarial dual assumption on the spurious coefficients
$\bfgamma$.
When the initial regularity conditions on $\bfbeta^\star$ are expressed by
$\ell_1$ or $\ell_\infty$ norms, this process results in uncertainty sets
$\mathcal{D}_{\bfgamma}$ which are convex polytopes that are
easy to manage when solving Problem~\eqref{eq:general:dual}, since they
can be defined as the convex hulls of a finite number possible perturbations.

% In this paper, we restrict our examples to 
The two sparsity-inducing penalizers presented below have a grouping effect.
The elastic net implements this grouping without predefining the group
structure: strongly correlated predictors tend to be in or out of the model
together \citep{2005_JRSS_Zou}.  
The $\ell_{\infty,1}$ group-Lasso that is presented subsequently is based on a
prescribed group structure and favors regression coefficients with identical
magnitude within activated groups.

\subsection{Elastic-Net} \label{sec:elasticnet}

As an introductory example, let us consider the assumption stating that the
$\ell_1$-norm of $\bfbeta^\star$ should be small. 
The dual norm is the $\ell_\infty$-norm:
%
\begin{align*}
  \uball[*]^\eta & = \left\{ \bfgamma \in \mathbb{R}^p :
\sup_{\norm[1]{\bfbeta}\leq1} \bfgamma^\intercal\bfbeta \leq \eta \right\} \\
    & = \left\{ \bfgamma \in \mathbb{R}^p : \norm[\infty]{\bfgamma} \leq \eta \right\} \\
    & = \mathbf{conv} \big\{ \left\{ -\eta, \eta \right\}^p \big\}
  \enspace,
\end{align*}
where $\mathbf{conv}$ denotes convex hull, so that Problem
\eqref{eq:general:dual} reads:
%
\begin{align}
  & \min_{\bfbeta\in\mathbb{R}^p} \max_{\bfgamma \in\uball[*]^\eta}
      \Big\{ \frac{1}{2} \norm{\bfX \bfbeta - \bfy}^2 + \frac{\lambda}{2} \norm{\bfbeta - \bfgamma}^2 
      \Big\} \nonumber \\
  \Leftrightarrow
    & \min_{\bfbeta\in\mathbb{R}^p}
       \frac{1}{2} \norm{\bfX \bfbeta - \bfy}^2 + \lambda \eta
       \norm[1]{\bfbeta} + \frac{\lambda}{2} \norm{\bfbeta}^2
  \enspace, \label{eq:elastic-net}
\end{align}
%
which is recognized as an elastic-net problem.
When $\eta$ is null, we recover ridge regression, and when $\eta$ goes to 
infinity, the problem approaches a Lasso problem.
A 2D pictorial illustration of this evolution is given in
Figure~\ref{fig:en-penalty}, where the shape of the uncertainty set
$\uball[*]^\eta$ is the convex hull of the points located at 
$(\pm \eta, \pm \eta)^\intercal$, which are identified by the cross markers.
Then, the sublevel set 
$\{\bfbeta : \max_{\bfgamma \in \uball[*]^\eta} \norm{\bfbeta-\bfgamma}^2 \leq t\}$
is simply defined as the intersection of the four sublevel sets
$\{\bfbeta : \norm{\bfbeta - \bfgamma}^2 \leq t\}$ for $\bfgamma=(\pm
1, \pm 1)^\intercal$, which are Euclidean balls centered at
these $\bfgamma$ values.
%
\begin{figure}
  \begin{center} 
    \smallxylabelsquare{../figures/en_decomposition1}{$\beta_1$}{$\beta_2$}{}%
    \smallxylabelsquare{../figures/en_decomposition2}{$\beta_1$}{$\beta_2$}{}%
    \smallxylabelsquare{../figures/en_decomposition3}{$\beta_1$}{$\beta_2$}{}%
    \smallxylabelsquare{../figures/en_decomposition4}{$\beta_1$}{$\beta_2$}{}%
    \caption{Sublevel sets for elastic net penalties (represented by the darker
             colored patches).  
             Each set is defined as the intersection of the the Euclidean balls
             (represented by the lighter color patches) whose centers are
             represented by crosses.}
    \label{fig:en-penalty}
    \end{center} 
\end{figure}

 

% \subsection{Pairwise Fused Lasso}

The sparsity inducing penalties can be adapted to pursue different goals, such
as having equal coefficients.  This was first implemented for ordered features
with the fused Lasso \citep{Tibshirani05}, which encourages sparse and locally
constant solutions by penalizing the $\ell_1$-norm of both the coefficients and
their successive differences.

The pairwise fused Lasso \citep{Petry11} does not assume that predictors are
ordered.  It selects features and favors some grouping by penalizing the
$\ell_1$-norm of both the coefficients and the differences between all pairs,
thus considering the following hypothesis space for
$\bfbeta^*$:
%
\begin{equation*}
  \mathcal{\clH}^\mathrm{PFL}_{\bfbeta^*} = \left\{ \bfbeta \in \mathbb{R}^p : 
    \norm[1]{\bfbeta}  + c
    \sum_{j=1}^{p} \sum_{k<j} \left| \beta_{j} - \beta_k \right| \leq \eta_\beta
  \right\}
  \enspace,
\end{equation*}
%
whose dual set is:
%
\begin{align*}
  \mathcal{D}^\mathrm{PFL}_{\bfgamma} & = 
    \left\{ \bfgamma \in \mathbb{R}^p : 
            \sup_{\bfbeta \in \mathcal{\clH}^\mathrm{PFL}_{\bfbeta^*}}
            \bfgamma^\intercal\bfbeta \leq 1 
    \right\}
    \\
    & = \left\{ ? \right\}
  \enspace,
\end{align*}
{\color{red}{YG: I believe that the definition of $\bfgamma$ is wrong for the fused Lasso}}
\begin{equation*}
  \bfD = 
  \begin{pmatrix}
      1      &  0     & 0      & \cdots & 0\\
     -1      &  1     & 0      & \ddots & \vdots  \\
      0      & -1     & 1      & \ddots & 0 \\
     \vdots  & \ddots & \ddots & \ddots & 0 \\
      0      & \cdots & 0      & -1     & -1 
  \end{pmatrix}
  \enspace,
\end{equation*}
%
so that Problem \eqref{eq:robust:general:form3} reads:
%
\begin{align*}
  & \min_{\bfbeta\in\mathbb{R}^p} \max_{\bfgamma_1 \in \left\{ -\eta_\gamma, \eta_\gamma \right\}^p}
    \max_{\bfgamma_2 \in \left\{ -\nu_\gamma, \nu_\gamma \right\}^{p-1}}
      \Big\{ \norm{\bfX \bfbeta - \bfy}^2 + \lambda \norm{\bfbeta - \bfgamma_1 - \bfD \bfgamma_2}^2 \Big\} \\
  \Leftrightarrow
    & \min_{\bfbeta\in\mathbb{R}^p}
      \norm{\bfX \bfbeta - \bfy}^2 + \lambda \eta_\gamma \norm[1]{\bfbeta} + \lambda \norm{\bfbeta}^2 
  \enspace,
\end{align*}
%


The Lagrangian formulation of the fused Lasso optimization problem is expressed
as:
\begin{equation}
  \min_{\bfbeta\in\mathbb{R}^p} \norm{\bfX \bfbeta - \bfy}^2 + 
    \lambda \sum_{j=1}^p \left|\beta_{j}\right| + \theta \sum_{j=1}^{p-1} \left| \beta_{j+1} - \beta_j \right|
  \enspace, 
\end{equation}


\subsection{Group-Lasso}

We consider here the $\ell_{\infty,1}$ variant of the group-Lasso, which was
first proposed by \citet{Turlach05} to perform variable selection in the
multiple response setup.
%
A group structure is defined on
the set of variables by setting a partition of the index set
$\mathcal{I}=\{1,\ldots,p\}$, that is,
$
  \mathcal{I}=\bigcup_{k=1}^K\group \enspace,\, \text{with}\enspace 
  \group \cap \group[\ell]=\emptyset \enspace
  \text{for}\enspace k\neq\ell \enspace.
$
%
Let $p_k$ denote the cardinality of group $k$, and $\bfbeta_{\group} \in
\Rset^{p_k}$ be the vector $(\beta_j)_{j\in \group}$.
%


The $\ell_{\infty,1}$ mixed-norm of $\bfbeta$ (that is, its groupwise 
$\ell_\infty$-norm) is defined as:
%
\begin{equation*}
  \uball = \left\{ 
    \bfbeta \in \mathbb{R}^p :\sum_{k=1}^K \norm[\infty]{\bfbeta_{\group}} \leq 1
  \right\}
  \enspace.
\end{equation*}
%
The dual norm is the groupwise $\ell_1$-norm:
%
\begin{align*}
  \uball[*]^\eta & = \left\{ \bfgamma \in \mathbb{R}^p :
\sup_{\bfbeta\in\uball} \bfgamma^\intercal\bfbeta \leq \eta \right\} \\
    & = \left\{ \bfgamma \in \mathbb{R}^p : \max_{k\in\{1,...,K\}}  \norm[1]{\bfgamma_{\group}} \leq \eta \right\} \\
    & = \mathbf{conv} \big\{ 
                        \left\{\eta\bfe^{p_1}_1, \ldots, \eta\bfe^{p_1}_{p_1},-\eta\bfe^{p_1}_1, \ldots, -\eta\bfe^{p_1}_{p_1} \right\} 
                        \times \ldots \\
    & \hspace*{4em} \times 
                        \left\{\eta\bfe^{p_K}_1, \ldots, \eta\bfe^{p_K}_{p_K},-\eta\bfe^{p_K}_1, \ldots, -\eta\bfe^{p_K}_{p_K} \right\} 
                      \big\}
  \enspace,
\end{align*}
so that Problem \eqref{eq:general:dual} becomes:
%
\begin{align*}
  & \min_{\bfbeta\in\mathbb{R}^p} \max_{\bfgamma \in \uball[*]^\eta}
      \Big\{ \frac{1}{2} \norm{\bfX \bfbeta - \bfy}^2 + \frac{\lambda}{2} \norm{\bfbeta - \bfgamma}^2 \Big\} \\
  \Leftrightarrow
    & \min_{\bfbeta\in\mathbb{R}^p}
      \frac{1}{2} \norm{\bfX \bfbeta - \bfy}^2 + \lambda \eta \sum_{k=1}^K \norm[\infty]{\bfbeta_{\group}} + \frac{\lambda}{2} \norm{\bfbeta}^2 
  \enspace,
\end{align*}
%

Notice that the limiting cases of this penalty are two classical problems: ridge
regression and the $\ell_{\infty,1}$ group-Lasso.
A 2D pictorial illustration of this evolution is given in
Figure~\ref{fig:group-penalty}, where the shape of the uncertainty set
$\mathcal{D}_{\bfgamma}$ is the convex hull of the points located on the axes at $\pm
\eta$, which are identified by the cross markers.
Then, the sublevel set 
$\{\bfbeta : \max_{\bfgamma \in \uball[*]^\eta} \norm{\bfbeta-\bfgamma}^2 \leq t\}$
is simply defined as the intersection of the four sublevel sets
$\{\bfbeta : \norm{\bfbeta - \bfgamma}^2 \leq t\}$ for 
$\bfgamma=\pm \eta\bfe^{2}_1$ and $\bfgamma=\pm \eta\bfe^{2}_1$,
which are Euclidean balls centered at these $\bfgamma$ values.
%
\begin{figure}
  \begin{center} 
    \smallxylabelsquare{../figures/linf_decomposition1}{$\beta_1$}{$\beta_2$}{}%
    \smallxylabelsquare{../figures/linf_decomposition2}{$\beta_1$}{$\beta_2$}{}%
    \smallxylabelsquare{../figures/linf_decomposition3}{$\beta_1$}{$\beta_2$}{}%
    \smallxylabelsquare{../figures/linf_decomposition4}{$\beta_1$}{$\beta_2$}{}%
    \caption{Sublevel sets for the $\ell_{\infty,1}$ group-Lasso penalties
             (represented by the darker colored patches).
             Each set is defined as the intersection of the the Euclidean balls
             (represented by the lighter color patches) whose centers are
             represented by crosses.}
    \label{fig:group-penalty}
    \end{center} 
\end{figure}

\iffalse
The  lagrangian formulation  of the  $\ell_{\infty,1}$ version  of the
group-Lasso as a constrained optimization can be expressed as
$$
 \min_{\bfbeta}     \norm{\bfX  \bfbeta  - \bfy  }^2+\lambda \sum_{g=1}^G \|\bfbeta_g\|_\infty,
$$
with $\bfbeta_g=(\beta_{gk})$ the coefficients of $\bfbeta$ corresponding to group $g$.


The penalty term can be expressed in a form close to our adverse quadratic penalty. Let us consider  the adverse vector domain to be
$$
 \clD_{\bfgamma}=\left\{ \bfgamma \in \mathbb{R}^p | \bfgamma= 
(\alpha_j \mathbb{I}_{(\mbox{rank}\left(  \max_k |\beta|_{gk}\right)=j) })_{j=1
      \cdots p } 
\norm[\infty]{(\alpha_1,\cdots,\alpha_p)} \leq \eta_\alpha \right\} .
$$

We can reformulate the previous lagrangian as 
\begin{equation}
    \min_{\bfbeta\in\mathbb{R}^p} \max_{\bfgamma \in \clD_{\bfgamma}}
    \norm{\bfX \bfbeta - \bfy } + \lambda \norm{\bfbeta +
    \bfgamma } \enspace.
\end{equation}

This  rewriting  of the  problem  allows to  see  that  the very  same
optimization adaptive  constraint algorithm  used for the  elastic net
can be used to solve the Group  $\ell_{\infty,1}$ problem.


{\color{red}{YG : Could'nt we derive the usual group-Lasso from the robust
optimization viewpoint, simply by changing the definition of $\clD_{\bfX}$, with
groupwise Frobenius norms?}

Christophe: Si  tu fais cela tu écris  bien le group Lasso  mais je ne
vois pas comment cela permet de faciliter la résolution du problème avec les mêmes
techniques que pour les autres....
}

{\color{blue}{Christophe:  une version group  oscar devrait  donner de
    meilleurs résultats}}
\fi

%\paragraph{OSCAR}

The sparsity inducing penalties can be adapted to pursue different goals, such
as having equal coefficients.  This was first implemented for ordered features
with the fused Lasso \citep{Tibshirani05}, which encourages sparse and locally
constant solutions by penalizing the $\ell_1$-norm of both the coefficients and
their successive differences.

Even when there is no ordering between features, equality can be desired for
interpretability purposes.  OSCAR (Octagonal Shrinkage and Clustering Algorithm
for Regression) has been conceived in this siprit, to infer clusters of
variables in a supervised setting \citep{Bondell08}.
It is based on a penalizer encouraging the sparsity of the regression
coefficients and the equality of the non-zero entries.
By this means, correlated predictors that have a similar effect on the response
form ``predictive clusters'' represented by a single coefficient.

% %
% \begin{equation*}
%   \clH^\mathrm{oscar}_{\bfbeta^*} = \left\{ 
% % \bfbeta \in \mathbb{R}^p : \norm[1]{\bfbeta} \leq \eta_\beta 
% \right\}
%   \enspace.
% \end{equation*}
% %
% The dual assumption is that the $\ell_\infty$-norm of $\bfgamma$ should be
% controlled, say:
% %
% \begin{align*}
%   \mathcal{D}^\mathrm{Oscar}_{\bfgamma} & = \left\{ 
% % \bfgamma \in \mathbb{R}^p :
% % \sup_{\bfbeta\in\clH^\mathrm{Lasso}_{\bfbeta^*}} \bfgamma^\intercal\bfbeta \leq 1 
% \right\} \\
% %     & = \left\{ \bfgamma \in \mathbb{R}^p : \norm[\infty]{\bfgamma} \leq \eta_\gamma \right\} \\
% %     & = \mathbf{conv} \big\{ \left\{ -\eta_\gamma, \eta_\gamma \right\}^p \big\}
%   \enspace,
% \end{align*}
% where $\eta_\gamma$ is defined from $\eta_\beta$ and $\mathbf{conv}$ denotes convex hull, so that Problem
% \eqref{eq:robust:general:form3} reads:
% %
% \begin{align*}
%   & \min_{\bfbeta\in\mathbb{R}^p} \max_{\bfgamma \in \left\{ -\eta_\gamma, \eta_\gamma \right\}^p}
%       \Big\{ \norm{\bfX \bfbeta - \bfy}^2 + \lambda \norm{\bfbeta - \bfgamma}^2 \Big\} \\
%   \Leftrightarrow
%     & \min_{\bfbeta\in\mathbb{R}^p}
%       \norm{\bfX \bfbeta - \bfy}^2 + 2 \lambda \eta_\gamma \norm[1]{\bfbeta} + \lambda \norm{\bfbeta}^2 
%   \enspace,
% \end{align*}

The lagrangian formulation of  OSCAR as a constrained optimization can be expressed as
$$
 \min_{\bfbeta}     \norm{\bfX  \bfbeta  - \bfy  }^2+\lambda \sum_{j=1}^p \left( c(j-1)+1 \right) |\beta|_{(j)},
$$
with $|\beta|_{(1)}\leq |\beta|_{(2)} \leq \cdots \leq |\beta|_{(p)}$. The penalty term can be expressed in a form close to our adverse quadratic penalty. Let us consider  the adverse vector domain to be
$$
 \clD_{\bfgamma}=\left\{ \bfgamma \in \mathbb{R}^p | \bfgamma= 
\begin{pmatrix}
\alpha_1 1\\
\alpha_2 (c+1) \\
\alpha_3 (2c+1) \\
\vdots \\
\alpha_p (p-1) c+1
\end{pmatrix}, \ c\in \mathbb{R}^+ , 
\norm[\infty]{(\alpha_1,\cdots,\alpha_p)} \leq \eta_\alpha \right\}
$$
and the permutation matrix
$$
 P_{\bfbeta} = \left\{ \mathbb{I}_{(\mbox{rank}\left( |\beta|_{(i)}=j \right)} \right\}_{i = 1 \cdots p, j=1 \cdots p}.
$$

We can reformulate the previous lagrangian as 
\begin{equation*}
    \min_{\bfbeta\in\mathbb{R}^p} \max_{\bfgamma \in \clD_{\bfgamma}}
    \norm{\bfX \bfbeta - \bfy } + \lambda \norm{\bfbeta +
     P_{\bfbeta}  \bfgamma } \enspace.
\end{equation*}

This rewriting of the initial problem  allows to see that the very same
optimization adaptive  constraint algorithm  used for the  elastic net
can be used to solve the OSCAR problem (Illustrated in Figure \ref{figure:oscar}).


\begin{figure}
  \begin{center} 
    \smallxylabelsquare{../figures/oscar_decomposition1}{$\beta_1$}{$\beta_2$}{}%
    \smallxylabelsquare{../figures/oscar_decomposition2}{$\beta_1$}{$\beta_2$}{}%
    \smallxylabelsquare{../figures/oscar_decomposition3}{$\beta_1$}{$\beta_2$}{}%
    \smallxylabelsquare{../figures/oscar_decomposition4}{$\beta_1$}{$\beta_2$}{}%
    \caption{Admissible sets (patches) for the OSCAR, defined by the
      intersection of the Euclidean balls whose centers are represented by
      crosses and boundaries are
      displayed in black.} \label{figure:oscar}
    \label{fig:oscar-penalty}
    \end{center} 
\end{figure}






%
% LASSO
% RIDGE
% ELASTIC NET
% STRUCTURE ELASTIC NET
% FUSED
% OSCAR